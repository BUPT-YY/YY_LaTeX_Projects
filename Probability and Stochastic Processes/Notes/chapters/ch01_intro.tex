%\input tit_xelatex_ctex.tex
%\XeTeXinputencoding "GBK"

\setcounter{chapter}{-1}
\chapter{预备知识}

%\input zzsetpage

\section{概率论评述}

A、起源

概率论这门学科可以说起源于赌博。尽管早在15世纪与16世纪意大利的一些数学家(如Cardano,Pacioli,Tartaglia等)已经对一些靠运气的游戏中的特定概率进行了计算,但是,概率论作为一门学科起源于17世纪。
1654年,一个名叫A.G.C. de Méré的法国贵族对赌博以及赌博中的问题很感兴趣,但他对一些问题感到很困惑,为了解决自己的困惑,他向数学家B.Pascal(1623-1662)求助。为了解答de Méré提出的问题,Pascal与法国数学家P.Fermat(1601-1655)进行了通信讨论。1655年,荷兰数学家C.Huygens(1629-1695)首次访问巴黎,期间他学习了Pascal与Fermat关于概率论的工作。Huygens是一个名声和Newton相当的大科学家. 人们熟知他的贡献之一是物理中的单摆公式. 他在概率论的早期发展历史上也占有重要的地位. 1657年,当他回到荷兰后,他写了一本小册子,名叫《De Ratiociniis in Ludo Aleae》(可译为《机遇的规律》),这是关于概率论的第一本书。在这部著作中, 他首先引进了"期望"这个术语.在此,"数学期望"这个基本概念以及关于概率的可加性、可乘性已经建立。他基于这些术语解决了一些当时感兴趣的博弈问题. 他在这部著作中提出了14条命题, 第一条命题是:

如果某人在赌博中以一半的可能性赢$a$元,以一半的可能性输$b$元, 则他的期望是\begin{equation}
	\frac{a-b}{2} (\text{元}).
\end{equation}

B、18-19世纪

概率论在18世纪得到了快速发展,这个期间的主要贡献者是J.Bernoulli(1654-1705)和A. de Moivre(1667-1754)。在19世纪,概率论的早期理论得到了进一步的发展与推广,这个期间的主要贡献者是P.S.M. Laplace(1749-1827),S.D. Poisson(1781-1840),C.F. Guass(1777-1855),P.L. Чебышёв(切比雪夫)(1821-1894), 马尔可夫(A.A.Марков)(1856-1922)与李雅普诺夫(A.M.Ляпуно́в)(1857-1918)。这个时期的研究主要围绕着概率极限定理展开。


Jacob Bernoulli是一位瑞士数学家,是Bernoulli家族的第一位数学家。 1705年, 他在瑞士Basel去世. 8年后,也就是1713年,他在概率论领域的代表作《Ars Conjectandi》一书正式出版(可译为《猜测的艺术》).他的这本书正式出版标志着概率论学科的开始. 该书仅包含一个数学定理,即著名的Bernoulli大数定律,这是概率论的第一个极限定理.

Bernoulli大数定律:假设$\{\xi_n\}$为一个独立同分布的随机变量序列,$P(\xi_n = 1) = p, P(\xi_n = 0) = 1-p$, 其中$0<p<1$. 令$S_n = \sum_{k=1}^{n}\xi_k$, 那么
\begin{equation}
	P\left(\omega:\left|\frac{S_n(\omega)}{n}-p\right|>\varepsilon \right)\to 0,~~n\to\infty.
\end{equation}
17世纪下半叶, Newton刚发明微积分不久, 人们对计算各种数列的极限有着相当大的兴趣, 并发展了不少有效的方法和技巧. 但是, "$\varepsilon-N$"语言并不能很好地解释"随机试验中频率是否收敛到概率"这样的问题.正是Bernoulli大数律首次给出了"频率收敛到概率"的数学解释和严格证明.

Bernoulli大数律内涵丰富,成为了后人发展概率论的源泉. (0.2)式充分肯定了经验观测可以揭示随机现象规律的基本思想; 提出了随机变量序列依概率收敛到常数(甚至收敛到随机变量)的基本概念. 受此启发,人们自然会问:如果考虑一般的随便变量的平均值, 情况会如何? 特别,假设$\{\xi_n\}$为一个独立同分布的随机变量序列,$\mathrm{E}\xi_n = \mu$.令$S_n = \sum_{k=1}^{n}\xi_k$, 那么\begin{equation}
P\left(\omega:\left|\frac{S_n(\omega)}{n}-p\right|>\varepsilon \right)\to 0,~~n\to\infty
\end{equation}成立么?

用观测平均值去计算真值的思想很早以前就已出现,并一直用于日常生活和社会实践,关键在于能否给出一个严格的数学证明吗?

回顾(0.2)式的证明,Bernoulli二项分布起着关键作用.事实上,$S_n\sim \mathcal{B}(n,p)$.因此
\begin{equation}
	P\left(\omega:\left|\frac{S_n(\omega)}{n}-p\right|>\varepsilon \right)	= \sum_{k:|k-np|>n\varepsilon}\binom{n}{k}p^k(1-p)^{n-k}.	
\end{equation}

利用杨辉三角或者Pascal二项组合式,容易得到\begin{equation}
	\sum_{k=0}^n\binom{n}{k}p^k(1-p)^{n-k} = 1.	
\end{equation}

但是,(0.4)的困难在于计算部分和,而不是对所有$k=0,1,\cdots,n$求和. 为此, Bernoulli利用了$n!$的渐进计算公式. 显然,这样一个计算技巧对(0.3)不合适, 因为$\xi_n$的分布并不知道. 事实上,即使$\xi_n$只取三个值, Bernoulli的计算方法仍然显得笨拙而不可行!

为了证明(0.3), 需要用到下面的 Чебышёв(切比雪夫)不等式:假设$X$是随机变量, $\mathrm{E}X = \mu,\mathrm{var}(X) = \sigma^2$,那么对任意$x>0$
\begin{equation}
	P(|X-\mathrm{E}X|\geqslant x)\leqslant \frac{\mathrm{var}(X)}{x^2}.
\end{equation}

Чебышёв(切比雪夫)被看作俄国现代数学之父.上述切比雪夫(Чебышёв)不等式在概率论学科发展中起着举足轻重的作用.事实上,概率自16世纪由赌博游戏引入以来,到19世纪末,历经300年.尽管由不少的大数学家热衷于概率的研究并积极倡导和应用,但相对于这一时期的分析、代数、几何等其它数学分支而言,概率论的发展可以说是十分缓慢, 并且, 在总体上来说, 概率论学科还停留在一些具体事件的概率计算上.  切比雪夫(Чебышёв)不等式可以看作一个时代的转折点.

作为应用,可以用(0.6)证明(0.2):
\begin{equation}
	P\left(\omega:\left|\frac{S_n(\omega)}{n}-p\right|>\varepsilon \right)
	\leqslant \frac{\mathrm{var}(S_n/n)}{\varepsilon^2} = \frac{\mathrm{var}(S_n)}{n^2\varepsilon^2}.
\end{equation}
既然$\varepsilon>0$是任意给定的正数,那么只要验证$\mathrm{var}(S_n) = o(n^2)$就够了.在二项分布情形,
\begin{equation}
	\mathrm{var}(S_n) = \sum_{k=0}^n(k-np)^2\binom{n}{k}p^k(1-p)^{n-k} = np(1-p).
\end{equation}
注意,正如(0.5)一样,(0.8)的计算比(0.4)要容易得多.更为重要的是,上述讨论不局限于Bernoulli二项分布,而适用于非常广泛的随机变量序列.以下是切比雪夫(Чебышёв)大数定律:
假设$\{\xi_n\}$为一个随机变量序列,$\mathrm{E}\xi_n = \mu_n, \mathrm{var}(\xi_n) = \sigma^2$. 如果\begin{equation}
	\mathrm{var}(S_n) = o(n^2),
\end{equation}
那么\begin{equation}
	P\left(\omega:\left|\frac{S_n(\omega)}{n}-\frac{1}{n}\sum_{k=1}^n\mu_k\right|>\varepsilon \right)\to 0,~~n\to\infty.
\end{equation}
该定理的条件(0.9)包含的范围非常广泛. 如独立同分布且方差有限; 独立不同分布且方差有界; 不独立但互不相关且方差有界; 其它相依情形.后几种情形更为常见, 在实际观测时, 并不能苛求试验环境, 数据之间不可避免地存在一定联系, 并且不能保证同分布.

切比雪夫(Чебышёв)大数律显然是Bernoulli大数律的极大推广.但是条件(0.9)对于一个大数律来说无疑有点强. Khinchin改进了切比雪夫(Чебышёв)大数律, 在独立同分布且数学期望存在有限的情况下,证明了(0.3).当然,此时无法直接使用切比雪夫(Чебышёв)不等式(0.6), 截尾方法应运而生.

Bernoulli大数律告诉人们: 给定任意精度, 只要试验重复足够多次, 频率就有很大可能接近概率真值, 以致误差在给定精度内. 人们自然会问, 试验次数究竟多大合适? 该如何确定呢? 当然, 给定精度$\varepsilon>0$, 无论$n$多么大, 都无法保证独立重复试验后$|S_n/n-p|\leqslant\varepsilon$. $n$的大小取决于事先给定的可靠度(置信度), 关键在于如何由精度和可靠度来确定$n$. 即, 给定$\varepsilon>0,\eta<1$, 如何有效地表达(近似)\begin{equation}
	P\left( \omega:\left| \frac{S_n(\omega)}{n}-p \right|>\varepsilon \right) = \eta.
\end{equation}

de Moivre和Laplace考虑了上面的问题, 并对独立二点分布的随机变量序列证明了以下中心极限定理: 对任意实数$a<b$,\begin{equation}
	P\left( \omega:a<\frac{S_n(\omega)-np}{\sqrt{np(1-p)}} \leqslant b  \right) \to \Phi(b)-\Phi(a).
\end{equation}
直接应用该结果, (0.11)可写成\begin{equation}
	P\left( \omega:\left| \frac{S_n(\omega)}{n}-p \right|>\varepsilon \right) \approxeq 2\left( 1-\Phi\left(\varepsilon\sqrt{\frac{n}{p(1-p)}}\right) \right)=: \eta.
\end{equation}
由此可计算$n$的大小(依赖于$\varepsilon,n,p$).

de Moivre是一位法国数学家,但是大部分时间他住在英国. de Moivre在前人, 特别是Bernoulli家族和Huygens的基础上, 研究和发展概率论,可以说, 他开创了概率论的现代方法:1711年出版了《The Doctrine of Chances》.在此书中,统计独立性的定义首次出现。该书在1738年与1756年出了扩版,生日问题出现在1738年的版本中,赌徒破产模型出现在1756年的版本中, 并在赌徒中很有影响和地位.
1730年,de Moivre的另一本专著《Miscellanea Analytica Supplementum》(可译为《解析方法》)正式出版。其中,关于对称Bernoulli试验的中心极限定理首次提出并得到证明。
他首先考虑了$p=1/2$情形,并和他的好基友Stirling同时发现了下列公式\begin{equation}
	n!\sim \sqrt{2\pi n}n^ne^{-n}.
\end{equation}
通常称(0.14)为Stirling公式, 实际上应该为de Moivre-Stirling公式.

差不多40年后, Laplace考虑了$p\neq 1/2$情形. 应该说, Laplace对概率统计和天体力学的贡献巨大. 他在1799-1825年间出版了五卷本《Celestial Mechanics》.
1812年,Laplace的伟大专著《Théorie Analytique des Probabilitiìés》(可译为《概率论的解析理论》)诞生,其中,他阐述了他自己及前辈在概率论方面的结果。特别地,他将de Moivre的定理推广到Bernoulli试验非对称情形。Laplace最重要的工作是将概率方法应用到观测误差,在很一般的条件下证明了观测误差的分布一定是渐进正态的。直到今天, 人们在概率极限理论方面的研究还受到Laplace的影响.

正如Bernoulli大数定律一样, de Moivre-Laplace中心极限定理对概率论学科的发展影响深远,受(0.12)启发, 人们提出了随机变量序列依分布收敛的概念, 并给出了一般形式的中心极限定理:假设$\xi_1,\xi_2,\cdots$是一独立同分布的随机序列,$\mathrm{E}\xi_n = \mu,\mathrm{var}(\xi_n) = \sigma^2$, 那么
\begin{equation}
	P\left( \omega:\frac{S_n(\omega)-n\mu}{\sqrt{n}\sigma}\leqslant x \right) \to \Phi(x),~~~n\to\infty.
\end{equation}
但是, 如何证明呢?我们需要找到一个有效的工具和判别准则. 随着调和分析的发展, 人们发现Fourier变换是研究分布函数的一个有效工具. 在概率论学科中, 大家称分布函数的Fourier变换为特征函数.

假设$X$是个随机变量, 定义$\phi(t) = \mathrm{E}\mathrm{e}^{itX}$为其特征函数(c.f.). 任意的随机变量都存在特征函数, 并且它具有非常良好的分析性质、运算性质和唯一性. 随着特征函数的引入, 20世纪20-30年代概率论的发展进入了一段黄金时期. 法国数学家Lévy建立了连续性定理:$X_n\stackrel{d}{\rightarrow}X$当且仅当$\phi_n(t)\to\phi(t)$. 其中, $\phi_n$和$\phi$分别为$X_n$和$X$的c.f.. 由此, 可以证明(0.15). 事实上, 令$X_n = \frac{S_n-n\mu}{\sqrt{n}\sigma}$, 那么
\begin{equation}
        \mathrm{E}\mathrm{e}^{itX_n} =\prod_{k=1}^n\mathrm{E}\mathrm{e}^{it\frac{\xi_k-\mu}{\sqrt{n}\sigma}}
        =\left( 1-\frac{t^2}{2n}+o(1/n) \right)^n
        \to \mathrm{e}^{-t^2/2}.
\end{equation}
所以,(0.15)成立. 这被称为Feller-Lévy中心极限定理, 这个证明已经写入许多的本科生概率论教材, 具有微积分基础的同学们都能掌握.

Lindeberg和Feller研究了独立不同分布情形, 他们再次运用了特征函数的方法证明了下列定理: 假设$\{\xi_n;n\geqslant 1\}$为一独立的随机序列, 相应的分布函数分别为$F_n$, 并且$\mathrm{E}\xi_n = \mu_n,~\mathrm{var}(\xi_n) = \sigma^2_n<\infty$. 若$B_n^2 = \sum_{k=1}^{n}\sigma_k^2\to\infty$. 那么
\begin{equation}
    \max_{1\leqslant k\leqslant n} \frac{\sigma^2}{B_n^2}\to 0
\end{equation}
和\begin{equation}
    \frac{1}{B_n}\sum_{k=1}^n(\xi_k-\mu_k)\stackrel{d}{\rightarrow}\mathcal{N}(0,1)
\end{equation}
成立当且仅当对任意$\varepsilon>0$,
\begin{equation}
    \frac{1}{B_n^2}\sum_{k=1}^n\int_{\abs{x-\mu_k}>\varepsilon B_n}(x-\mu_k)^2\mathrm{d}F_k(x)\to 0.
\end{equation}
称(0.17)为Feller条件, (0.19)为Lindeberg条件.  当随机变量同分布且方差都存在时, 这些条件都满足. 当随机变量不同分布时, (0.17)意味着各个随机变量$\xi_k/B_n$"一致地无穷小, 没有一个起显著作用". 正如切比雪夫(Чебышёв)大数律一样, Lindeberg-Feller定理应用非常广泛, 譬如说各类测量误差可以近似地用正态分布描述.

中心极限定理不仅适用于独立随机变量的部分和, 而且可以推广到许多相依随机变量序列情形, 如鞅差序列,马尔可夫(Марков)链, 各类混合序列, 正、负相依(伴)序列等, 从而发展了许多新的方法, 如20世纪70年代提出的Stein方法.



在当代概率论中,与Poisson相关的有Poisson分布、Poisson过程。Guass创立了误差理论,特别地,创立了最小二乘的基本方法。切比雪夫(Чебышёв),马尔可夫(Марков)与李雅普诺夫(Ляпуно́в)在研究独立但不同分布的随机变量和的极限定理方面发展了有效的方法。
在切比雪夫(Чебышёв)之前,概率论的主要兴趣在于对随机试验的概率进行计算。而切比雪夫(Чебышёв)是第一个清晰认识并充分利用随机变量及其数学期望的人。切比雪夫(Чебышёв)思想的主要倡导者是他忠诚的学生马尔可夫(Марков),他将老师的结果完整清晰地展现出来。马尔可夫(Марков)自己对概率论的重大贡献之一是创立了概率论的一个分支:研究相依随机变量的理论,成为"马尔可夫(Марков)过程"
为证明概率论的中心极限定理,切比雪夫(Чебышёв)和马尔可夫(Марков)利用的是矩方法,而李雅普诺夫(Ляпуно́в)利用了特征函数的方法。极限定理的后续发展表面特征函数方法是一种强大的解析工具。

D、20世纪

20世纪可成为概率论发展的现代时期,本时期开始于概率论的公理化。在这个方向上的早期贡献者有S.N. Berstein(1880-1968), R. von Mises(1883-1953)与E.Borel(1871-1956)。1933年,俄罗斯著名数学家柯尔莫哥洛夫(A.H.Колмого́ров)出版了他的伟大专著《Foundations of the Theory of Probability》。其中,他为概率论建立了目前广泛采纳的公理化体系。这一时期,中国数学家、概率论先驱许宝騄先生(Paolu Hsu,1910-1970)在内曼-皮尔逊理论、参数估计理论、多元分析、极限理论等方面取得卓越成就,许宝騄先生是多元统计分析学科的开拓者之一。
在20世纪,随机过程理论(马尔可夫(Марков)过程,平稳过程, Martingales(鞅论),随机过程的极限定理等)得到了快速的发展。另外,还有许多分支,比如(排名不分先后)随机微分方程、随机偏微分方程、倒向随机微分方程、随机微分几何、Malliavin变分、白噪声分析、狄氏型理论、遍历理论、数理金大偏差理论、交互粒子系统、测度值过程、概率不等式、泛函不等式、渗流、最有传输、SLE、随机矩阵、随机优化、随机控制、随机动力系统等众多概率论、随机分析及相关领域中的分支得到了快速的发展。




\section{集合与点集}

{Set Theory}

集合是数学中最原始的概念.若要给它下定义,不得不引入新的概念来说明它.若要给这些新的概念下定义,又不得不引入另外的新的概念.这样就会导致无休止的讨论.因此,在直观的朴素集合论(naive set theory)中,集合被看成是无需下定义的基本概念.为了方便,我们愿意给集合概念一个直观的描述,但这不是给它下定义.(我们不打算介绍ZFC公理体系等公理化集合论,只介绍比较直观的朴素集合论,对于我们理解这门课,这已足够.)

\begin{blist}
	\item 集合论自19世纪80年代由G.Cantor创立以来,现在已经发展成独立的数学分支.
	\item Russell悖论(Russell's paradox 或Russell's antinomy)的出现促使数学家产生了许多改进方案,就是把集合论公理化.
	
	\item 最著名是ZF(Zermelo-Fraenkel)公理集合论体系,及其保守扩张GB(Gödel以及Bernays).
\end{blist}

在此,请同学们相信我们所提到的构造集合的方法都不会导致悖论.集合论中涉及数学基础的那些深层问题,也不会自己跳出来颠覆人们所发展的概率论方法.

在正式开始学习概率论之前,让我们先复习一下集合论的一些简单知识.这些知识构成了概率论语言的基础.我们假定大家对于朴素的集合论已经具有了一定的了解,因此只将这些介绍蜻蜓点水式地简单回顾.


{Sets}
\textbf{Definition}
\begin{blist}
	\item 集合就是一些东西的总体.
	\item 总体中的东西称为这个集合的元素
	\item 元素$\omega$是集合$A$的一个元素,称作元素$\omega$属于集合$A$,记作$\omega\in A$或者$A\ni\omega$.
	\item 在一个数学问题中,常有这样一个集合$\Omega$,使得我们打算研究的所有对象都是这个特定集合的元素.这个集合$\Omega$常被称为(这个问题的)空间/全集(universal set).
\end{blist}	




众所周知,一个集合(set)就是把一堆元素(element)放在一起当作一个整体来看待.当$x$时$A$的元素的时候,我们也称$x$属于(belong to)$A$,记为$x\in A$,否则称$x$不属于$A$,记为$x\notin A$.当两个集合相等当且仅当所含元素完全相同.于是证明两个集合$A=B$的标准办法就是去证明\begin{equation}
x\in A\Longleftrightarrow x\in B.
\end{equation}
不含任何元素的集合称为空集(empty set),记做$\emptyset$.


但并不是说随便把一堆对象拿出来放在一起就可以构成一个集合,这会导致悖论.比如著名的Russell悖论(Russell's paradox或Russell's antinomy)是说,如果定义$X$为“又所有不属于自己的集合构成的集合”,则从$X\in X$就能推出$X\notin X$,反过来从$X\notin X$也能推出$X\in X$,这是自相矛盾的.换言之有些构造集合的方法是安全的,有些则是不安去的,为了避免悖论的产生,无法用安全的方法实现的构造我们都不应当承认它是集合.下面介绍几种常规(即安全的)构造方法.



一种最常用的构造方法是从一个已知的集合中筛选出满足特定条件的元素,构成新的集合.已知$X$是一个集合,设$P(x)$是一个关于$x$的命题,则$X$中的元素使得命题$P(x)$成立的那些元素$x$就构成一个集合,记为\begin{equation}
\{x\in X|P(x) \}.
\end{equation}
有些时候我们打算研究的对象都是某个特定集合$X$的元素,这时就把$X$称为全集(universal set).所以Russell悖论和上述合法构造的差别就在于:悖论里构造的那个东西没有一个全集限定元素的选择范围.
\\ \hspace*{\fill} \\%空行
设$A,B$是两个集合,如果$x\in A$蕴含$x\in B$,则称$A$为$B$的子集(subset)或者$A$包含于(be included in)$B$,并称$B$包含$A$,记作$A\subset B$或$B\supset A$.如果$A\subset B$,并且$A\neq B$,则称$A$为$B$的真子集(proper subset).




$X$的全体子集也构成一个集合,称其为幂集(power set),记为$2^X$.用这个记号是因为当$X$是恰好有$n$个元素的有限集时,$2^X$是恰好有$2^n$个元素的有限集.
\\ \hspace*{\fill} \\%空行
一个所有元素都是集合的集合称为集合族(collection of sets).设$\mathcal{A}$是集合族,则可以定义$\mathcal{A}$的所有元素(都是集合)的交集(intersection)$\cap\mathcal{A}$以及并集(union)$\cup\mathcal{A}$如下:
\begin{blist}
	\item $x\in\cap\mathcal{A}$当且仅当$x$是每一个集合$A\in\mathcal{A}$的元素;
	\item $x\in\cup\mathcal{A}$当且仅当存在集合$A\in\mathcal{A}$,使得$x$是$A$的元素.
\end{blist}




这里有个小小的问题:如果$\mathcal{A}$不含任何元素,那么如何判断一个元素$x$是不是每个集合$A\in\mathcal{A}$的元素呢?为此我们规定$\cap\emptyset = \emptyset$.
\\ \hspace*{\fill} \\%空行
如果集合族$\mathcal{A}$只有两个元素$A$和$B$,则交集和并集就记为$A\cap B$和$A\cup B$.对于一般的集合族,人们也往往喜欢弄出一个指标集$\Lambda$,然后把集合族写成$\{A_{\lambda}\}_{\lambda\in\Lambda}$,并把交集和并集写成$\bigcap_{\lambda\in\Lambda}A_{\lambda}$和$\bigcup_{\lambda\in\Lambda}A_{\lambda}$的样子.请注意,指标集可以是任何集合,而不一定是自然数集或其子集,所以一般来说是不能对指标集应用数学归纳法的.



交和并的运算满足各自的交换律和结合律,也满足分配律,即
\begin{blist}
	\item \begin{equation}
	A\cup\left(\bigcap_{\lambda\in\Lambda}B_{\lambda} \right) = \bigcap_{\lambda\in\Lambda} \left(A\cup B_{\lambda} \right)
	\end{equation}
	\item \begin{equation}
	A\cap\left(\bigcup_{\lambda\in\Lambda}B_{\lambda} \right) = \bigcup_{\lambda\in\Lambda} \left(A\cap B_{\lambda} \right)
	\end{equation}
\end{blist}
两个集合$A,B$的差集(difference)定义为\begin{equation}
A\backslash B = \{x\in A|x\notin B \},
\end{equation}
有时候,差集也记做$A-B$.



差集满足De Morgan定律(De Morgan's law)
\begin{blist}
	\item \begin{equation}
	B\backslash\left(\bigcap_{\lambda\in\Lambda}A_{\lambda} \right) = \bigcup_{\lambda\in\Lambda}\left(B\backslash A_\lambda\right)
	\end{equation}
	\item \begin{equation}
	B\backslash\left(\bigcup_{\lambda\in\Lambda}A_{\lambda} \right) = \bigcap_{\lambda\in\Lambda}\left(B\backslash A_\lambda\right)
	\end{equation}
\end{blist}
特别地,如果对于我们研究的问题有一个预先取定的全集$X$,则差集$X\backslash A$称为$A$对全集$X$的余集或补集(complement),记为$A^{c}$.注意,有些文献会习惯使用$\overline{A}$表示$A$的余集,但我们如果也使用这个符号,就会在后面讨论$A$的"闭包"时,会产生歧义,因而此处只使用$A^{c}$来表示$A$的余集.


\begin{definition}[映射]
\begin{blist}
	\item 一个映射(map或mapping)~$f$是两个集合$X$和$Y$的元素之间的一个对应关系,使得每个$X$的元素$x$都对应唯一一个$Y$的元素$f(x)$.
	\item 我们有时也把映射的定义简写为\begin{equation}
	f:X\rightarrow Y,x\mapsto f(x).
	\end{equation}
	\item 称$X$为$f$的定义域(domain),$Y$为$f$的陪域(codomain).
	\item 全体$X$到$Y$的映射也构成一个集合,记作$Y^X$.用这个记号是因为当$X$有$m$个元素,$Y$含$n$个元素时,$Y^X$恰好含$n^m$个元素.
\end{blist}	
\end{definition}

\medskip
\begin{note}
	作为特殊情形, 集合到自身的映射称为\myind{变换}.
\end{note}


注意:幂集$2^\Omega$的元素可以和$\{0,1\}^\Omega$的元素建立起一一对应关系:每一子集$A\subset \Omega$对应的映射就定义为\begin{equation}
\bm{1}_A:\Omega\rightarrow \{0,1\},\omega\mapsto\left\{
\begin{aligned}
1 &,~\omega\in A ;\\
0  &,~\omega\in \Omega\backslash A.
\end{aligned}
\right.
\end{equation}
\\ \hspace*{\fill} \\%空行
如果$y=f(x)$,则称$y$为$x$的像(image),称$x$为$y$的一个原像(preimage或inverse image).从定义可以看出,每个$x\in X$只能有一个像,但是每个$y\in Y$可以有很多原像,也可以没有原像.$y$的所有原像构成的集合记为$f^{-1}(y)$.如果$A$是$X$的子集,则称\begin{equation}
f(A) = \{y\in Y|\text{存在} x\in A,\text{使得}f(x)=y\}
\end{equation}
为$A$的像集(image).



如果$B$是$Y$的子集,则称\begin{equation}
f^{-1}(B) = \{x\in X|f(x)\in B\}
\end{equation}
为$B$的原像集(preimage或inverse image).
\\ \hspace*{\fill} \\%空行
不难验证,原像集满足下述简单的规律:
\begin{blist}
	\item \begin{equation}
	f^{-1}\left(\bigcap_{\lambda\in\Lambda}A_{\lambda} \right) = \bigcap_{\lambda\in\Lambda}f^{-1}\left(A_{\lambda}\right);
	\end{equation}
	\item \begin{equation}
	f^{-1}\left(\bigcup_{\lambda\in\Lambda}A_{\lambda} \right) = \bigcup_{\lambda\in\Lambda}f^{-1}\left(A_{\lambda}\right);
	\end{equation}
	\item \begin{equation}
	f^{-1}(A\backslash B) = f^{-1}(A) \backslash f^{-1}(B).
	\end{equation}
\end{blist}



但是,像集却完全可能不满足这些规律.
\\ \hspace*{\fill} \\%空行	
如果任取两个点$x_1,x_2\in X$,$x_1\neq x_2$蕴含$f(x_1)\neq f(x_2)$,则称$f$为单射(injection).如果任取一个$Y$中的点$y$,$f^{-1}(y)\neq\emptyset$,则称$f$是满射(surjection).如果$f$既是单射又是满射,则称$f$为双射 (bijection)或一一对应(one-to-one correspondence).有时,也称双射为可逆映射(invertible map),因为此时每一个$y\in Y$存在唯一原像,从而可以定义逆映射(inverse)
\begin{equation}
f^{-1}:Y\to X,
\end{equation}
把每个$y\in Y$对应到它关于$f$的那个唯一原像.



任何集合上都可以定义恒同映射(identity)\begin{equation}
\mathrm{id}_X:X\to X,x\mapsto x.
\end{equation}
如果$A\subset X$,则可定义含入映射(inclusion),也叫包含映射\begin{equation}
i_A:A\to X,x\mapsto x.
\end{equation}
若有两个映射$f:X\to Y$和$g:Y\to Z$,则可以定义复合映射(composition)\begin{equation}
g\circ f:X\to Z,x\mapsto g(f(x))
\end{equation}
不难验证,复合应黑色满足下述两个基本性质:
\begin{blist}
	\item 如果$g\circ f$是单射,则$f$是单射;
	\item 如果$g\circ f$是满射,则$g$是满射.
\end{blist}



若$X,Y$是两个集合,则全体有序对$(x,y)$(其中$x\in X,y\in Y$)也构成一个集合,称为$X$和$Y$的笛卡儿积(Cartesian product)或直积(direct product),记作$X\times Y$(英文Cartesian其实是将Descartes(笛卡儿)的名字形容词化之后的结果).也可以类似地定义有限多个集合的直积$X_1\times \cdots \times X_n$,即所有序列$(x_1,\cdots,x_n)$(其中每个$x_j\in X_j$)构成的集合,特别地,$n$个$X$的直积记为$X^n$,例如,实$n$维空间$\mathbb{R}^n = \mathbb{R}\times\cdots\times\mathbb{R}$.

集合上的一个\myind{二元运算}即是由$A\times A$到$A$的一个映射. $A$上的一个\myind{二元关系}$R$定义为$A\times A$的一个子集. 如果$(a_1,a_2)\in R$, 就称$a_1$与$a_2$有关系$R$, 记为$a_1Ra_2$.

\begin{definition}[]
	
\end{definition}


Cantor最早发明集合论的时候说得很含糊,他说集合是“一些确定的、不同的东西的总体”.他开始想的其实只是实数集$\mathbb{R}$的子集.当然了,他更关心的是无穷子集.他首先发现,一个集合含有无穷多个元素的充分必要条件是它能和自己的一部分一一对应.然后他又提出了比较两个无穷集合所含元素的"个数"谁多谁少的方法.一般集合上这种相当于“个数”的概念称为基数(cardinality)或势(potency).如果集合$A$和$B$的元素能建立起一一对应,就称它们的基数相同,记为\begin{equation}
\|A\| = \|B\|.
\end{equation}
如果$\|A\|\neq\|B\|$,但是$A$能和$B$的一部分一一对应,则称$A$的基数小于$B$的基数,记为\begin{equation}
\|A\| < \|B\|.
\end{equation}



Cantor很快便证明了$\|\mathbb{Q}\| = \|\mathbb{N}\|$,而$\|\mathbb{R}\| > \|\mathbb{N}\|$.然后他被一个问题难倒了:是否存在一个集合$A$,使得$\|\mathbb{N}\|<\|A\|<\|\mathbb{R}\|$? Cantor猜想这样的集合不存在,这个猜想被称为连续统假设(因为$\mathbb{R}$的基数被称为连续统),但是他无法证明.
\\ \hspace*{\fill} \\%空行
给朴素的集合论想法带来更大打击的是悖论:不加限制地把“满足某种条件的对象”放在一起,有的时候构造出来的东西不能当作集合对待,也就是说,那些对于集合应该自然成立的结论到这里不一定成立,否则会导出自相矛盾的结果.
\\ \hspace*{\fill} \\%空行
比如Cantor就发现,把所有的对象放在一起后成一个大全集$\Omega$,$\Omega$就不能当作集合对待.因为否则的话,一方面任何集合的基数都不应该比$\Omega$的基数更大;而另一方面,如果考虑$\Omega$的幂集合$2^\Omega$,又可以像证明$\|\mathbb{R}\| > \|\mathbb{N}\|$那样证明$\|2^\Omega\|>\|\Omega\|$.这被称为Cantor悖论(Cantor's paradox).



随后Russell(罗素)又发现了更简单直接的Russell悖论(Russell's paradox或Russell's antinomy):把所有"不是自己的元素的集合"放在一起构成的那个东西也不能当作集合对待,因为否则的话记这个集合为$\Omega$,则$(\Omega\in \Omega)\Longleftrightarrow (\Omega\notin \Omega)$.
\\ \hspace*{\fill} \\%空行
想要避免悖论的发生,就必须要限制构造集合的方法,需要明确知道那些构造方法是安全的,只有那些符合安全规范的方法构造出来的对象才全是集合,才能放心大胆地应用关于集合的那些性质和命题.像前面零个悖论中的$\Omega$就都不算集合了.那然,那些最最基础的构造方法是不是安全是没法证明的,只能把每一条当做一个公理,这样得到的就是集合论的公理系统.历史的发展证明,只要对集合的构造方法加以限制,集合论就可能成为数学的可靠基础.





最常用的集合论公理系统是ZF及其保守扩张GB(这里Z,F,G,B分别是其发明人Zermelo, Fraenkel, Gödel 以及 Bernays 名字的第一个字母)
\\ \hspace*{\fill} \\%空行
ZF是Zermelo-Fraenkel集合论(Zermelo-Fraenkel set theory)的简称,包括九条公理,每条公理是一个逻辑命题.注意ZF系统的所有命题中提到的对象都可以是集合,包括其中提到集合的元素时,这些元素课都可以是集合.仔细观察以下不难发现,这些公理中“某某集合存在”表达的其实都是“某某方式构造是对象算是集合”的意思.



外延公理(axiom of extensionality) 如果$x\in X$当且仅当$x\in Y$,则$X=Y$,即集合由其元素完全决定
\\ \hspace*{\fill} \\%空行
无序对公理(axiom of pairing)$\forall x,y$,存在集合$\{x,y\}$(无序对),使得$z\in \{x,y\}$当且仅当$z=x$或$z=y$.

Rmk:这里允许$x=y$,此时$\{x,y\}=\{x\}$.能区分$x,y$地位的有序对则定义为集合
\begin{equation}
(x,y) = \{\{x\},\{x,y\}\}.
\end{equation}
\\ \hspace*{\fill} \\%空行
并集公理(axiom of union)$\forall \mathcal{A}$,存在集合$\cup\mathcal{A}$,使得$x\in\cup\mathcal{A}$当且仅当存在$A\in\mathcal{A}$满足$x\in A$.

结合无序对公理和并集公理可以归纳定义无序多元组如下:
\begin{equation}
\{x_1,\cdots,x_{n+1} \} = \{x_1,\cdots,x_{n} \} \cup \{x_{n+1}\}.
\end{equation}




幂集公理(axiom of power set)$\forall X$,存在集合$2^{X}$(幂集),使得$A\in 2^{X}$当且仅当 $A\subset X$.这里$A\subset X$是逻辑命题
\begin{equation}
\forall x\in A\Rightarrow x\in X
\end{equation}
(即$A$是$X$的子集)的简写.
\\ \hspace*{\fill} \\%空行
分离公理(axiom schema of separation或axiom schema of specification) 任取集合$X$以及关于$x$的命题$P(x)$,存在集合$A$使得$x\in A$当且仅当$x\in X$并且$P(x)$成立.集合$A$通常记为$\{x\in X|P(x)\}$.

结合分离公理和并集公理,就可以把交集定义为
\begin{equation}
\cap \mathcal{A} = \{x\in\cup\mathcal{A}|\forall A\in\mathcal{A},x\in A \}.
\end{equation}



注意到当$x\in X,y\in Y$时,前面定义的有序对$(x,y)$其实是一个由$X\cup Y$的子集构成的集合,这样的集合是$2^{X\cup Y}$的子集,也就是$2^{(2^{X\cup Y})}$的元素,因此利用分离公理,可以把直积(direct product)定义为
\begin{equation}
X\times Y = \{ z\in 2^{(2^{X\cup Y})}| \exists x\in X,y\in Y,\text{使得}z=(x,y) \}.
\end{equation}
直积也称作笛卡儿积(Cartesian product),其中英文Cartesian其实是将Descartes(笛卡儿)的名字形容词化之后的结果.

每个从$X$射到$Y$的映射$f:X\rightarrow Y$可以用它在$X\times Y$中的函数图像来刻画,因此也不需要添加新的公理,就可以把从$X$打到$Y$的映射定义为集合
\begin{equation}
Y^X = \{f\in 2^{X\times Y} | \forall x\in X,\text{存在唯一}\ y\in Y\ \text{使得}(x,y)\in f \}
\end{equation}
的元素.并且此时如果$(x,y)\in f$,则把$y$记为$f(x)$.



空集公理(axiom of empty set) 存在集合$\emptyset$(空集),使得$\forall x,x\notin \emptyset$.

由外延公理可以知道空集是唯一的.你也许会认为空集公理是多余的,因为有了分离公理可以随便取个集合,再取一个该集合中元素永远不满足的性质,然后就可以构造出空集来了,比如说$\emptyset=\{x\in A|x\notin A\}$.很多的书籍上也确实是这么写的,但是这实际上依赖于另外的一条其它公理中没有提到的假设:在这个世界上确实至少存在着那么一个集合$A$.缺少这个假设,形式逻辑的推理过程就没有了起点.当然,也有群体认为下面的一条公理"无穷公理"本身就包含了一定存在一个空集的意思.



无穷公理(axiom of infinity) 存在一个集合$\omega$(含有无穷多个元素的集合),使得$\emptyset\in\omega$,并且$x\in\omega$蕴含$x\cup\{x\}\in\omega$.

在集合论中,每个自然数$n$被归纳地定义成一个恰好有$n$个元素的特殊集合:
\begin{equation}
0=\emptyset,1=\{0\},\cdots,n+1=\{0,\cdots,n\} = n\cup\{n\}.
\end{equation}	

比较一下这个定义和上述无穷公理不难看出,自然数集$\mathbb{N}$就是满足无穷公理的最小集合.有趣的是关于自然数的加减乘除运算的各种基本性质都可以从这个定义以及ZF系统的其它公理推导出来.有了自然数之后当然也可以定义整数和有理数,并把数学分析中定义实数的方法(比如Cauchy序列或者Dedekind分割等等)也改写成符合ZF系统的形式.



替换公理(axiom schema of replacement)任取集合$X$以及关于$x,y$的逻辑命题$R(x,y)$,如果$R$满足$\forall x\in X$,存在唯一$y$使得该命题成立,则存在集合$Y$,使得$y\in Y$当且仅当存在一个$x\in X$使得$R(x,y)$成立.
\\ \hspace*{\fill} \\%空行
如果我们把$R(x,y)$理解成一个映射$f:x\mapsto y$,那么替换公理构造的$Y$其实就是$X$的象集$f(X)$.
\\ \hspace*{\fill} \\%空行
正则公理(axiom of regularity)任取非空集合$X$,其中元素关于$\in$关系存在一个极小元素,即存在$x\in X$,使得$\forall y\in X,y\notin x$.
\\ \hspace*{\fill} \\%空行
注意:极小元素的选取并不一定唯一,也不一定是最小元素.有趣的是,任取两个自然数$m$和$n$,$m<n$当且仅当按照前述公理化定义把它们当作集合看待时$m\in n$.因此,这最后一条公理给出了数学归纳法(mathematical induction)的一种新的理解:要想证明一系列命题$P_n$对任意$n\in\mathbb{N}$都成立,可以取$X=\{ n\in\mathbb{N}|\text{命题}P_n\text{不成立} \}$,则$X$一定含有一个极小元素$m$,从而任取自然数$i<m$,命题$P_i$成立.于是如果我们能从$P_0,\cdots,P_{m-1}$成立推导出命题$P_m$成立,就完成了证明.




ZF是集合论的公理化体系中最简单可靠的一个.当然,简单可靠的代价就是应用上的局限性.有许多数学中的著名论断都需要在ZF之外再添加一条选择公理才能推推导出来.
\\ \hspace*{\fill} \\%空行
选择公理(axiom of choice)任取一个由两两不相交的集合构成的集合族$\mathcal{A}$,存在一个集合$C$,它与$\mathcal{A}$的每个元素(元素是集合)都恰好交于一点.
\\ \hspace*{\fill} \\%空行
选择公理的一个简单推论是:任取一个集合族$\mathcal{A}$(不一定两两不交),存在一个映射$f:\mathcal{A}\mapsto\cup\mathcal{A}$,使得$\forall A\in\mathcal{A},f(A)\in A$.换言之,允许同时在集合族$\mathcal{A}$的每个元素(集合)里选择一个元素.






选择公理是一个饱受争议的公理,数学家们一方面对于是否应该允许“同时”进行无穷多项的构造提出了强烈的质疑,另一方面又利用它证明了很多虚无缥缈、无法构造的东西存在.比如线性代数中任意线性空间中基的存在性,或者抽象代数中任意环的极大理想的存在性,或者实变函数(测度论)中不可测集的存在性,或者泛函分析中的Hahn-Banach定理,它们的证明过程中都直接或者间接地用到了选择公理,或者用到了在ZF中与选择公理等价的良序定理或Zorn引理.
\\ \hspace*{\fill} \\%空行
值得一提的是与不可测集的存在性相关的一个著名的结论,人们称之为Banach-Tarski悖论(Banach-Tarski paradox).这个悖论大意是说,如果承认选择公理正确,则可以证明存在三维欧氏空间中的一族有限多个互不相交的子集,它们并起来是一个半径为$1$的实心球,但是把每个子集只进行一些旋转和平移,还可以在新的位置上保持互不相交地重新拼出两个半径为$1$的实心球.也就是将神话里的"分身术"找到了数学依据!这看上去显然是非常不符合常识的,这个结论也曾经一度被认为是选择公理不成立的直接证明,因此它才被称为"悖论"而不是"定理".




当然,这个结论在逻辑上并没有任何矛盾,它只是和我们从多面体体积那里得来的常识相矛盾,有什么理由认为这些常识对于那些复杂得不可想象的子集也应该正确呢?今天的大部分数学家都倾向于站在选择公理一边,也就是说,当我们把一大堆没有体积的点胡乱堆在一起的时候,不应当假定这堆点的"体积"就一定符合积木方砖那种东西带给我们的几何常识.只不过在每一个需要用到选择公理的结论上,数学家们都会特意标注一下,省得将来反悔的时候不知道该丢弃些什么.我们也会这样处理.当然,对于本课的主要内容来说,ZF已经足够应用了.




有趣的是,即使是用ZF加上选择公理构成所谓的ZFC,依然解决不了最初难道Cantor的连续统假设,或者说已经解决了,却不是Cantor想要的答案.Gödel在1940年证明了从ZFC出发不可能证明连续统假设是错误的,而Cohen则在1963年证明了从ZFC出发不可能证明连续统假设是正确的.


{Cantor-Bernstein 定理}




\section{抽象代数温习}
我们首先回忆抽象代数中建立起的几个基本代数结构.

集合$S$上的一个\myind{二元代数运算}是指$S\times S$到$S$的一个映射.
\begin{definition}[群]
	设$G$是一个非空集合.如果在$G$上定义了一个二元运算$\circ$,通常称为乘法,并且满足:
	\begin{blist}
		\item (1)~\myind{结合律}: $(a\circ b)\circ c = a\circ (b\circ c), \forall a,b,c\in G$;
		\item (2)存在\myind{幺元}: 存在$e\in G$,使得\begin{equation}
			e\circ a = a\circ e = a, \forall a\in G
		\end{equation}($e$称为$G$的幺元);
		\item (3)存在\myind{逆元}: 对每个的$a\in G$,存在$b\in G$,使得\begin{equation}
			 a\circ b = b\circ a = e, \forall a\in G
		\end{equation}($b$称为$a$的逆元,记作$a^{-1}$),
	\end{blist}
	则称$G$关于运算$\circ$构成一个\myind{群},记为$(G,\circ)$,或者简记为$G$.
	
	群$G$中若还成立以下的:
	\begin{blist}
		\item (4)交换律:$a\circ b = b\circ a,\forall a,b\in G$,
	\end{blist}
	则称$G$为\myind{交换群}或\myind{Abel 群}.
\end{definition}
在不致引起混淆的情况下,运算符号"$\circ$"经常略去不写.

由结合律(1)可以推出下面的广义结合律:
\begin{blist}
	\item (1')~\myind{广义结合律}:对于任意有限多个元素$a_1,\cdots,a_n\in G$, 乘积$a_1a_2\cdots a_n$的任何一种"有意义的加括号方式"(即给定乘积的顺序)都得出相同的值, 因而上述的乘积是有意义的.
\end{blist}

顺便介绍一下半群和幺半群的概念.
\begin{definition}[半群、幺半群]
	如果一个非空集合$S$上有二元运算, 此运算满足结合律, 则称此集合关于这个二元运算构成一个\myind{半群}.
	
	具有幺元的半群称为\myind{幺半群}.
\end{definition}

\begin{definition}[环]
	如果一个非空集合$R$上定义了两个二元代数运算$+,\cdot$(分别称为加法和乘法),满足:
	\begin{blist}
		\item (1)~$(R,+)$构成Abel群;
		\item (2)~\myind{乘法结合律}: $(a\cdot b)\cdot c = a\cdot (b\cdot c),~\forall a,b,c\in R$;
		\item (3)~\myind{分配律}: $(a+b)\cdot c = a\cdot c + b\cdot c$, $c\cdot(a+b) = c\cdot a+c\cdot b$, $\forall a,b,c\in R$,
	\end{blist}
	则称$R$关于运算$+,\cdot$构成一个\myind{环},记为$(R,+,\cdot)$,或简记为$R$.
	
	环$R$中若成立
	\begin{blist}
		\item (4)~\myind{乘法交换律}: $a\cdot b = b\cdot a, ~\forall a,b\in R$,
	\end{blist}
	则称$R$为\myind{交换环}.
\end{definition}

\begin{definition}[体、域]
	设$D$是至少含有两个元素的幺环.如果$D$的每个非零元都可逆, 则称$D$是一个\myind{体}.
	
	具有乘法交换律的体称为\myind{域}.
\end{definition}

下面介绍模和格.模是线性空间的推广. 粗略地说, 线性空间和模都是Abel群, 不同的是, 线性空间用域的元素做数量乘法, 而模则可以用幺环中的元素作数量乘法.

由此,先温习线性空间的定义.线性空间是线性代数中最基本的概念之一, 它是定义在某个域上并满足一定条件的一个集合.
\begin{definition}[线性空间]
	设 $V$ 是一个非空集合, $\mathbb{F}$ 是一个数域.
	在 $V$ 上定义一种代数运算, 称为\myem{加法}, 记为 "$+$"
	(即对任意 $x, y\in V$, 都存在唯一的 $z\in\SS$, 使得 $z=x+y$),
	并定义一个从 $\mathbb{F}\times\SS$ 到 $\SS$ 的代数运算, 称为\myem{数乘},
	记为 ``$\cdot$" (即对任意 $\alpha\in\mathbb{F}$ 和任意 $x\in\SS$,
	都存在唯一的 $y\in\SS$, 使得 $y=\alpha\cdot x$).
	如果这两个运算满足下面的规则,
	则称 $(\SS,+,\cdot)$ 是数域 $\mathbb{F}$ 上的一个\myind{线性空间}
	(通常简称 $\SS$ 是数域 $\mathbb{F}$ 上的一个线性空间):
	\begin{blist}
		\item 加法四条规则
		\begin{nlist}
			\item \myem{交换律}: $x + y = y + x, \quad
			\forall\ x,y\in\SS$;
			\item \myem{结合律}: $(x + y) + z = x + (y + z),
			\quad \forall\ x, y, z\in\SS$;
			\item \myem{零元素}: 存在一个元素 0, 使得
			$x + 0 = x,\quad \forall\ x\in\SS$;
			\item \myem{逆运算}: 对任意 $x\in\SS$, 都存在\myem{负元素}
			$y\in\SS$, 使得 $x + y=0$, 记 $y=-x$;
		\end{nlist}
		\item 数乘四条规则
		\begin{nlist}
			\item \myem{单位元}:
			$1\cdot x=x,\quad 1\in\mathbb{F},\ \forall\ x\in\SS$;
			\item \myem{结合律}: $\alpha\cdot(\beta\cdot x)=(\alpha \beta)\cdot x,\quad
			\forall\ \alpha,\beta\in\mathbb{F},\ x\in\SS$;
			\item \myem{分配律}:  $(\alpha+\beta)\cdot x=\alpha\cdot x+\beta\cdot x,\quad
			\forall\ \alpha,\beta\in\mathbb{F},\ x\in\SS$;
			\item \myem{分配律}:  $\alpha\cdot(x+\beta)=\alpha\cdot x+\alpha\cdot y,\quad
			\forall\ \alpha\in\mathbb{F},\ x, y\in\SS$.
		\end{nlist}
	\end{blist}
	为了表示方便, 通常省略数乘符号, 即将 $\alpha\cdot x$ 写成 $\alpha x$.
\end{definition}

\section{集合系}
集合系就是由集合组成的集合.在本课中可能用到的集合系有$\pi-\text{系}$、半环、半代数、环、代数、$\sigma-\text{环}$、$\sigma-\text{代数}$、单调系、$\lambda-\text{系}$等.这些集合系中最重要的是$\sigma-\text{代数}$,下面一一介绍这些集合系.

$\pi-\text{系}$:如果$X$上的非空集合系$\mathcal{P}$对交运算封闭,

即\begin{equation}
\forall A,B\in\mathcal{P} \implies AB\in\mathcal{P}.
\end{equation} 
则称$\mathcal{P}$为$\pi-\text{系}$.

例:$\mathcal{P}_\mathbb{R} = \{(-\infty,a]|a\in\mathbb{R}  \}$对有限交的运算封闭,因而组成实数空间$\mathbb{R}$上的$\pi-\text{系}$.



半环:满足下列条件的$\pi-\text{系}~\mathcal{S}$称为半环

$(\mathcal{S},\Delta)$构成一个交换的含幺半群

与$(\mathcal{S}\backslash\{0\},\cap)$一个含幺半群.

加法对乘法满足分配律

{集合系之间的关系}

\begin{tabular}{ccccc}
$\sigma-\text{代数}$ & $\Longleftrightarrow$ & $\pi-\text{系}$ & $+$ & $\lambda-\text{系}$ \\
~ & ~ & ~ & ~ & $\Downarrow$ \\
~  & $\Longleftrightarrow$ & 代数 & $+$ & 单调系 \\
\end{tabular}
\\ \hspace*{\fill} \\%空行
\hspace*{\fill} \\%空行
\begin{tabular}{ccccccc}
$\sigma-\text{代数}$ & $\Longrightarrow$ & 代数 & $\Longrightarrow$ & 半代数 & ~ & ~ \\
$\Downarrow$ & ~ & $\Downarrow$ & ~ & $\Downarrow$ & ~ & ~ \\
$\sigma-\text{环}$ & $\Longrightarrow$ & 环 & $\Longrightarrow$ & 半环 & $\Longrightarrow$ & $\pi-\text{系}$
\end{tabular}


{测度论中的典型方法}
在测度论和概率论中,为了证明一个关于可测函数的命题,常常分解为如下几个比较容易的步骤进行:
\begin{blist}
\item (1)证明该命题对最简单的函数——示性函数成立.
\item (2)证明该命题对非负简单函数——示性函数的线性组合成立.
\item (3)证明该命题对非负可测函数——非降的非负简单函数列的极限成立.
\item (4)证明该命题对一般的可测函数——两个非负可测函数,即它的正部和负部之差成立.
\end{blist}
按上述步骤证明命题的方法叫做测度论中的典型方法.典型方法符合人们的认识过程,是一种具有普遍意义、行之有效的方法,必须熟练掌握.




\section{点集拓扑初步}


\section{二十世纪最优秀的十大算法}

\noindent
The Best of the 20th Century: Editors Name Top 10 Algorithms,
SIAM News, Volume 33, Number 4, 2000.

\begin{blist}\addtolength{\itemsep}{-0.5ex}
	\item[1.] Monte Carlo method (1946)
	\item[2.] \underline{Simplex Method for Linear Programming} (1947)
	\item[3.] \myem{Krylov Subspace Iteration Methods} (1950)
	\item[4.] \myem{The Decompositional Approach to Matrix Computations} (1951)
	\item[5.] The Fortran Optimizing Compiler (1957)
	\item[6.] \myem{QR Algorithm for Computing Eigenvalues} (1959-61)
	\item[7.] Quicksort Algorithm for Sorting (1962)
	\item[8.] \underline{Fast Fourier Transform} (1965)
	\item[9.] Integer Relation Detection Algorithm (1977)
	\item[10.] \underline{Fast Multipole Method} (1987)
\end{blist}
(蓝色: 属于数值线性代数; 下划线: 与数值线性代数密切相关.)

\section{课程主页}

\url{http://math.ecnu.edu.cn/~jypan/Teaching/MatrixComp/}


\section{计算数学}
% \subsection[计算数学]{计算数学介绍}

1947 年, Von Neumann 和 Goldstine 在 Bulletin of the AMS (美国数学会通报)
上发表了题为 ``Numerical inverting of matrices of high order"
(高阶矩阵的数值求逆) 的著名论文, 开启了现代计算数学的研究.
半个多世纪以来, 伴随着计算机技术的不断进步, 计算数学得到了蓬勃发展,
并逐渐成为了一个独立和重要的学科.

通俗地讲, 科学计算就是通过数学建模将实际问题转化为数学问题,
然后对数学问题进行离散和数值求解, 从而得到原问题的近似解,
同时对得到的近似解进行误差估计, 以确保近似解的可靠性.
科学计算利用先进的计算能力认识和解决复杂的科学工程问题,
它融建模、算法、软件研制和计算模拟为一体,
是计算机实现其在高科技领域应用的必不可少的纽带和工具.
计算已不仅仅只是作为验证理论模型正确性的手段,
大量的事例表明它已成为重大科学发现的一种重要手段.
科学计算已经和理论研究与实验研究一起, 成为当今世界科学技术创新的主要方式,
也是当前公认的从事现代科学研究的三种方法.

\medskip

\begin{note}
	有关计算数学和数值线性代数的定义可以参考 Golub 教授的
	History of numerical linear algebra: A personal view,
	或 Trefethen 教授的 Numerical analysis.
\end{note}

\begin{Block}
	国家自然科学基金委员会关于\myem{计算数学}的分类 (2018):
	\begin{blist}
		\item[] \myem{数学} (数理科学部, A01)
		\begin{blist}
			\item 计算数学与科学工程计算 (A0117)
			\begin{blist}
				\item[-] 偏微分方程数值计算 (A011701)
				\item[-] 流体力学中的数值计算 (A011702)
				\item[-] 一般反问题的计算方法 (A011703)
				\item[-] 常微分方程数值计算 (A011704)
				\item[-] 数值代数 (A011705)
				\begin{blist}
					\item[] 结构矩阵、稀疏性与模型约化、科学与工程中的矩阵计算、
					矩阵方程求解、非线性方程求解、矩阵分析、特征值问题、
					大型线性方程组求解、最小二乘问题等
				\end{blist}
				\item[-] 数值逼近与计算几何 (A011706)
				\item[-] 谱方法及高精度数值方法 (A011707)
				\item[-] 有限元和边界元方法 (A011708)
				\item[-] 多重网格技术与区域分解 (A011709)
				\item[-] 自适应方法 (A011720)
				\item[-] 并行计算 (A011711)
			\end{blist}
			\item 运筹学 (A0112)
			\begin{blist}
				\item[-] 线性与非线性规划 (A011201)
				\item[-] 组合最优化 (A011202)
				\item[-] 随机最优化 (A011203)
				\item[-] 可靠性理论 (A011204)
			\end{blist}
		\end{blist}
		
		\item[] \myem{力学} (数理科学部, A02)
		\begin{blist}
			\item 固体力学 (A0203)
			\begin{blist}
				\item[-] 计算固体力学 (A020317)
			\end{blist}
			\item 流体力学 (A0204)
			\begin{blist}
				\item[-] 计算流体力学 (A020415)
			\end{blist}
		\end{blist}
		
		\item[] \myem{电子学与信息系统} (信息科学部, F01)
		\begin{blist}
			\item 信号理论与信号处理 (F0111)
			\begin{blist}
				\item[-] 压缩感知理论与方法 (F011110)
			\end{blist}
			\item 图像处理 (F0115)
			% \item 图像表征与显示 (F0116)
		\end{blist}
		
		\item[] \myem{人工智能} (信息科学部, F06) $\to$ {\color{red} 2018 年新增条目}
		\begin{blist}
			\item 机器学习 (F0602)
			\begin{blist}
				\item[-] 机器学习基础理论与方法 (F060201)
				\item[-] 监督学习 (F060202)
				\item[-] 弱监督学习 (F060203)
				\item[-] 无监督学习 (F060204)
				\item[-] 统计学习 (F060205)
				\item[-] 集成学习 (F060206)
				\item[-] 强化学习 (F060207)
				\item[-] 深度学习理论与方法 (F060208)
			\end{blist}
			\item 机器感知与模式识别 (F0603)
			% \item ... ...
		\end{blist}
		
		\item[] \myem{交叉学科中的信息科学} (信息科学部, F07) $\to$ {\color{red} 2018 年新增条目}
		\begin{blist}
			\item 信息与数学交叉问题 (F0702)
			\begin{blist}
				\item[-] 电子通信与数学交叉 (F070201)
				\item[-] 计算机与数学交叉 (F070202)
				\item[-] 自动化与数学交叉 (F070203)
				\item[-] 人工智能与数学交叉 (F070204)
				\item[-] 半导体与数学交叉 (F070205)
				\item[-] 光学与数学交叉 (F070206)
			\end{blist}
		\end{blist}
	\end{blist}
\end{Block}



\chapter{概率空间}

\section{线性空间与内积空间}
\subsection{线性空间}


\begin{example}
	常见的线性空间有:
	\begin{blist}
		\item $\R^n$ $\to$ 所有 $n$ 维实向量组成的集合, 是 $\R$ 上的线性空间.
		\item $\C^n$ $\to$ 所有 $n$ 维复向量组成的集合, 是 $\C$ 上的线性空间.
		\item $\Rmn$ $\to$ 所有 $m\times n$ 阶实矩阵组成的集合, 是 $\R$ 上的线性空间.
		\item $\C^{m\times n}$ $\to$ 所有 $m\times n$ 阶复矩阵组成的集合,
		是 $\C$ 上的线性空间.
		\item $\Pbb_n$ $\to$ 所有次数不超过 $n$ 的多项式组成的集合.
		\item $C[a,b]$ $\to$ 区间 $[a,b]$ 上所有连续函数组成的集合.
		\item $C^p[a,b]$ $\to$ 区间 $[a,b]$ 上所有 $p$ 次连续可微函数组成的集合.
	\end{blist}
\end{example}

\subsubsection{线性相关性和维数}

设 $\SS$ 是数域 $\mathbb{F}$ 上的一个线性空间,
$x_1, x_2, \ldots, x_k$ 是 $\SS$ 中的一组向量.
如果存在 $k$ 个不全为零的数 $\alpha_1,\alpha_2,\ldots,\alpha_k\in\mathbb{F}$,
使得
$$ \alpha_1 x_1 + \alpha_2 x_2 + \cdots + \alpha_k x_k = 0,$$
则称 $x_1, x_2, \ldots, x_k$ \myind{线性相关},
否则就是\myind{线性无关}.

设 $x_1, x_2, \ldots, x_k$ 是 $\SS$ 中的一组向量.
如果 $x\in\SS$ 可以表示为
$$ x = \alpha_1 x_1 + \alpha_2 x_2 + \cdots + \alpha_k x_k,$$
其中 $\alpha_1,\alpha_2,\ldots,\alpha_k\in\mathbb{F}$,
则称 $x$ 可以由 $x_1, x_2, \ldots, x_k$ \myind{线性表出},
或者称 $x$ 是 $x_1, x_2, \ldots, x_k$ 的\myind{线性组合}.

设向量组 $\{x_1, x_2, \ldots, x_m\}$,
如果存在其中的 $r$ ($r\leq m$) 个线性无关向量 $x_{i_1},x_{i_2},\ldots,x_{i_r}$,
使得所有向量都可以由它们线性表示,
则称 $x_{i_1},x_{i_2},\ldots,x_{i_r}$ 为向量组
$\{x_1, x_2, \ldots, x_m\}$ 的一个\myind{极大线性无关组},
并称这组向量的\myind{秩}为 $r$, 记为
$\rank(\{x_1, x_2, \ldots, x_m\})$.

设 $x_1, x_2, \ldots, x_n$ 是 $\SS$ 中的一组线性无关向量.
如果 $\SS$ 中的任意一个向量都可以由 $x_1, x_2, \ldots, x_n$ 线性表示,
则称 $x_1, x_2, \ldots, x_n$ 是 $\SS$ 的一组\myind{基},
并称 $\SS$ 是 $n$ 维的, 即 $\SS$ 的\myind{维数}为 $n$, 记为 $\dim(\SS)=n$.
如果 $\SS$ 中可以找到任意多个线性无关向量, 则称 $\SS$ 是\myind{无限维}的.


\subsubsection{子空间}
设 $\SS$ 是一个线性空间, $\WW$ 是 $\SS$ 的一个非空子集合.
如果 $\WW$ 关于 $\SS$ 上的加法和数乘也构成一个线性空间,
则称 $\WW$ 为 $\SS$ 的一个\myind{线性子空间},
有时简称\myind{子空间}.


\begin{example}
	设 $\SS$ 是一个线性空间, 则由零向量组成的子集 $\{0\}$
	是 $\SS$ 的一个子空间, 称为零子空间.
	另外, $\SS$ 本身也是 $\SS$ 的子空间.
	这两个特殊的子空间称为 $\SS$ 的\myind{平凡子空间},
	其他子空间都是\myind{非平凡子空间}.
\end{example}
%
下面给出子空间的一个重要判别定理.
\begin{theorem}
	设 $\SS$ 是数域 $\mathbb{F}$ 上的一个线性空间,
	$\WW$ 是 $\SS$ 的一个非空子集合.
	则 $\WW$ 是 $\SS$ 的一个子空间的充要条件是 $\WW$
	关于加法和数乘封闭, 即
	\begin{nlist}
		\item 对任意 $x,y\in\WW$, 有 $x+y\in\WW$;
		\item 对任意 $\alpha\in\mathbb{F}$ 和任意 $x\in\WW$,
		有 $\alpha x\in\WW$.
	\end{nlist}
\end{theorem}
%
下面是关于子空间的维数的一个很重要性质.
\begin{theorem}[\myind{维数公式}]
	设 $\SS_1$, $\SS_2$ 是线性空间 $\SS$ 的两个有限维子空间,
	则 $\SS_1+\SS_2$ 和 $\SS_1\cap\SS_2$ 也都是 $\SS$ 的子空间, 且
	$$
	\dim(\SS_1+\SS_2) + \dim(\SS_1\cap\SS_2)
	= \dim(\SS_1) + \dim(\SS_2).
	$$
\end{theorem}



\subsection{内积空间}
内积空间就是带有内积运算的线性空间, 是 $n$ 维欧氏空间的推广.
\begin{definition}[内积空间]
	设 $\SS$ 是数域 $\mathbb{F}$ ($\C$ 或 $\R$) 上的一个线性空间,
	定义一个从 $\SS\times\SS$ 到 $\mathbb{F}$ 的代数运算,
	记为 ``$(\,\cdot\, ,\,\cdot\,)$", 即对任意 $x,y\in\SS$,
	都存在唯一的 $f\in\mathbb{F}$, 使得 $f=(x,y)$.
	如果该运算满足
	\begin{nlist}
		\item $(x,y) = \ol{(y,x)},\quad
		\forall\ x,y\in\SS $;
		\item $(x+y,z) = (x,z) +(y,z),\quad
		\forall\ x,y,z\in\SS $;
		\item $(\alpha x,y)=\alpha (x,y),\quad
		\forall\ \alpha\in\mathbb{F},\ x,y\in\SS $;
		\item $(x,x)\geq 0$, 等号当且仅当 $x=0$ 时成立;
	\end{nlist}
	则称 $(\,\cdot\, ,\,\cdot\,)$ 为 $\SS$ 上的一个 \myind{内积}
	(\myind{inner product}), 定义了内积的线性空间称为 \myind{内积空间}.
\end{definition}

\medskip
\begin{note}
	内积有时也称为 \myind{标量积} (\myind{scalar product}).
\end{note}

\begin{note}
	定义在实数域 $\R$ 上的内积空间称为 \myind{欧氏空间},
	定义在复数域 $\C$ 上的内积空间称为 \myind{酉空间}.
\end{note}

\begin{example}
	设 $(\,\cdot\, ,\,\cdot\,)$ 是 $\SS$ 上的一个内积, 则容易验证:
	$$ (x,\alpha y)=\bar\alpha (x,y),\quad
	\forall\ \alpha\in\mathbb{F},\ x,y\in\SS.
	$$
\end{example}


\subsection{正交与正交补}
\begin{definition}[正交]
	设 $\SS$ 是内积空间, $x,y\in\SS$, 如果 $(x,y)=0$,
	则称 $x$ 与 $y$ \myind{正交}, 记为 $x\bot y$;
	
	设 $\SS_1$ 是 $\SS$ 的子空间, $x\in\SS$, 如果对任意 $y\in\SS_1$ 都有
	$ (x,y)=0$, 则称 $x$ 与 $\SS_1$ 正交, 记为 $x\bot\SS_1$;
	
	设 $\SS_1$, $\SS_2$ 是 $\SS$ 的两个子空间,
	如果对任意 $x\in\SS_1$, 都有 $x\bot\SS_2$,
	则称 $\SS_1$ 与 $\SS_2$ 正交, 记为 $\SS_1\bot\SS_2$.
\end{definition}

\begin{theorem}
	设 $\SS_1$, $\SS_2$ 是内积空间 $\SS$ 的两个子空间,
	如果 $\SS_1\bot\SS_2$, 则 $\SS_1+\SS_2$ 是直和.
\end{theorem}

\begin{definition}[正交补]
	设 $\SS_1$ 是内积空间 $\SS$ 的一个子空间, 则 $\SS_1$ 的\myind{正交补}定义为
	$$
	\SS_1^{\bot} \triangleq \big\{\ x\in\SS\ :\ x\bot \SS_1\ \big\},
	$$
	即 $\SS$ 中所有与 $\SS_1$ 正交的元素组成的集合.
\end{definition}

容易验证, 子空间 $\SS_1$ 的正交补 $\SS_1^{\bot}$ 也是 $\SS$ 的一个子空间.
另外, 我们还可以得到下面的结论.
\begin{theorem}
	设 $\SS_1$ 是内积空间 $\SS$ 的一个有限维子空间,
	则 $\SS_1^{\bot}$ 存在唯一, 且
	$$ \SS = \SS_1\oplus\SS_1^{\bot}.$$
\end{theorem}

\begin{example}
	设 $A\in\C^{m\times n}$, 则
	$$ \Ker(A^*) = \Ran(A)^\bot.$$
	\begin{proof}
		首先证明 $\Ker(A^*) \subseteq \Ran(A)^\bot$.
		设 $y\in\Ker(A^*)$, 则 $A^*y=0$.
		设 $z$ 是 $\Ran(A)$ 中的任意一个向量, 则存在 $x\in\C^n$, 使得 $z=Ax$.
		于是
		$$ z^*y = (Ax)^* y = x^*(A^*y) = 0, \quad \forall\ z\in\Ran(A), $$
		即 $y\in\Ran(A)^\bot$. 所以 $\Ker(A^*) \subseteq \Ran(A)^\bot$.
		
		另一方面, 设 $y\in\Ran(A)^\bot$, 则对任意向量 $z\in\Ran(A)$,
		都有 $y^*z=0$. 又 $AA^*y\in\Ran(A)$, 所以
		$$ (A^*y)^* (A^*y) = y^*(AA^*y)=0,$$
		即 $A^*y=0$, 也就是说 $y\in\Ker(A^*)$.
		所以 $\Ran(A)^\bot\subseteq\Ker(A^*).$
		由此可知, 结论成立.
	\end{proof}
\end{example}


\section{向量范数与矩阵范数}
\subsection{向量范数}
\begin{definition}[向量范数]
	若函数 $f:\C^n\to\R$ 满足
	\begin{nlist}
		\item $f(x)\geq 0$, $\forall\ x\in\C^n$ 且等号当且仅当 $x=0$ 时成立;
		(非负性, nonnegativity)
		\item $f(\alpha x)=|\alpha|\cdot f(x)$, $\forall\ x\in\C^n$, $\alpha\in\C$
		(正齐次性, homogeneity)
		\item $f(x+y)\leq f(x)+f(y)$, $\forall x,y\in\C^n$;
		(三角不等式, triangular inequality)
	\end{nlist}
	则称 $f(x)$ 为 $\C^n$ 上的 \myind{范数} (\myind{norm}),
	通常记作 $\|\cdot\|$.
\end{definition}

\begin{note}
	相类似地, 我们可以定义实数空间 $\R^n$ 上的向量范数.
\end{note}

\begin{note}
	如果 $f$ 只满足 $f(x)\geq 0$, 正齐次性和三角不等式,
	则称为 \myind{半范数} (\myind{seminorm}).
\end{note}

\begin{example}
	常见的向量范数:
	\begin{blist}
		\item $1$-范数:
		$\|x\|_1 = |x_1| + |x_2| + \cdots + |x_n|;$\medskip
		\item $2$-范数:
		$\|x\|_2 = \sqrt{|x_1|^2 + |x_2|^2 + \cdots + |x_n|^2};$\medskip
		\item $\infty$-范数:
		$\|x\|_\infty = \max\limits_{1\leq i\leq n} |x_i|;$\medskip
		\item $p$-范数:
		$\|x\|_p=\left(\sum\limits_{i=1}^n |x_i|^p\right)^{1/p}, \quad 1\leq p< \infty.$
	\end{blist}
\end{example}


\begin{definition}[范数的等价性]
	设 $\|\cdot\|_\alpha$ 与 $\|\cdot\|_\beta$ 是 $\C^n$ 空间上的两个向量范数,
	若存在正常数 $c_1$, $c_2$, 使得
	$$ c_1 \|x\|_\alpha \leq \|x\|_\beta \leq c_2 \|x\|_\alpha $$
	对任意 $x\in\C^n$ 都成立, 则称 $\|\cdot\|_\alpha$ 与 $\|\cdot\|_\beta$
	是等价的.
\end{definition}

\begin{theorem}\label{Th:01-norm10}
	$\C^n$ 空间上的所有向量范数都是等价的, 特别地, 有
	\begin{align*}
		& \|x\|_2 \leq \|x\|_1 \leq \sqrt{n}\ \|x\|_2 ,\\
		& \|x\|_\infty \leq \|x\|_1 \leq n\ \|x\|_\infty, \\
		& \|x\|_\infty \leq \|x\|_2 \leq \sqrt{n}\ \|x\|_\infty.
	\end{align*}
\end{theorem}

\begin{note}
	事实上, 有限维赋范线性空间上的所有范数都是等价的.
\end{note}

\begin{theorem}[Cauchy-Schwartz 不等式]\label{Th:01-norm11}
	设 \ip{\cdot,\cdot} 是 $\C^n$ 上的内积, 则对任意 $x,y\in\C^n$, 有
	$$|\ip{x,y}|^2 \leq \ip{x,x}\cdot \ip{y,y}, $$
	且等号成立的充要条件是: $x$ 与 $y$ 线性相关.
\end{theorem}
\begin{proof}
	若 $y=0$, 则结论显然成立.
	
	假设 $y\neq0$, 则对任意 $\alpha\in\C$ 有
	\begin{align*}
		0 & \leq \ip{x-\alpha y, x-\alpha y}
		= \ip{x,x} -\bar{\alpha}\ip{x,y}
		-\alpha\Big(\ip{y,x}-\bar{\alpha}\ip{y,y}\Big).
	\end{align*}
	由于 $y\neq0$, 所以 $\ip{y,y}>0$.
	取 $\bar\alpha=\dfrac{\ip{y,x}}{\ip{y,y}}$, 代入上式可得
	$$ 0\leq \ip{x,x} - \frac{\ip{y,x}}{\ip{y,y}}\ip{x,y}.$$
	由于 $\ip{y,x}=\ol{\ip{x,y}}$, 所以上式即为
	$$ |\ip{x,y}|^2 \leq \ip{x,x}\cdot \ip{y,y}.$$
	
	下面考虑等号成立的条件.
	
	充分性: 如果 $x$ 与 $y$ 线性相关, 则通过直接验证即可知等号成立.
	
	必要性: 假设等号成立. 如果 $y=0$, 则显然 $x$ 与 $y$ 线性相关.
	现假定 $y\neq 0$. 取 $\alpha=\dfrac{\ip{x,y}}{\ip{y,y}}$,
	则
	$$ \ip{x-\alpha y, x-\alpha y}
	=\ip{x,x} -\frac{|\ip{x,y}|^2}{\ip{y,y}}
	=0,
	$$
	即 $x-\alpha y=0$. 所以 $x$ 与 $y$ 线性相关.
\end{proof}

更一般地, 我们有下面的 Holder 不等式.
\begin{theorem}[Holder 不等式]
	设 $\ip{\cdot,\cdot}$ 是 $\C^n$ 上的 Euclidean 内积,
	则对任意 $x,y\in\C^n$, 有
	$$|\ip{x,y}| \leq \|x\|_p \cdot \|y\|_q, $$
	其中 $p,q>0$, 且 $\dfrac1p + \dfrac1q = 1$.
\end{theorem}
%
下面的结论告诉我们, 任何一个内积都可以定义一个相应的范数.
\begin{corollary}\label{Cor:innner-product}
	设 \ip{\cdot,\cdot} 是 $\C^n$ 上的内积,
	则 $\|x\|\triangleq \sqrt{\ip{x,x}}$ 是 $\C^n$ 上的一个向量范数.
\end{corollary}


\begin{theorem}[范数的连续性]
	设 $\|\cdot\|$ 是 $\C^n$ 上的一个向量范数, 则
	$f(x)\triangleq \|x\|$ 是 $\C^n$ 上的连续函数.
\end{theorem}


\section{课后习题}

\begin{exerlist}
	
	\item
	设 $A\in\C^{m\times n}$ ($m<n$) 是满秩矩阵,
	$Z\in\C^{n\times(n-m)}$ 是由 $\ker(A)$ 的一组基构成的矩阵.
	试证明:
	$$ \Ran(A^*) = \Ker(Z^*). $$
	
	\item 证明:
	$\|x\|_\infty=\lim\limits_{p\to\infty}%
	\left(\sum\limits_{i=1}^n |x_i|^p\right)^\frac1p$.
	
	\item 证明定理 \ref{Th:01-norm10} 中的三个不等式: \\
	(1) $\|x\|_2 \leq \|x\|_1 \leq \sqrt{n}\ \|x\|_2$\ ,\\
	(2) $\|x\|_\infty \leq \|x\|_1 \leq n\ \|x\|_\infty$ \ , \\
	(3) $\|x\|_\infty \leq \|x\|_2 \leq \sqrt{n}\ \|x\|_\infty$\ .
	
	
\end{exerlist}
