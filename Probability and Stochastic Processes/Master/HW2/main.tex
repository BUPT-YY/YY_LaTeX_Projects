\documentclass[AutoFakeBold,oneside,12pt]{ctexart}
\usepackage{geometry}

\geometry{a4paper,scale=0.9}

\newcommand{\myind}[1]{{\heiti\upshape\color{black} #1 }\index{#1}}

%%%%%%%% 这两个宏包冲突, 只可保留其中一个
%\usepackage{amsfonts}
\usepackage{amsmath,amssymb,bm}
\usepackage{esint}
\usepackage{mathrsfs}
%%%%%%%%

\usepackage{enumerate}
\usepackage{fancyhdr}
\usepackage{extramarks}
\usepackage{amsmath}
\usepackage{amsthm}

\usepackage{tikz}
\usepackage[plain]{algorithm}
\usepackage{algpseudocode}
\usepackage{amsthm}
\usepackage{mathtools}
\usepackage{physics}
\usepackage{calligra}
\usepackage{csquotes}
\usepackage{tensor}
\usepackage[thicklines]{cancel}
\usepackage{tcolorbox}
\usepackage{pstricks}
\usepackage{listings}

\usetikzlibrary{automata,positioning}

%
% Basic Document Settings
%

\topmargin=-0.45in
\evensidemargin=0in
\oddsidemargin=0in
\textwidth=6.5in
\textheight=9.0in
\headsep=0.25in

\linespread{1.1}

\pagestyle{fancy}
\lhead{\hmwkAuthorName}
\chead{\hmwkClass\ \hmwkClassInstructor\ \hmwkClassTime: \hmwkTitle}
\rhead{\firstxmark}
\lfoot{\lastxmark}
\cfoot{\thepage}

\renewcommand\headrulewidth{0.4pt}
\renewcommand\footrulewidth{0.4pt}

\setlength\parindent{0pt}

%
% Create Problem Sections
%
\newtheorem*{lemma}{{\color{blue}引理}}

\newcommand{\enterProblemHeader}[1]{
	\nobreak\extramarks{}{习题 \arabic{#1} 接下页\ldots}\nobreak{}
	\nobreak\extramarks{习题 \arabic{#1} (continued)}{习题 \arabic{#1} 接下页\ldots}\nobreak{}
}

\newcommand{\exitProblemHeader}[1]{
	\nobreak\extramarks{习题 \arabic{#1} (continued)}{习题 \arabic{#1} 接下页\ldots}\nobreak{}
	\stepcounter{#1}
	\nobreak\extramarks{习题 \arabic{#1}}{}\nobreak{}
}

\setcounter{secnumdepth}{0}
\newcounter{partCounter}
\newcounter{homeworkProblemCounter}
\setcounter{homeworkProblemCounter}{1}
\nobreak\extramarks{Problem \arabic{homeworkProblemCounter}}{}\nobreak{}

\newenvironment{homeworkProblem}{
	\section{习题 \arabic{homeworkProblemCounter}}
	\setcounter{partCounter}{1}
	\enterProblemHeader{homeworkProblemCounter}
}{
	\exitProblemHeader{homeworkProblemCounter}
}

%
% Homework Details
%   - Title
%   - Due date
%   - Class
%   - Section/Time
%   - Instructor
%   - Author
%

\newcommand{\hmwkTitle}{作业\ \#2}
\newcommand{\hmwkDueDate}{10月2日, 2019}
\newcommand{\hmwkClass}{概率论与随机过程}
\newcommand{\hmwkClassTime}{}
\newcommand{\hmwkClassInstructor}{}
\newcommand{\hmwkAuthorName}{杨勇,2019110294}

%
% Title Page
%

\title{
	\vspace{2in}
	\textmd{\textbf{\hmwkClass:\ \hmwkTitle}}\\
	\normalsize\vspace{0.1in}\small{完成于 \hmwkDueDate}\\
	\vspace{0.1in}\large{\textit{\hmwkClassInstructor\ \hmwkClassTime}}
	\vspace{3in}
}

\author{\textbf{\hmwkAuthorName}}
\date{}

\renewcommand{\part}[1]{\textbf{\large Part \Alph{partCounter}}\stepcounter{partCounter}\\}

%
% Various Helper Commands
%

% Useful for algorithms
\newcommand{\alg}[1]{\textsc{\bfseries \footnotesize #1}}

% For derivatives
\newcommand{\deriv}[1]{\frac{\mathrm{d}}{\mathrm{d}x} (#1)}

% For partial derivatives
\newcommand{\pderiv}[2]{\frac{\partial}{\partial #1} (#2)}

% Integral dx
\newcommand{\dx}{\mathrm{d}x}

% Alias for the Solution section header
\newcommand{\solution}{\textbf{\large 解.}}

% Probability commands: Expectation, Variance, Covariance, Bias
\newcommand{\E}{\mathrm{E}}
\newcommand{\Var}{\mathrm{Var}}
\newcommand{\Cov}{\mathrm{Cov}}
\newcommand{\Bias}{\mathrm{Bias}}

\begin{document}
	
	\maketitle
	
	
	
	%%%%%%%%%%%%%%%%%%%%%%%%%%%%%%%%%
	\pagebreak
	\begin{homeworkProblem}
		设$F(x) = P(\xi<x)$, 试证:$F(x)$单调不减、左连续, 且$F(-\infty) = 0$, $F(+\infty) = 1$.
		\begin{proof}
			若$x_1<x_2$, 则$\{ \xi<x_2 \} = \{ \xi<x_1\}  \cup\{ x_1\leqslant\xi<x_2 \}$, 
			注意到等式右边的两个集合不交, 故
			\begin{equation}
			 F(x_2) = P(\xi<x_2) = P(\xi<x_1)+P( x_1\leqslant\xi<x_2) \geqslant P(\xi<x_1) = F(x_1).
			\end{equation}
			由于函数$F(x)$单调且有界, 所以它必存在单侧极限. 为了证明$F(x-0)=F(x)$, 必须且只需对某一列$\{x_n\}$, $x_1<x_2<\cdots,x_n\to x$, 有$\lim\limits_{n\to\infty}F(x_n) = F(x)$即可. 令
			\begin{equation}
				A_n = \{ x_n\leqslant \xi < x \},
			\end{equation}
			则$A_n\supset A_{n+1}$ 且$\bigcap\limits_{n=1}^{\infty}A_n = \varnothing$, 利用概率测度在$\varnothing$处上连续得到:
			\begin{equation}
				\lim_{n\to\infty}F(x_n) - F(x) = \lim_{n\to\infty}[F(x_n) - F(x)] = \lim_{n\to\infty}P(A_n) = P(\varnothing) = 0.
			\end{equation}
			令$B_n = \{ \xi<-n \}$, 则$B_n\downarrow \varnothing$; 类似地, 令$C_n = \{\xi<n\}$,则$C_n\uparrow\Omega$.
			由$F$的单调性和概率测度$P$的连续性,
			\begin{align}
				F(+\infty) &= \lim_{n\to\infty}F(n) = \lim_{n\to\infty}P(C_n) = P(\Omega) = 1,\nonumber\\
				F(-\infty) &= \lim_{n\to\infty}F(-n) = \lim_{n\to\infty}P(B_n) =  P(\varnothing) = 0.
			\end{align}
		\end{proof}
	\end{homeworkProblem}
	
	
	\begin{homeworkProblem}
		设随机变量$\xi$取值于$(0,1)$, 若对一切$0<x\leqslant y<1$,$P(x<\xi\leqslant y)$只与长度$y-x$有关, 试证:
		$\xi$服从$(0,1)$上的均匀分布.
	\end{homeworkProblem}
	
	\begin{homeworkProblem}
		试证:$f(x,y) = k\mathrm{e}^{-(ax^2+2bxy+cy^2)}$为分布密度的充要条件是$a>0$,$c>0$,$ac-b^2>0$,$k=\sqrt{ac-b^2}/\pi$.
	\end{homeworkProblem}

	\begin{homeworkProblem}
		若$(\xi,\eta)$的分布密度为\begin{equation}
			f(x,y) = \begin{dcases}
				A\mathrm{e}^{-(2x+y)},&x>0,y>0,\\
				0,&\text{otherwise},
			\end{dcases}
		\end{equation}
		试求:\begin{enumerate}[(1)]
			\item 常数$A$;
			\item 关于$\xi$,$\eta$的边沿密度;
			\item $f_{\xi|\eta}(x|y)$;
			\item $P(\xi\leqslant x|\eta<1)$.
		\end{enumerate}
	\end{homeworkProblem}
	
	\begin{homeworkProblem}
		求证: 若$F(x)$为分布函数, 则对任意$h>0$, 函数
		\begin{equation}
			\varPhi(x) = \frac{1}{h}\int_{x}^{x+h}F(y)\mathrm{d}y,
			\varPhi(x) = \frac{1}{2h}\int_{x-h}^{x+h}F(y)\mathrm{d}y
		\end{equation}
		都是分布函数.
	\end{homeworkProblem}
	
	\begin{homeworkProblem}
		设$\xi,\eta$独立, 且都服从Poisson分布,\begin{align}
			P(\xi = m) &= \frac{\lambda_1^m}{m!}\mathrm{e}^{-\lambda_1},~m=0,1,2,\cdots\nonumber\\
			 P(\eta = n) &= \frac{\lambda_2^n}{n!}\mathrm{e}^{-\lambda_2},~n=0,1,2,\cdots
		\end{align}
		求证:\begin{enumerate}[(1)]
			\item $\xi+\eta$仍服从Poisson分布;
			\item \begin{equation}
				P(\xi=k|\xi+\eta=N) = \binom{N}{k}\left(\frac{\lambda_1}{\lambda_1+\lambda_2}\right)^k\left(\frac{\lambda_2}{\lambda_1+\lambda_2}\right)^{N-k},~k=0,1,\cdots,N.
			\end{equation}
		\end{enumerate}
	\end{homeworkProblem}
	   
	\begin{homeworkProblem}
	    设$\xi,\eta$独立, 且分布密度分别为
	    \begin{equation}
	        f_{\xi}(x) = \begin{dcases}
	            \frac{1}{2}, &1<x<3,\\
	            0,&\text{otherwise},
	        \end{dcases}
	        \quad f_{\eta}(y) = \begin{dcases}
	            \mathrm{e}^{-(y-2)},&y>2,\\
	            0,&\text{otherwise},
	        \end{dcases}
	    \end{equation}
	    求证:\begin{equation}
	        f_{\xi/\eta}(x) = \begin{dcases}
	            \frac{1}{2x}\mathrm{e}^2\left[\mathrm{e}^{-1/x}(1+x)-\mathrm{e}^{-3/x}(x+3)\right], &0<x<\frac{1}{2},\\
	            \frac{3}{2}-\frac{\mathrm{e}^{2}}{2x}\mathrm{e}^{-3/x}(x+3),&\frac{1}{2}<x<\frac{3}{2},\\
	            0,&\text{otherwise}.
	        \end{dcases}
	    \end{equation}
	\end{homeworkProblem}
	
	\begin{homeworkProblem}
	    设$f_1(x),f_2(x),f_3(x)$对应的分布函数为$F_1(x),F_2(x),F_3(x)$, 证明: 对一切$\alpha$($-1<\alpha<1$), 下列函数是分布密度, 且有相应的边沿密度$f_1(x),f_2(x),f_3(x)$,
	    \begin{align}
	        &f_{\alpha}(x_1,x_2,x_3) \nonumber\\
	        =&f_1(x_1)f_2(x_2)f_3(x_3)\{ 1+\alpha[2F_1(x_1)-1][2F_2(x_2)-1][2F_3(x_3)-1] \}.
	    \end{align}
	\end{homeworkProblem}
	
	\begin{homeworkProblem}
	    设$(\xi,\eta,\zeta)$的分布密度为\begin{equation}
	        f(x,y,z) =\begin{dcases}
	            \frac{6}{(1+x+y+z)^4},&x>0,y>0,z>0,\\
	            0,&\text{otherwise},
	        \end{dcases}
	    \end{equation}
	    试求$U = \xi+\eta+\zeta$的分布密度.
	    
	    \solution 我们愿意用一个初等的结论:
	        设$g(x)$可积, 并设$\Omega_a$是$\mathbb{R}^n$中的如下区域:
	        \begin{equation}
	            \Omega_a = \{(x_1,\cdots,x_n):0\leqslant \sum_{j=1}^nx_j\leqslant a,x_j\leqslant 0,j=1,\cdots,n\}.
	        \end{equation}
	        则,
	        \begin{equation}
	            \int_{\Omega_a} g(x_1+\cdots+x_n)\mathrm{d}x_1\cdots\mathrm{d}x_n
	            =\int_{0}^{a}\frac{x^{n-1}}{(n-1)!}g(x)\mathrm{d}x.
	        \end{equation}
        由此, 可以第五节(一)中的方法求解本题:
	        \begin{equation}
	            F_U(u) = \iiint_{\{x+y+z\leqslant u\}}f(x,y,z)\mathrm{d}x\mathrm{d}y\mathrm{d}z = \int_{0}^{u}\frac{3x^2}{(1+x)^4}\mathrm{d}x,
	        \end{equation}
	        可见, $U$的概率密度函数为\begin{equation}
	            f_U(u) = \begin{dcases}
	                \frac{3u^2}{(1+u)^4}, &u>0,\\
	                0,&u\leqslant 0.
	            \end{dcases}
	        \end{equation}
	        
	\end{homeworkProblem}
	
    \begin{homeworkProblem}
        设$\xi,\eta$独立且均服从$\mathcal{N}(0,1)$,证明$U = \xi^2+\eta^2$与$V=\dfrac{\xi}{\eta}$是独立的.
    \end{homeworkProblem}
    
    \begin{homeworkProblem}
        设$\xi,\eta$独立且它们的分布密度为\begin{equation}
            f_{\xi}(x) = f_{\eta}(x) = \begin{dcases}
                \mathrm{e}^{-x},&x>0,\\
                0,&x\leqslant 0.
            \end{dcases}
        \end{equation}
        试研究$\xi+\eta$与$\dfrac{\xi}{\xi+\eta}$是否独立.
    \end{homeworkProblem}
    
    \begin{homeworkProblem}
        证明: 任一广义分布函数最多有可列个不连续点.
        \begin{proof}
            设$F$是一个广义分布函数, 即$F$是一个$\mathbb{R}$上的非降右连续的实值函数. 若$x_0$是$F(x)$的不连续点, 则有
            \begin{equation} F(x_0-0)<F(x_0+0).
            \end{equation}
            因此, $x_0$就对应着一个开区间$(F(x_0-0),F(x_0+0))$. 对于两个不同的不连续点$x_1$及$x_2$, 区间$(F(x_1-0),F(x_1+0))$与$(F(x_2-0),F(x_2+0))$不交. 因而, $F$的不连续点构成一个$\mathbb{R}$上的互不相交的开区间族, 所以它是至多可列集.(Remark: 设$\mathscr{G}$是$\mathbb{R}$中互不相交的开区间族,可从每个区间取一个有理数, 而有理数是可列集, 从而$\mathscr{G}$是至多可列集).
        \end{proof}
    \end{homeworkProblem}
    
\end{document}
