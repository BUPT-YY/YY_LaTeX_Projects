\documentclass[AutoFakeBold,oneside,12pt]{ctexart}
\usepackage{geometry}

\geometry{a4paper,scale=0.9}

\newcommand{\myind}[1]{{\heiti\upshape\color{black} #1 }\index{#1}}

%%%%%%%% 这两个宏包冲突, 只可保留其中一个
%\usepackage{amsfonts}
\usepackage{mathrsfs}
%%%%%%%%


\usepackage{fancyhdr}
\usepackage{extramarks}
\usepackage{amsmath}
\usepackage{amsthm}

\usepackage{tikz}
\usepackage[plain]{algorithm}
\usepackage{algpseudocode}
\usepackage{amsthm}
\usepackage{mathtools}
\usepackage{bm}
\usepackage{physics}
\usepackage{calligra}
\usepackage{csquotes}
\usepackage{tensor}
\usepackage[thicklines]{cancel}
\usepackage{tcolorbox}
\usepackage{pstricks}
\usepackage{amssymb}
\usepackage{listings}
\usepackage{enumerate}

\usetikzlibrary{automata,positioning}

%
% Basic Document Settings
%

\topmargin=-0.45in
\evensidemargin=0in
\oddsidemargin=0in
\textwidth=6.5in
\textheight=9.0in
\headsep=0.25in

\linespread{1.1}

\pagestyle{fancy}
\lhead{\hmwkAuthorName}
\chead{\hmwkClass\ \hmwkClassInstructor\ \hmwkClassTime: \hmwkTitle}
\rhead{\firstxmark}
\lfoot{\lastxmark}
\cfoot{\thepage}

\renewcommand\headrulewidth{0.4pt}
\renewcommand\footrulewidth{0.4pt}

\setlength\parindent{0pt}

%
% Create Problem Sections
%

\newcommand{\enterProblemHeader}[1]{
	\nobreak\extramarks{}{习题 \arabic{#1} 接下页\ldots}\nobreak{}
	\nobreak\extramarks{习题 \arabic{#1} (continued)}{习题 \arabic{#1} 接下页\ldots}\nobreak{}
}

\newcommand{\exitProblemHeader}[1]{
	\nobreak\extramarks{习题 \arabic{#1} (continued)}{习题 \arabic{#1} 接下页\ldots}\nobreak{}
	\stepcounter{#1}
	\nobreak\extramarks{习题 \arabic{#1}}{}\nobreak{}
}

\setcounter{secnumdepth}{0}
\newcounter{partCounter}
\newcounter{homeworkProblemCounter}
\setcounter{homeworkProblemCounter}{1}
\nobreak\extramarks{Problem \arabic{homeworkProblemCounter}}{}\nobreak{}

\newenvironment{homeworkProblem}{
	\section{习题 \arabic{homeworkProblemCounter}}
	\setcounter{partCounter}{1}
	\enterProblemHeader{homeworkProblemCounter}
}{
	\exitProblemHeader{homeworkProblemCounter}
}

%
% Homework Details
%   - Title
%   - Due date
%   - Class
%   - Section/Time
%   - Instructor
%   - Author
%

\newcommand{\hmwkTitle}{作业\ \#1}
\newcommand{\hmwkDueDate}{9月22日, 2019}
\newcommand{\hmwkClass}{概率论与随机过程}
\newcommand{\hmwkClassTime}{}
\newcommand{\hmwkClassInstructor}{}
\newcommand{\hmwkAuthorName}{杨勇,2019110294}

%
% Title Page
%

\title{
	\vspace{2in}
	\textmd{\textbf{\hmwkClass:\ \hmwkTitle}}\\
	\normalsize\vspace{0.1in}\small{完成于 \hmwkDueDate}\\
	\vspace{0.1in}\large{\textit{\hmwkClassInstructor\ \hmwkClassTime}}
	\vspace{3in}
}

\author{\textbf{\hmwkAuthorName}}
\date{}

\renewcommand{\part}[1]{\textbf{\large Part \Alph{partCounter}}\stepcounter{partCounter}\\}

%
% Various Helper Commands
%

% Useful for algorithms
\newcommand{\alg}[1]{\textsc{\bfseries \footnotesize #1}}

% For derivatives
\newcommand{\deriv}[1]{\frac{\mathrm{d}}{\mathrm{d}x} (#1)}

% For partial derivatives
\newcommand{\pderiv}[2]{\frac{\partial}{\partial #1} (#2)}

% Integral dx
\newcommand{\dx}{\mathrm{d}x}

% Alias for the Solution section header
\newcommand{\solution}{\textbf{\large 解.}}

% Probability commands: Expectation, Variance, Covariance, Bias
\newcommand{\E}{\mathrm{E}}
\newcommand{\Var}{\mathrm{Var}}
\newcommand{\Cov}{\mathrm{Cov}}
\newcommand{\Bias}{\mathrm{Bias}}

\begin{document}
	
	\maketitle
	
	
	
	%%%%%%%%%%%%%%%%%%%%%%%%%%%%%%%%%
	\pagebreak
	\begin{homeworkProblem}
		设$A_1,A_2,\cdots$为任一集序列, 若令
		\begin{equation}
			A_1' = A_1,~A_n' = A_n\backslash \bigcup_{k=1}^{n-1}A_k,n=2,3,\cdot.
		\end{equation}
		试证:$A_1',\cdots,A_n',\cdots$两两互不相交, 且
		\begin{equation}
			\bigcup_{n=1}^{+\infty}A_n = \bigcup_{n=1}^{+\infty}A_n'.
		\end{equation}
		\begin{proof}
			为表达方便, 令$A_0 = \varnothing$.
			先证明$\{A_n':n\in\mathbb{N}^*\}$互不相交. 据De-Morgan律知
			\begin{equation}
				A_n' = A_nA_0^c\cdots A_{n-1}^c
			\end{equation}
			对任何不等的$j,k\in\mathbb{N}^*$, 无妨$j<k$, 这时有$A_k'\subset A_j^c$, 而$A_j'\subset A_j$, 这说明$A_k'A_j' = \varnothing$.
			
			一方面,若$\omega\in \bigcup_{n=1}^{+\infty}A_n'$, 则$\exists n\in\mathbb{N}^*(\omega\in A_n').$
			由于$A_n'\subset A_n$, 故有$\exists n\in\mathbb{N}^*(\omega\in A_n).$
			即\begin{equation}\label{eq:subset}
				\bigcup_{n=1}^{+\infty}A_n' \subset \bigcup_{n=1}^{+\infty}A_n.
			\end{equation}
			令一方面, 设$\omega\in \bigcup_{n=1}^{+\infty}A_n$, 根据并的定义知:$\exists n\in\mathbb{N}^*(\omega\in A_n).$
				
			此时, 必有整数$k\in\{1,\cdots,n\}$使得$\omega\notin A_0,\cdots,A_{k-1}$, 但$\omega\in A_k$. (否则$\omega\notin \bigcup_{k=1}^nA_k$,矛盾)
			即$\omega\in A_k'$.
			所以有$\omega\in \bigcup_{n=1}^{+\infty}A_n'$.
			
			这说明\begin{equation}\label{eq:supset}
				\bigcup_{n=1}^{+\infty}A_n \subset \bigcup_{n=1}^{+\infty}A_n'.
			\end{equation}
			结合式\ref{eq:subset}与\ref{eq:supset}知道结论成立.
		\end{proof}
		
	\end{homeworkProblem}
	
	\begin{homeworkProblem}
		设$(\Omega,\mathscr{F})$为一可测空间, $A_n\in\mathscr{F},n=1,2,\cdots$, 试证:
		\begin{equation}
			\bigcap_{n=1}^{+\infty}\bigcup_{k=n}^{+\infty}A_k = \{ \omega:\omega\text{属于无穷多个}A_n \},~~\bigcup_{n=1}^{+\infty}\bigcap_{k=n}^{+\infty}A_k = \{ \omega:\omega\text{只不属于有限多个}A_n \}.
		\end{equation}
		\begin{proof}
			(1)若$\omega\in\bigcap_{n=1}^{+\infty}\bigcup_{k=n}^{+\infty}A_k$, 则根据交与并的定义知道:\begin{equation}
				\forall n\in\mathbb{N}^*,\exists k\geqslant n~\left( \omega\in A_k \right).
			\end{equation}
			这说明$\omega\in\bigcap_{n=1}^{+\infty}\bigcup_{k=n}^{+\infty}A_k$当且仅当$\omega$属于集列$\{A_j\}$中的无穷多个集合.
			
			(2)若$\omega\in\bigcup_{n=1}^{+\infty}\bigcap_{k=n}^{+\infty}A_k$, 则根据交与并的定义知道:
			\begin{equation}
				\exists j_0\in\mathbb{N}^*,\forall k\geqslant j_0~\left( x\in A_k \right).
			\end{equation}
			这说明$\omega\in\bigcup_{n=1}^{+\infty}\bigcap_{k=n}^{+\infty}A_k$当且仅当$\omega$仅不属于集列$\{A_j\}$中的有限多个集合.
		\end{proof}
	\end{homeworkProblem}

	\begin{homeworkProblem}
		若记\begin{equation}
			\bigcap_{n=1}^{+\infty}\bigcup_{k=n}^{+\infty}A_k = A^{\star} = \limsup_{n\to\infty}A_n,
		\end{equation}
		称为集序列$\{A_n\}$的上限集. 若记
		\begin{equation}
		\bigcup_{n=1}^{+\infty}\bigcap_{k=n}^{+\infty}A_k = A_{\star} = \liminf_{n\to\infty}A_n,
		\end{equation}
		称为集序列$\{A_n\}$的下限集. 试证:
		\begin{enumerate}
			\item $\liminf\limits_{n\to\infty}A_n\subset \limsup\limits_{n\to\infty}A_n$;
			\item 若$A_n\uparrow$, 则$A^\star = A_\star = \cup_{n=1}^{+\infty}A_n$;
			\item 若$A_n\downarrow$, 则$A^\star = A_\star = \cap_{n=1}^{+\infty}A_n$;
			\item 若$A$为任一集合, 则$A\backslash A_\star = \limsup\limits_{n\to\infty}(A\backslash A_n),A\backslash A^\star = \liminf\limits_{n\to\infty}(A\backslash A_n)$
		\end{enumerate}
		\begin{proof}
			\begin{enumerate}
				\item 根据习题2知道: $\omega\in A_\star$当且仅当
				除去集列$\{A_j\}$中的有限个集合外, 元$\omega$属于该序列的其余集合. 这可推出 $\omega$属于集列$\{A_j\}$中的无穷多个集, 也就是$\omega\in A^\star$. 即证.
				\item 不难看出对一般的集合列$\{A_j\}$,总有下列的包含关系:
				\begin{equation}\label{eq:包含关系}
					\bigcap_{n=1}^{+\infty}A_n\subset \liminf_{n\to\infty}A_n \subset \limsup_{n\to\infty}A_n\subset \bigcup_{n=1}^{+\infty}A_n
				\end{equation}
				若$\{A_n\}$非降, 则$\bigcap_{k=n}^{+\infty}A_k = A_n$.
				所以$\liminf\limits_{n\to\infty} A_n = \bigcup_{n=1}^{+\infty}A_n$. 
				结合式\ref{eq:包含关系}知结论成立.
				\item $\{A_n\}$非增, 则$\bigcup_{k=n}^{+\infty}A_k = A_n$.
				所以$\limsup\limits_{n\to\infty} A_n = \bigcap_{n=1}^{+\infty}A_n$.
				结合式\ref{eq:包含关系}知结论成立.
				\item 回忆de Morgen法则:
				\begin{equation}
					A\backslash \bigcup_{\lambda\in\Lambda}B_{\lambda} = \bigcap_{\lambda\in\Lambda}(A\backslash B_{\lambda}),~~A\backslash \bigcap_{\lambda\in\Lambda}B_{\lambda} = \bigcup_{\lambda\in\Lambda}(A\backslash B_{\lambda}).
				\end{equation}
				我们由此知道
				\begin{align}
					A\backslash \liminf_{n\to\infty}A_n &= A\backslash \bigcup_{n=1}^{+\infty}\bigcap_{k=n}^{+\infty}A_k=\bigcap_{n=1}^{+\infty}\left( A\backslash \bigcap_{k=n}^{+\infty}A_k \right)\nonumber\\
					&= \bigcap_{n=1}^{+\infty}\bigcup_{k=n}^{+\infty}\left( A\backslash A_k \right)=\limsup_{n\to\infty}(A\backslash A_n),
				\end{align}
				和\begin{align}
				A\backslash \limsup_{n\to\infty}A_n &= A\backslash \bigcap_{n=1}^{+\infty}\bigcup_{k=n}^{+\infty}A_k=\bigcup_{n=1}^{+\infty}\left( A\backslash \bigcup_{k=n}^{+\infty}A_k \right)\nonumber\\
				&= \bigcup_{n=1}^{+\infty}\bigcap_{k=n}^{+\infty}\left( A\backslash A_k \right)=\liminf_{n\to\infty}(A\backslash A_n).
				\end{align}
			\end{enumerate}
		\end{proof}
	\end{homeworkProblem}
	
	
	
	\begin{homeworkProblem}
		证明: 包含一切形如$(-\infty,x)$的区间的最小$\sigma$-代数是一维Borel域.
		\begin{proof}
			根据定义,一维Borel集合系$\mathscr{B}_\mathbb{R}$是由$\pi$-系$\mathscr{P}_\mathbb{R} = \{(-\infty,a]:a\in\mathbb{R}\}$生成的$\sigma$-代数:
			\begin{equation}
				\mathscr{B}_R = \sigma(\mathscr{P}_\mathbb{R}).
			\end{equation}
			对任意$a\in\mathbb{R}$,我们有
			\begin{align}
				&\bigcap_{n=1}^{+\infty}\left(-\infty,a+\frac{1}{n}\right) = (-\infty,a],~~\bigcup_{n=1}^{+\infty}\left(-\infty,a-\frac{1}{n}\right] = (-\infty,a)
			\end{align}
			因此, 对任何$x\in\mathbb{R},(-\infty,x)\in\mathscr{B}_{\mathbb{R}}$.
			所以$\sigma(\{ (-\infty,x):x\in\mathbb{R} \})\subset \mathscr{B}_{\mathbb{R}}$.
			
			另外, 对任何$x\in\mathbb{R},(-\infty,x]\in\sigma(\{ (-\infty,x):x\in\mathbb{R} \})$.
			所以$\sigma(\{ (-\infty,x):x\in\mathbb{R} \})\supset \mathscr{B}_{\mathbb{R}}$.
			
			二者结合起来便说明$\sigma(\{ (-\infty,x):x\in\mathbb{R} \})= \mathscr{B}_{\mathbb{R}}$.
		\end{proof}
	\end{homeworkProblem}
	
	\begin{homeworkProblem}
		求包含二集合$A,B$的最小$\sigma$-代数, 其中$\Omega,AB\neq\varnothing, A\cup B\neq\Omega$, 且$A,B$互不包含.
		
		{\kaishu{解}.}
		所求的集合系为:\begin{equation}
		    \mathscr{A} = \{\varnothing,A,B,A\cup B,A\triangle B,A\backslash B, B\backslash A,\Omega,A^c,B^c,A^cB^c,AB^c,BA^c,(AB)^c,(A\triangle B)^c \}
		\end{equation}
		\begin{proof}
			由于$\mathscr{A}$对任何可列次的集合运算封闭, 且$\Omega\in\mathscr{A}$, 所以$\mathscr{A}$是$\sigma$代数. 又因为$\sigma$代数对有限次的集合运算都封闭, 所以任何一个包含$\{A,B\}$的$\sigma$代数都包含$\mathscr{A}$中的所有集. 因此, 集类$\mathscr{A} = \sigma(\{A,B\})$.
		\end{proof}
	\end{homeworkProblem}
	
	
	\begin{homeworkProblem}
		若$\mathscr{G} = \{A_k:A_k\subset\Omega,k=1,2,\cdots,\text{两两不交}\}$, 试求$\sigma(\mathscr{G})$.
		
		\solution
		引入$A_0 = \left(\bigcup\limits_{k=1}^{+\infty}A_k\right)^c$, 则所求为:
		\begin{equation}\label{习题6}
			\sigma(\mathscr{G}) = \left\{ \bigcup_{k\in K}A_{k}:K\subset\mathbb{N} \right\}.
		\end{equation}
		\begin{proof}
		    不妨记(\ref{习题6})式右端为$\mathscr{D}$. 由于$\sigma$代数是对于可列并的运算以及补的运算封闭的, 故$\sigma(\mathscr{G}) \supset \mathscr{D}$. 因此, 要完成这个定理的证明, 必须且只需证$\sigma(\mathscr{G}) \subset \mathscr{D}$. 为此, 又只需证$\mathscr{D}$是$\sigma$-代数. 下面, 分别验证$\sigma$-代数的三个条件:
		    \begin{enumerate}[1)]
		        \item $\Omega = \bigcup\limits_{k=0}^{+\infty}A_k \in \mathscr{D}$;
		        \item $A = \bigcup\limits_{k\in K}A_{k}\in \mathscr{D} \Rightarrow A^c = \bigcup\limits_{k\in (\mathbb{N}\backslash K)}A_{k} \in\mathscr{D}$;
		        \item 对$B_n\in\mathscr{D},n\in\mathbb{N}^*$, 有对应的$K_n\in\mathbb{N}$, 使$
		            B_n = \bigcup_{k\in K_n}A_{k}, n=1,2,\cdots$.
		        这时,\begin{equation}
		            \bigcup_{n=1}^{+\infty}B_n = \bigcup_{n=1}^{+\infty}\bigcup_{k\in K_n}A_{k} = \bigcup_{k\in K}A_{k}\in\mathscr{D},~~K = \bigcup_{n=1}^{+\infty}K_n.
		        \end{equation}
		    \end{enumerate}
		    由此知道$\mathscr{D}$确是$\sigma$代数.
		\end{proof}
	\end{homeworkProblem}
	
	\begin{homeworkProblem}
		设$\Omega$是不可列集, $\mathscr{A}$是$\Omega$的一切有限子集、可列子集及以有限子集或可列子集为余集的子集所作成的集合类, 试证$\mathscr{A}$是一$\sigma$-代数.
		\begin{proof}
		    对集类$\mathscr{A}$分别验证$\sigma$-代数的三个条件:
		    \begin{enumerate}[1)]
		        \item 由于$\varnothing$是$\Omega$的一个有限子集, 所以 $\Omega = \varnothing^{c} \in \mathscr{A}$;
		        \item 设$A\in\mathscr{A}$, 即$A$是$\Omega$的至多可列子集或 以$\Omega$的至多可列子集为余集的子集, 则 $A^c$分别是$\Omega$的以至多可列子集为余集的子集 或 $\Omega$的至多可列子集, 因此, $A^c\in\mathscr{A}$;
                                                                                     		        
		        \item 设$A_1,A_2,\cdots \in \mathscr{A}$. 若诸$A_j$均是$\Omega$的至多可列子集, 则$\cup_j A_j$仍是$\Omega$的至多可列子集; 若诸$A_j$中至少有一个是 以$\Omega$的至多可列子集为余集的子集, 则$\cup_j A_j$仍是以$\Omega$的至多可列子集为余集的子集.
		        
		        所以总有$\bigcup\limits_{n=1}^{+\infty}A_n\in\mathscr{A}$.
		    \end{enumerate}
		    由此知道$\mathscr{A}$确是$\sigma$-代数.
		\end{proof}
	\end{homeworkProblem}

	\begin{homeworkProblem}
		设$\mathscr{G}$是由$\Omega$的子集组成的集合类, $A$是$\Omega$的一个非空子集, 令\begin{equation}
			\mathscr{G}\cap A = \{ A':A'=B\cap A, B\in\mathscr{G} \},
		\end{equation}
		试证:$\sigma(\mathscr{G})\cap A$是以$A$为空间的包含集合类$\mathscr{G}\cap A$的最小$\sigma$-代数.
		\begin{proof}
		    先证明$\sigma(\mathscr{G})\cap A = \{BA: B\in \sigma(\mathscr{G})\}$是以$A$为空间的一个$\sigma$-代数. 
		    为此, 我们分别对它验证$\sigma$-代数的三个条件.
		    \begin{enumerate}[1)]
		        \item 因$\Omega\in\sigma(\mathscr{G})$, 所以$A = \Omega\cap A\in \sigma(\mathscr{G})\cap A$;
		        \item 设$E\in \sigma(\mathscr{G})\cap A$. 即$E = AB$, 其中$B\in\sigma(\mathscr{G})$.
		        $E$相对$A$的补是$A\backslash E = A\backslash AB = AB^c$. 因为$B^c\in\sigma(\mathscr{G})$, 所以$A\backslash E\in \sigma(\mathscr{G})\cap A$;
		        \item 设$E_1,E_2,\cdots \in \sigma(\mathscr{G})\cap A$. 即$E_j = AB_j$, 其中$B_j\in\sigma(\mathscr{G}),j \in\mathbb{N}^*$. 由于$\sigma(\mathscr{G})$是一个$\sigma$-代数, 所以$\cup_{j=1}^{+\infty}B_j\in\sigma(\mathscr{G})$. 这说明
		        \begin{equation}
		            \bigcup_{j = 1}^{+\infty}E_j = \bigcup_{j = 1}^{+\infty}AB_j  =A\left(\bigcup_{j = 1}^{+\infty}B_j \right)\in \sigma(\mathscr{G})\cap A.
		        \end{equation}
		    \end{enumerate}
		    再设某个以$A$为空间的$\sigma$-代数$\mathscr{E}\supset (\mathscr{G}\cap A)$, 下证$ \left(\sigma(\mathscr{G})\cap A\right)\subset \mathscr{E}$.
		    因$\mathscr{E}\supset (\mathscr{G}\cap A)$, 所以$\forall B\in\mathscr{G}, AB\in\mathscr{E}$.
		    令\begin{equation}
		        \mathscr{H}_A = \{H\subset\Omega:AH\in\mathscr{E}\}.
		    \end{equation}
		    则$\mathscr{G}\subset\mathscr{H}_A$. 继续证明$\mathscr{H}_A$是一个$\sigma$-代数.
		    注意到$\mathscr{E}$是以$A$为空间的一个$\sigma$-代数, 所以有
		    \begin{enumerate}[1)]
		        \item $A\Omega =A\in\mathscr{E}$, 即$\Omega\in \mathscr{H}_A$;
		        \item 设$H\in \mathscr{H}_A$, 则$AH\in\mathscr{E}$. 利用上面条件,有$AH^c = A\backslash AH\in\mathscr{E}$. 这说明$H^c\in \mathscr{H}_A$;
		        \item 再设$H_1,H_2,\cdots\in \mathscr{H}_A$, 则$AH_j\in\mathscr{E},j=1,2,\cdots$. 再次利用$\mathscr{E}$是以$A$为空间的一个$\sigma$-代数, 得到\begin{equation}
		            A\left(\bigcup_{j=1}^{+\infty}H_j\right) = \bigcup_{j=1}^{+\infty}AH_j\in \mathscr{E}.
		        \end{equation}
		        这说明$\bigcup_{j=1}^{+\infty}H_j\in \mathscr{H}_A$.
		    \end{enumerate}
		    因此, $\mathscr{H}_A$确是个$\sigma$-代数. 又因为$\mathscr{G}\subset\mathscr{H}_A$. 根据$\sigma$-代数的生成的定义, $\sigma(\mathscr{G})\subset \mathscr{H}_A$.
		    
		    即\begin{equation}
		        \forall B\in \sigma(\mathscr{G}),\text{总有}BA\in\mathscr{E}.
		    \end{equation}
		    这正是$ \left(\sigma(\mathscr{G})\cap A\right)\subset \mathscr{E}$.
		\end{proof}
	\end{homeworkProblem}

	\begin{homeworkProblem}
		设$A$是$\Omega$的一个子集, $\mathscr{G}$是$\Omega$的包含$A$的一切子集所组成的集合类, 试问$\sigma(\mathscr{G})$是由哪些子集组成的?
		
		\solution
		\begin{equation}
		    \sigma(\mathscr{G}) = \mathscr{G}\cup\{ G\cup A^c:G\in\mathscr{G} \}.
		\end{equation}
	\end{homeworkProblem}

	\begin{homeworkProblem}
		设$\xi(\omega)$是定义在$\Omega$上而值域为$\mathbb{R}^{(1)}$的单值实函数, $B\subset\mathbb{R}^{(1)}$, $\{\omega:\xi(\omega)\in B\}$表示使$\xi(\omega)$的值属于$B$的一切$\omega(\in\Omega)$的集合, 试证:\begin{enumerate}
			\item $\overline{\{\omega:\xi(\omega)\in B  \}} = \{ \omega:\xi(\omega)\in\overline{B} \}$;
			\item 若$B_k\subset\mathbb{R}^{(1)},k=1,2,\cdots$, 则有\begin{equation}
				\bigcup_{k=1}^{+\infty}\{\omega:\xi(\omega)\in  B_k\} = \left\{ \omega:\xi(\omega)\in\bigcup_{k=1}^{+\infty}B_k \right\}.
			\end{equation}
		\end{enumerate}
        \textbf{分析:}
		    回忆数学分析中引入的\myind{映射的原像}(preimage/inverse image).
		    
		    设映射$\varphi:A\to B$, 集合$D\subset B$的原像定义为
		    \begin{equation}
		        \varphi^{-1}(D) \triangleq \{x\in A:\varphi(x)\in D\}.
		    \end{equation}
		    那时候, 曾经证明过这样两个关于映射与集合运算之间关系的公式.
		    \begin{align}
		        &\varphi^{-1}\left( \bigcup_{\lambda\in\Lambda}E_\lambda \right) = \bigcup_{\lambda\in\Lambda}\varphi^{-1}\left( E_\lambda \right); \\
		        &\left(\varphi^{-1}(D)\right)^c = \varphi^{-1}(D^c).
		    \end{align}
            我们愿意在此重新温习一下它的证明.
            \begin{proof}
                \begin{align}
                    x\in \varphi^{-1}\left( \bigcup_{\lambda\in\Lambda}E_\lambda \right) &\Longleftrightarrow \exists y\in \bigcup_{\lambda\in\Lambda}E_{\lambda}\left(y = \varphi(x)\right)\nonumber\\
                    &\Longleftrightarrow \exists \lambda\in\Lambda\exists y\in E_{\lambda}\left(y = \varphi(x)\right)\nonumber\\
                    &\Longleftrightarrow \exists \lambda\in\Lambda \left(x\in\varphi^{-1}(E_{\lambda})\right).
                \end{align}
                还需要讨论空集的情况.
                \begin{align}
                   ~~\varphi^{-1}\left( \bigcup_{\lambda\in\Lambda}E_\lambda \right) = \varnothing &\Longleftrightarrow \left(\bigcup_{\lambda\in\Lambda}E_\lambda\right)\bigcap \mathrm{Im}\varphi = \varnothing \nonumber\\
                    &\Longleftrightarrow \forall\lambda\in\Lambda\left( E_{\lambda}\cap \mathrm{Im}\varphi  \right)= \varnothing\nonumber\\
                    &\Longleftrightarrow \forall\lambda\in\Lambda\left( \varphi^{-1}(E_{\lambda}) = \varnothing \right)\nonumber\\
                    &\Longleftrightarrow \bigcup_{\lambda\in\Lambda}\varphi^{-1}(E_{\lambda}) = \varnothing.
                \end{align}
                再考虑另一个公式.
                \begin{align}
                    x\in \left(\varphi^{-1}(D)\right)^c &\Longleftrightarrow x\notin \varphi^{-1}(D) \Longleftrightarrow \varphi(x)\notin D\nonumber\\
                    &\Longleftrightarrow\varphi(x)\in D^c\Longleftrightarrow x\in \varphi^{-1}(D^c)
                \end{align}
            \end{proof}
            利用这两个公式, 就容易完成本题的证明.
            \begin{proof}
                \begin{enumerate}
                    \item \begin{equation}
                        \overline{\{\omega:\xi(\omega)\in B  \}} = \left(\xi^{-1}(B)\right)^c = \xi^{-1}(B^c) = \{ \omega:\xi(\omega)\in\overline{B} \};
                    \end{equation}
                    \item \begin{equation}
                        \bigcup_{k=1}^{+\infty}\{\omega:\xi(\omega)\in  B_k\} 
                        =\bigcup_{k=1}^{+\infty}\xi^{-1}(B_k) = \xi^{-1}\left(\bigcup_{k=1}^{+\infty}B_k\right)
                        = \left\{ \omega:\xi(\omega)\in\bigcup_{k=1}^{+\infty}B_k \right\}.
                    \end{equation}
                \end{enumerate}
            \end{proof}
	\end{homeworkProblem}
	
\end{document}
