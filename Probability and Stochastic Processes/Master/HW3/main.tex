%!TEX program = xelatex
\documentclass[a4paper,AutoFakeBold,oneside,10pt]{article}

%need xelatex
\usepackage[fntef]{ctex} % invole CJKfntef
\usepackage{ctex}

%%%%%%%% 这两个宏包冲突, 只可保留其中一个
%\usepackage{amsfonts}
\usepackage{mathrsfs}
%%%%%%%%


\usepackage{fancyhdr}
\usepackage{extramarks}
\usepackage{amsmath}
\usepackage{amsthm}

\usepackage{tikz}
\usepackage[plain]{algorithm}
\usepackage{algpseudocode}
\usepackage{amsthm}
\usepackage{mathtools}
\usepackage{bm}
\usepackage{physics}
\usepackage{calligra}
\usepackage{csquotes}
\usepackage{tensor}
\usepackage[thicklines]{cancel}
\usepackage{tcolorbox}
\usepackage{pstricks}
\usepackage{amssymb}
\usepackage{listings}
\usepackage{enumerate}

\usetikzlibrary{automata,positioning}

%
% Basic Document Settings
%

\topmargin=-0.45in
\evensidemargin=0in
\oddsidemargin=0in
\textwidth=6.5in
\textheight=9.0in
\headsep=0.25in

\linespread{1.1}

\pagestyle{fancy}
\lhead{\hmwkAuthorName}
\chead{\hmwkClass\ \hmwkClassInstructor\ \hmwkClassTime: \hmwkTitle}
\rhead{\firstxmark}
\lfoot{\lastxmark}
\cfoot{\thepage}

\renewcommand\headrulewidth{0.4pt}
\renewcommand\footrulewidth{0.4pt}

\setlength\parindent{0pt}

%
% Create Problem Sections
%

\newcommand{\enterProblemHeader}[1]{
	\nobreak\extramarks{}{习题 \arabic{#1} 接下页\ldots}\nobreak{}
	\nobreak\extramarks{习题 \arabic{#1} (continued)}{习题 \arabic{#1} 接下页\ldots}\nobreak{}
}

\newcommand{\exitProblemHeader}[1]{
	\nobreak\extramarks{习题 \arabic{#1} (continued)}{习题 \arabic{#1} 接下页\ldots}\nobreak{}
	\stepcounter{#1}
	\nobreak\extramarks{习题 \arabic{#1}}{}\nobreak{}
}

\setcounter{secnumdepth}{0}
\newcounter{partCounter}
\newcounter{homeworkProblemCounter}
\setcounter{homeworkProblemCounter}{1}
\nobreak\extramarks{Problem \arabic{homeworkProblemCounter}}{}\nobreak{}

\newenvironment{homeworkProblem}{
	\section{习题 \arabic{homeworkProblemCounter}}
	\setcounter{partCounter}{1}
	\enterProblemHeader{homeworkProblemCounter}
}{
	\exitProblemHeader{homeworkProblemCounter}
}

%
% Homework Details
%   - Title
%   - Due date
%   - Class
%   - Section/Time
%   - Instructor
%   - Author
%

\newcommand{\hmwkTitle}{作业\ \#3}
\newcommand{\hmwkDueDate}{9月22日, 2019}
\newcommand{\hmwkClass}{概率论与随机过程}
\newcommand{\hmwkClassTime}{}
\newcommand{\hmwkClassInstructor}{}
\newcommand{\hmwkAuthorName}{杨勇,2019110294}

%
% Title Page
%

\title{
	\vspace{2in}
	\textmd{\textbf{\hmwkClass:\ \hmwkTitle}}\\
	\normalsize\vspace{0.1in}\small{完成于 \hmwkDueDate}\\
	\vspace{0.1in}\large{\textit{\hmwkClassInstructor\ \hmwkClassTime}}
	\vspace{3in}
}

\author{\textbf{\hmwkAuthorName}}
\date{}

\renewcommand{\part}[1]{\textbf{\large Part \Alph{partCounter}}\stepcounter{partCounter}\\}

%
% Various Helper Commands
%

% Useful for algorithms
\newcommand{\alg}[1]{\textsc{\bfseries \footnotesize #1}}

% For derivatives
\newcommand{\deriv}[1]{\frac{\mathrm{d}}{\mathrm{d}x} (#1)}

% For partial derivatives
\newcommand{\pderiv}[2]{\frac{\partial}{\partial #1} (#2)}

% Integral dx
\newcommand{\dx}{\mathrm{d}x}

% Alias for the Solution section header
\newcommand{\solution}{\textbf{\large 解.}}

% Probability commands: Expectation, Variance, Covariance, Bias
\newcommand{\E}{\mathrm{E}}
\newcommand{\Var}{\mathrm{Var}}
\newcommand{\Cov}{\mathrm{Cov}}
\newcommand{\Bias}{\mathrm{Bias}}

\begin{document}
	
	\maketitle
	
	
	
	%%%%%%%%%%%%%%%%%%%%%%%%%%%%%%%%%
	\pagebreak
	\begin{homeworkProblem}
        设$\xi,\eta$是两个相互独立的随机变量, 且$\mathrm{var}(\xi)<\infty$, $\mathrm{var}(\eta)<\infty$, 试证:	
        \begin{equation}
            \mathrm{var}(\xi\eta) = \mathrm{var}(\xi)\mathrm{var}(\eta)+\left[ \mathrm{E}(\xi) \right]^2\mathrm{var}(\eta)+\left[ \mathrm{E}(\eta) \right]^2\mathrm{var}(\xi).
        \end{equation}
        
        \begin{proof}
            反复利用结论:$\mathrm{var}(X) = \mathrm{E}X^2-(\mathrm{E}X)^2$.
            
            则等式左边可以化为:
            \begin{align}
                \mathrm{var}(\xi\eta) &= \mathrm{E}\left[ (\xi\eta)^2 \right] - \left[\mathrm{E}(\xi\eta)\right]^2\nonumber\\
                &=\mathrm{E}(\xi^2)\mathrm{E}(\eta^2) - \left[\mathrm{E}(\xi)\right]^2\left[\mathrm{E}(\eta)\right]^2.
            \end{align}
            而等式右边为\begin{align}
                 &\mathrm{var}(\xi)\mathrm{var}(\eta)+\left[ \mathrm{E}(\xi) \right]^2\mathrm{var}(\eta)+\left[ \mathrm{E}(\eta) \right]^2\mathrm{var}(\xi)\nonumber\\
                 &=\mathrm{E}\left(\xi^2 \right)\mathrm{var}(\eta)+\left[ \mathrm{E}(\eta) \right]^2\mathrm{var}(\xi)\nonumber\\
                 &=\mathrm{E}\left(\xi^2 \right)\left\{ \mathrm{E}\left(\eta^2 \right) - \left[\mathrm{E}(\eta)\right]^2 \right\} + \left[ \mathrm{E}(\eta) \right]^2\left\{ \mathrm{E}\left( \xi^2 \right) - \left[\mathrm{E}(\xi)\right]^2 \right\}\nonumber\\
                 &=\mathrm{E}(\xi^2)\mathrm{E}(\eta^2) - \left[\mathrm{E}(\xi)\right]^2\left[\mathrm{E}(\eta)\right]^2.
            \end{align}
            因而二者相等.
        \end{proof}
	\end{homeworkProblem}
	
    \begin{homeworkProblem}
        设随机变量$\xi$的分布函数为$F(x)$, 若$\mathrm{E}(\xi)$有限, 则
        \begin{enumerate}
            \item \begin{equation}
                \lim_{x\to-\infty}xF(x) = 0,~~\lim_{x\to\infty}x\left[ 1-F(x) \right] = 0;
            \end{equation}
            \item \begin{equation}
                \mathrm{E}(\xi) = \int_{0}^{+\infty}\left[1-F(x)-F(-x)\right]\mathrm{d}x.
            \end{equation}
        \end{enumerate}
        \begin{proof}
        	由于$\mathrm{E}(\xi^+) = \int_{0}^{+\infty}x\mathrm{d}F(x)<+\infty$, 则$\lim\limits_{A\to\infty}\int_{A}^{+\infty}x\mathrm{d}F(x) = 0$. 故
            \begin{equation}
            	0\leqslant A(1-F(A))=A\int_{A}^{+\infty}\mathrm{d}F(x) \leqslant \int_{A}^{+\infty}x\mathrm{d}F(x)\to 0~~\text{as}~ A\to\infty.
            \end{equation}	类似地,根据$\mathrm{E}(\xi^-) = \int_{-\infty}^0x\mathrm{d}F(x)<+\infty$,则$\lim\limits_{A\to-\infty}\int_{-\infty}^{A}x\mathrm{d}F(x) = 0$. 因此,
            \begin{equation}
            	0\leqslant AF(A) = A\int_{-\infty}^{A}\mathrm{d}F(x)\leqslant \int_{-\infty}^Ax\mathrm{d}F(x)\to 0~~\text{as}~A\to-\infty.
            \end{equation}利用分部积分公式,\begin{align}
            	\int_0^{+\infty}\left[1-F(x)-F(-x)\right]\mathrm{d}x &= \lim_{A\to+\infty}\int_0^{A}\left[1-F(x)-F(-x)\right]\mathrm{d}x \nonumber\\
                &=\lim_{A\to+\infty}\left( x(1-F(x))\big\vert_0^A + xF(x)\big\vert_{-A}^0 +\int_{-A}^Ax\mathrm{d}F(x)\right)\nonumber\\
                &\geqslant \liminf_{A\to+\infty}\int_{-A}^Ax\mathrm{d}F(x).
            \end{align}
            由此知道\begin{align}
                \int_0^{+\infty}\left[1-F(x)-F(-x)\right]\mathrm{d}x &= \lim_{A\to+\infty}x(1-F(x))\big\vert_0^A+\lim_{A\to+\infty}xF(x)\big\vert_{-A}^0+\lim_{A\to+\infty}\int_{-A}^Ax\mathrm{d}F(x)\nonumber\\
                &=\int_{-\infty}^{\infty}x\mathrm{d}F(x) = \mathrm{E}(\xi).
            \end{align}
        \end{proof}
    \end{homeworkProblem}
    
    \begin{homeworkProblem}
        设$\{\xi_n,n\geqslant 1\}$是均值有限的同分布的随机变量列, 证明:
        \begin{equation}
            \lim_{n\to\infty}\frac{1}{n}\mathrm{E}\left[ \max_{1\leqslant j\leqslant n}\abs{\xi_j} \right] = 0.
        \end{equation}
    \end{homeworkProblem}
    
    \begin{homeworkProblem}
        设$r>1$, $\xi$为非负随机变量, 证明:
        \begin{equation}
            \int_{0}^{+\infty}\frac{1}{u^r}\mathrm{E}(\xi \wedge u^r)\mathrm{d}u = \frac{r}{r-1}\mathrm{E}(\xi^{1/r}).
        \end{equation}
        其中, $\xi\wedge u^r = \min(\xi,u^r)$.
    \end{homeworkProblem}
    
    \begin{homeworkProblem}
        设$\xi$, $\eta$是取非负整数的独立随机变量, $\mathrm{E}(\xi)<\infty$, $\mathrm{E}(\eta)<\infty$, 试证:
        \begin{enumerate}
            \item \begin{equation}
                \mathrm{E}(\xi) = \sum_{m=1}^{+\infty}P(\xi\geqslant m);
            \end{equation}
            \item \begin{equation}
                \mathrm{E}(\xi) = 2\sum_{m=1}^{+\infty}mP(\xi\geqslant m)-\mathrm{E}(\xi)\left[\mathrm{E}(\xi)+1\right];
            \end{equation}
            \item \begin{equation}
                \mathrm{E}\left[ \min(\xi,\eta) \right] = \sum_{m=1}^{\infty}P(\xi\geqslant m)P(\eta\geqslant m).
            \end{equation}
        \end{enumerate}
    \end{homeworkProblem}
    
    \begin{homeworkProblem}
        假设$\xi$是非负随机变量, 只取有限个值$x_1,\cdots,x_m$, 试证明:\begin{equation}
            \lim_{n\to\infty}\frac{\mathrm{E}(\xi^{n+1})}{\mathrm{E}(\xi^n)} = \lim_{n\to\infty}\sqrt{\mathrm{E}(\xi^n)} = \max\{x_1,x_2,\cdots,x_m\}.
        \end{equation}
    \end{homeworkProblem}
    
    \begin{homeworkProblem}
        设$\xi_1,\cdots,\xi_n$是任意$n$个独立同分布的正值随机变量, 试证明: 对任意
        $1\leqslant k<n$, 有\begin{equation}
            \mathrm{E}\left( \frac{\xi_1+\cdots+\xi_k}{\xi_1+\cdots+\xi_n} \right) = \frac{k}{n}.
        \end{equation}
    \end{homeworkProblem}
    
    \begin{homeworkProblem}
        假设随机变量$\xi_1,\cdots,\xi_n$, $\eta_1,\cdots,\eta_n$相互独立, 其中$\mathrm{E}\xi_i=\mu$, $\mathrm{var}(\xi_i)=\sigma^2$($i=1,\cdots,n$), 而$\eta_j$($j=1,2,\cdots,n$)服从参数为$p$的$0-1$分布, 令
        \begin{equation}
            \zeta_n = \xi_1+\cdots+\xi_n,~~\zeta_n^* = \xi_1\eta_1+\cdots+\xi_n\eta_n,
        \end{equation}
        求$\mathrm{E}(\zeta_n),\mathrm{var}(\zeta_n),\mathrm{E}(\zeta_n^*),\mathrm{var}(\zeta_n^*)$.
    \end{homeworkProblem}
    
    \begin{homeworkProblem}
        若$\xi_1,\xi_2$相互独立均服从$\mathcal{N}(a,\sigma^2)$, 试证:
        \begin{equation}
            \mathrm{E}\left[ \max(\xi_1,\xi_2) \right] = a+\frac{\sigma}{\sqrt{\mathrm{\pi}}}.
        \end{equation}
        \begin{proof}
            Note that:
            \begin{equation}
                \abs{\xi_1-\xi_2} = \max(\xi_1,\xi_2)-\min(\xi_1,\xi_2) = \max(\xi_1-a,\xi_2-a)+\max(a-\xi_1,a-\xi_2)
            \end{equation}
            Since the joint distribution of $(\xi_1-a,\xi_2-a)$ is symmetric about $0$, the distribution of $\max(\xi_1-a,\xi_2-a)$ and $\max(a-\xi_1,a-\xi_2)$ are the same. Hence, $\mathrm{E}\abs{\xi_1-\xi_2} = 2\mathrm{E}(\max(\xi_1-a,\xi_2-a)$. From the property of the normal distribution, $\xi_1-\xi_2$ is normally distributed with mean $0$ and variance $\mathrm{var}(\xi_1-\xi_2) = \mathrm{var}(\xi_1)+\mathrm{var}(\xi_2) = 2\sigma^2$. Then,
            \begin{equation}
                \mathrm{E}\left[ \max(\xi_1,\xi_2) \right] = a+\mathrm{E}\abs{\xi_1-\xi_2} = a+ 2^{-1}\sqrt{2/\mathrm{\pi}}\sqrt{2\sigma^2} = a+\frac{\sigma}{\sqrt{\mathrm{\pi}}}.
            \end{equation}
        \end{proof}
    \end{homeworkProblem}
    
    \begin{homeworkProblem}
        设$f(x)(x>0)$是一正值上升函数, $\xi$是一随机变量, 若$\mathrm{E}\left[ f(\abs{\xi}) \right] < +\infty$, 则有
        \begin{equation}
            P(\abs{\xi}\geqslant \varepsilon)\leqslant\frac{\mathrm{E}\left[ f(\abs{\xi}) \right]}{f(\varepsilon)}~~(\varepsilon>0).
        \end{equation}
    \end{homeworkProblem}
    
    \begin{homeworkProblem}
        试证明:如果$\mathrm{E}(\xi)$有限, 则
        \begin{align}
            \mathrm{E}(\xi) &= -\int_{0}^{+\infty}F_{\xi}(x)\mathrm{d}x+\int_{0}^{+\infty}\left[ 1-F_{\xi}(x) \right]\mathrm{d}x;\nonumber\\
            \mathrm{E}(\abs{\xi}) &= \int_{0}^{+\infty}F_{\xi}(x)\mathrm{d}x+\int_{0}^{+\infty}\left[ 1-F_{\xi}(x) \right]\mathrm{d}x.
        \end{align}	\begin{proof}
        By Fubini's theorem,
        \begin{align}
        	\int_{0}^{\infty}[1-F(x)]\mathrm{d}x&=\int_{0}^{\infty}\int_{(x,\infty)}d\mathrm{d}F(y)\mathrm{d}x\nonumber\\
            &=\int_{0}^{\infty}\int_{(0,y)}\mathrm{d}x\mathrm{d}F(y)\nonumber\\
            &=\int_{0}^{\infty}y\mathrm{d}y.
        \end{align}
        Similarly,\begin{equation}
        \int_{-\infty}^0F(x)\mathrm{d}x=\int_{-\infty}^0\int_{(-\infty,x]}\mathrm{d}F(y)\mathrm{d}x = -\int_{-\infty}^0y\mathrm{d}F(y).
        \end{equation}
        If $\mathrm{E}\xi$ exists, then at least one of $\int_{0}^{\infty}y\mathrm{d}F(y)$ and $\int_{-\infty}^0y\mathrm{d}F(y)$ is finite and
        \begin{equation}
        \mathrm{E}\xi = \int_{-\infty}^{\infty}y\mathrm{d}F(y) =  -\int_{0}^{+\infty}F_{\xi}(x)\mathrm{d}x+\int_{0}^{+\infty}\left[ 1-F_{\xi}(x) \right]\mathrm{d}x;
        \end{equation}
        \begin{equation}
        	\mathrm{E}\abs{\xi} = \int_{-\infty}^\infty \abs{y}\mathrm{d}F(y) = \int_{0}^{+\infty}F_{\xi}(x)\mathrm{d}x+\int_{0}^{+\infty}\left[ 1-F_{\xi}(x) \right]\mathrm{d}x.
        \end{equation}
        \end{proof}
    \end{homeworkProblem}
    
    \begin{homeworkProblem}
        设$\xi_1,\cdots,\xi_n$是独立的随机变量, $\mathrm{var}(\xi_j) = \sigma_j^2$($j=1,\cdots,n$), 试求满足$\sum\limits_{j=1}^na_j = 1$的正数$a_1,\cdots,a_n$,使$\sum\limits_{j=1}^na_j\xi_j$的方差最小.
    \end{homeworkProblem}
    
    \begin{homeworkProblem}
        设随机向量$\bm{\xi} = (\xi,\eta)$有密度
        \begin{equation}
            f(x,y) = \begin{dcases}
                \frac{p(x+y)}{x+y},&x>0,y>0,\\
                0,&\text{otherwise},
            \end{dcases}
        \end{equation}
        其中$p(u),u\in(-\infty,+\infty)$是一维分布密度, $p(u) = 0$($u\in(-\infty,0)$), 求$\bm{\xi}$的协方差矩阵.
    \end{homeworkProblem}
    
    \begin{homeworkProblem}
        设$\xi$和$\eta_j$($j=1,\cdots,n$)不相关, 证明: 对于任意$a_j(j=1,\cdots,n)$, $\xi$与$\sum\limits_{j=1}^na_j\eta_j$不相关.
    \end{homeworkProblem}
    
    \begin{homeworkProblem}
        设$\xi_1,\xi_2,\cdots,x_n$独立同标准正态分布, 证明:$\zeta_1=\sum\limits_{j=1}^n\alpha_j\xi_j$和$\zeta_2=\sum\limits_{j=1}^n\beta_j\xi_j$独立的充要条件是$\sum\limits_{j=1}^n\alpha_j\beta_j = 0$.
    \end{homeworkProblem}
    
    \begin{homeworkProblem}
        设随机变量$\xi_1,\cdots,\xi_n$相互独立, 并具有相同的数学期望$\mu$和方差$\sigma^2$, 令\begin{equation}
            \bar{\xi} = \frac{1}{n}\sum_{j=1}^n\xi_j,~~S^2 = \frac{1}{n-1}\sum_{j=1}^n(\xi_j-\bar{\xi})^2,
        \end{equation}
        试证:\begin{enumerate}
            \item \begin{equation}
                \mathrm{E}(\bar{\xi}) = \mu,\mathrm{var}(\bar{\xi}) = \frac{1}{n}\sigma^2;
            \end{equation}
            \item \begin{equation}
                \mathrm{Cov}(\bar{\xi},\xi_j-\bar{\xi})=\mathrm{E}\left[ (\bar{\xi}-\mu)(\xi_j-\bar{\xi}) \right] = 0,~j=1,\cdots,n;
            \end{equation}
            \item \begin{equation}
                \mathrm{E}(S^2) = \sigma^2.
            \end{equation}
        \end{enumerate}
    \end{homeworkProblem}
    
    \begin{homeworkProblem}
        求$\mathrm{E}(\xi|\eta=y),y>0$. 设$\xi$和$\eta$的联合密度为
        \begin{enumerate}
            \item \begin{equation}
                f(x,y) = \begin{dcases}
                    \frac{1}{y}\mathrm{e}^{-y-\frac{x}{y}}, &x>0,y>0;\\
                    0,&\text{otherwise};
                \end{dcases}
            \end{equation}
            \item \begin{equation}
                f(x,y) = \begin{dcases}
                    \lambda^2\mathrm{e}^{-\lambda x}, &0<y<x;\\
                    0,&\text{otherwise};
                \end{dcases}
            \end{equation}
        \end{enumerate}
        
    \end{homeworkProblem}
    
    \begin{homeworkProblem}
        设$\xi$和$\eta$独立同分布, 求$\mathrm{var}(\xi|\xi+\eta)$, 已知:
        \begin{enumerate}
            \item $\xi$在$[0,1]$上有均匀分布;
            \item $\xi$服从参数为$\lambda$的指数分布.
        \end{enumerate}
    \end{homeworkProblem}
    
    
    
\end{document}
