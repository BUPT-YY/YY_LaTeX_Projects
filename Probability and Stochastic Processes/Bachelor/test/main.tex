\documentclass[aspectratio=43]{beamer}
\usepackage[english]{babel}
\usepackage[fntef]{ctex} % invole CJKfntef
\usepackage{ctex}
\usepackage{fontspec}
\setmainfont{CMU Serif}

\usepackage{mathrsfs}
\usepackage{amssymb,amsmath, amsthm,bm}
\usepackage{mathtools}
\input{chapters/preamble}
\title{Solution} %->->->->-> Check hyperref title <-<-<-<-<-
\author[Yong YANG]{Yong YANG, 2019110294}
\institute[BUPT]{
	Beijing University of Posts and Telecommunications%
} %You can change the Institution if you are from somewhere else
\date{\today}
%\logo{\includegraphics[width= 0.2\textwidth]{images/a-logo.png}}

\usepackage{graphicx}
\definecolor{hellmagenta}{rgb}{1,0.75,0.9}
\definecolor{hellcyan}{rgb}{0.75,1,0.9}
\definecolor{hellgelb}{rgb}{1,1,0.8}
\definecolor{colKeys}{rgb}{0,0,1}
\definecolor{colIdentifier}{rgb}{0,0,0}
\definecolor{colComments}{rgb}{1,0,0}
\definecolor{colString}{rgb}{0,0.5,0}
\definecolor{darkyellow}{rgb}{1,0.9,0}

\begin{document}
	
	\frame{\titlepage}
	
   \section{}
\begin{frame}
	求$x\to 1^-$时, 与$\sum\limits_{n=0}^{+\infty}x^{n^2}$等价的无穷大量.
\begin{block}{解:}
	当$n\leqslant t<n+1$时,
	\begin{equation*}
		\mathrm{e}^{\textcolor{red}{n^2}\ln x}\geqslant \mathrm{e}^{\textcolor{red}{t^2}\ln x}>\mathrm{e}^{\textcolor{red}{(n+1)^2}\ln x},
	\end{equation*}
	积分得到\begin{equation*}
		\mathrm{e}^{\textcolor{red}{n^2}\ln x}\geqslant \int_{n}^{n+1}\mathrm{e}^{\textcolor{red}{t^2}\ln x}\mathrm{d}t>\mathrm{e}^{\textcolor{red}{(n+1)^2}\ln x}.
	\end{equation*}
	求和, 得\begin{equation*}
		\sum_{n=0}^{+\infty}\mathrm{e}^{\textcolor{red}{n^2}\ln x}\geqslant \int_{0}^{+\infty}\mathrm{e}^{\textcolor{red}{t^2}\ln x}\mathrm{d}t>\sum_{n=1}^{+\infty}\mathrm{e}^{\textcolor{red}{n^2}\ln x} = \sum_{n=0}^{+\infty}\mathrm{e}^{\textcolor{red}{n^2}\ln x} - 1.
	\end{equation*}
\end{block}

\end{frame}

\begin{frame}
	\begin{block}{接上页...}
		因此, 与其等价的无穷大量是\begin{equation*}
			\begin{split}
				\sum_{n=0}^{+\infty}\mathrm{e}^{\textcolor{red}{n^2}\ln x} &\sim \int_{0}^{+\infty}\mathrm{e}^{t^2\ln x}\mathrm{d}t \\
				&= \frac{1}{\sqrt{-\ln x}}\frac{1}{2}\sqrt{\pi}\\
				&\sim \frac{1}{2}\sqrt{\frac{\pi}{1-x}}.
			\end{split}
		\end{equation*}
	\end{block}
\end{frame}

\section{}
\begin{frame}{}
\centering
\Huge\bfseries
\textcolor{orange}{谢~~谢}
\end{frame}
\end{document}
