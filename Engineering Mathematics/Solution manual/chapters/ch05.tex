

\chapter{数学物理方程及其定解条件}

\section{习题1-5}

\begin{yyEx}
	一均匀杆的原长为$l$, 一端固定, 另一端沿杆的轴线方向拉长$e$而静止, 突然放手任其振动, 试建立振动方程与定解条件.
\end{yyEx}

\begin{yyEx}
	长为$l$的弦两端固定, 开始时在$x = c$受冲量$k$的作用, 试写出相应的定解问题.
\end{yyEx}

\begin{yyEx}
	长为$l$的均匀杆, 侧面绝缘. 一端温度为零, 另一端有恒热流$q$进入(即单位时间内通过单位截面流入的热量为$q$), 杆的初始温度分布为$\begin{lgathered}\frac{x(l-x)}{2}\end{lgathered}$, 试写出相应的定解问题.
\end{yyEx}

\begin{yyEx}
	半径为$R$而表面熏黑的金属长圆柱体, 受到阳光照射, 阳光方向垂直于柱轴(图5.7), 热流强度为$M$, 写出这个圆柱的热传导问题的边界条件.
\end{yyEx}

\begin{yyEx}
	若$F(z),G(z)$为两个任意二次连续可微函数, 验证\begin{equation*}
		u = F(x+at)+G(x-at)
	\end{equation*}
	满足方程$\begin{lgathered}
		\frac{\partial^2 u}{\partial t^2} = a^2\frac{\partial^2 u}{\partial x^2}.
	\end{lgathered}$
\end{yyEx}

\section{习题6-7}

\begin{yyEx}
	验证线性齐次方程的叠加原理, 即若$u_1(x,y),u_2(x,y),\cdots,u_n(x,y),\cdots$均是线性二阶齐次方程
	\begin{equation*}
		A\frac{\partial^2 u}{\partial x^2} + 2B\frac{\partial^2 u}{\partial x\partial y} + C\frac{\partial^2 u}{\partial y^2} + D\frac{\partial u}{\partial x} + E\frac{\partial u}{\partial y} + Fu = 0
	\end{equation*}
	的解, 其中$A,B,C,D,E,F$都只是$x,y$的函数, 而且级数$u = \sum\limits_{i = 1}^{+\infty}c_iu_i(x,y)$收敛, 其中$c_i(i=1,2,\cdots)$为任意常数, 并且对$x,y$可以逐次微分两次, 求证$u = \sum\limits_{i = 1}^{+\infty}c_iu_i(x,y)$仍是原方程的解.
\end{yyEx}

\begin{yyEx}
	把下列方程化为标准型
	\begin{enumerate}
		\item $u_{xx}+4u_{xy} + 5u_{yy} + u_x+2u_y = 0$;
		\item $u_{xx}+yu_{yy} = 0$;
		\item $u_{xx}+xu_{yy} = 0$;
		\item $y^2u_{xx}+x^2u_{yy} = 0$.
	\end{enumerate}
\end{yyEx}
