

\chapter{矢量分析与场论初步}

\section{习题1-5}

\begin{yyEx}
	求向量函数$\begin{lgathered}
	\bm{F}(M) = \left( \frac{xz}{\sqrt{xz+1}-1}, \mathrm{e}^{x^2z+y^2},\frac{\sin(xy)}{y}\right)
	\end{lgathered}$的极限$\begin{lgathered}
	\lim\limits_{M\to(0,1,-1)}\bm{F}(M).
	\end{lgathered}$
\end{yyEx}

\begin{yyEx}
	求向量函数$\begin{lgathered}
	\bm{F}(M) = \left( \mathrm{e}^{x^2+y^2},yz+x^2,\frac{y-1}{1+xz} \right)
	\end{lgathered}$的极限$\begin{lgathered}
		\lim\limits_{M\to(1,0,1)}\bm{F}(M).
	\end{lgathered}$
\end{yyEx}

\begin{yyEx}
	讨论下列向量函数在指定点处的连续性:
	\begin{enumerate}
		\item $\begin{lgathered}
		\bm{F}(M) = \left( \frac{x^3+y^3}{x^2+y^2},x+y+z,x^2+z^2 \right)
		\end{lgathered}$在点$M_0(0,0,2)$;
		\item $\begin{lgathered}
		\bm{F}(M) = \left( \frac{\sin(xy)}{x},x+3y,3z+xy \right)
		\end{lgathered}$在点$M_0(1,1,-1)$.
	\end{enumerate}
\end{yyEx}

\begin{yyEx}
	求矢量函数$\bm{A}(x,y,z) = x\sin(x+y)\bm{i}+x^4y^2\bm{j}+(x+\mathrm{e}^{yz})\bm{k}$的偏导数 $\begin{lgathered}
		\frac{\partial \bm{A}}{\partial x}
	\end{lgathered}$,$\begin{lgathered}
	\frac{\partial \bm{A}}{\partial y}
	\end{lgathered}$和$\begin{lgathered}
	\frac{\partial \bm{A}}{\partial z}
	\end{lgathered}$.
\end{yyEx}

\begin{yyEx}
	\begin{enumerate}
		\item 已知$\begin{lgathered}
		\bm{A}(t) = (1+3t^2)\bm{i} - 2t^3\bm{j}+\frac{t}{2}\bm{k}
		\end{lgathered}$, 求$\begin{lgathered}
			\int_{0}^{2}\bm{A}(t)\mathrm{d}t;
		\end{lgathered}$
		\item 计算$\begin{lgathered}
			\int \bm{F}(M)\mathrm{d}x
		\end{lgathered}$, 其中$\begin{lgathered}
			\bm{F}(M) = \left( 1+3x^2,x^2+\frac{\mathrm{e}^x}{x},\ln x \right)
		\end{lgathered}$
	\end{enumerate}
\end{yyEx}

\section{习题6-10}

\begin{yyEx}
	计算下列各题:
	\begin{enumerate}
		\item 设数量场$u = \ln\sqrt{x^2+y^2+z^2}$, 求$\mathrm{div}(\mathrm{grad}u)$.
		\item 设$\bm{r}=x\bm{i}+y\bm{j}+z\bm{k}$, $r = \sqrt{x^2+y^2+z^2}$是$\bm{r}$的模, $\bm{c}$是常向量, 求$\mathrm{rot}[f(r)\bm{c}]$.
		\item 求向量场$\bm{F} = xy^2\bm{i}+y\mathrm{e}^z\bm{j}+x\ln(1+z^2)\bm{k}$ 在点$P(1,1,0)$处的散度$\mathrm{div}\bm{F}$.
	\end{enumerate}
\end{yyEx}
	
\begin{yyEx}
	设数量场$\begin{lgathered}
	u = \frac{a}{r}
	\end{lgathered}$, 其中$r = \sqrt{x^2+y^2+z^2}$, $a$为常数. 求:
	\begin{enumerate}
		\item $u$在$P(x_0,y_0,z_0)$处的梯度$\begin{lgathered}\mathrm{grad}u\big\vert_{P}\end{lgathered}$;
		\item $u$在$P$处沿$x_0\bm{i}+y_0\bm{j}+z_0\bm{k}$方向的方向导数.
	\end{enumerate}
\end{yyEx}
	
\begin{yyEx}
	一质点在力场$\bm{F} = (y-z)\bm{i}+(z-x)\bm{j}+(x-y)\bm{k}$的作用下, 沿螺旋线$x = a\cos t,y = a\sin t, z = bt$ 运动, 求其从$t = 0$到$t = 2\pi$时所做的功.
\end{yyEx}

\begin{yyEx}
	设曲面$S$由平面$x = 0,y = 0,z = 0$和$x+y+z = 1$所构成, 求向量场$\bm{F} = x\bm{i}+y\bm{j}+z\bm{k}$ 从内穿出闭曲面$S$的通量$\varPhi$.
\end{yyEx}

\begin{yyEx}
	已知数量场$\begin{lgathered}
		u =\ln\frac{1}{r}
	\end{lgathered}$, 其中 
	
	$r = \sqrt{(x-a)^2 + (y-b)^2 + (z-c)^2}$. 在空间$Oxyz$的哪些点上$\abs{\mathrm{grad}u} = 1$成立.
\end{yyEx}

\section{习题11-15}

\begin{yyEx}
	\begin{enumerate}
		\item 证明:$\begin{lgathered}
			\grad\frac{1}{r} = -\frac{\bm{r}}{r^3},\grad\frac{1}{r^3} = -\frac{3\bm{r}}{r^5}
		\end{lgathered}$, 其中$r = x\bm{i}+y\bm{j}+z\bm{k}$.
		\item 若$u = u(v,w), v = v(x,y,z), w = w(x,y,z)$,
		
		 证明:$\begin{lgathered}
			\grad u = \frac{\partial u}{\partial v}\grad v+\frac{\partial u}{\partial w}\grad w.
		\end{lgathered}$
		\item 求标量场$u = x^2+2y^2+3z^2+xy+3x-2y-6z$在点$(1,-2,1)$处的梯度大小和方向.
		\item 证明$\grad u$为常矢量的充要条件是$u$为线性函数$u = ax+by+cz+d$.
	\end{enumerate}
\end{yyEx}

\begin{yyEx}
	\begin{enumerate}
		\item 证明: $\begin{lgathered}
			\grad\bm{\cdot}\frac{\bm{r}}{r} = \frac{2}{r},\grad\bm{\cdot}(r\bm{k}) = \frac{\bm{r}}{r}\bm{\cdot}\bm{k}
		\end{lgathered}$($\bm{k}$为常矢量,$r = \sqrt{x^2+y^2+z^2}$).
		\item 若$\bm{A} = \bm{A}(u),u = u(x,y,z)$, 证明: $\begin{lgathered}
		\grad\bm{\cdot A} = \frac{\mathrm{d}\bm{A}}{\mathrm{d}u}\bm{\cdot}\grad u.
		\end{lgathered}$
		\item 可压缩流体的密度为非稳定场$\rho(x,y,z,t)$, 流体的质量守恒定律为$\begin{lgathered}
			\varoiint\limits_S\rho\bm{v\cdot}\mathrm{d}\bm{s} = -\frac{\mathrm{d}}{\mathrm{d}\bm{s}}\varoiint\limits_{\Omega}\rho\mathrm{d}\Omega
		\end{lgathered}$, 试由此推出流体力学的连续性方程$\begin{lgathered}
			\frac{\partial\rho}{\partial t}+\grad\bm{\cdot}(\rho\bm{v}) = 0.
		\end{lgathered}$
	\end{enumerate}
\end{yyEx}

\begin{yyEx}
	\begin{enumerate}
		\item 证明:$\begin{lgathered}
			\grad\times\frac{\bm{r}}{r^3} = \bm{0},\grad\times\left[ F(r)\bm{r} \right] = \bm{0}.
		\end{lgathered}$
		\item 若$\bm{k}$为常矢量, 证明$\begin{lgathered}
			\grad\times\frac{\bm{k}}{r} = \bm{k}\times\frac{\bm{r}}{r^3}.
		\end{lgathered}$
		\item 若$\bm{k}$为常矢量, 证明$\begin{lgathered}
		\grad\times\left[ F(r) \bm{k} \right] = F'(r)\frac{\bm{r}}{r}\times\bm{k}.
		\end{lgathered}$
		\item 若$\bm{A} = \bm{A}(u),u = u(x,y,z)$, 证明: $\begin{lgathered}
			\grad\bm{\cdot A} = \grad u\times\frac{\mathrm{d}\bm{A}}{\mathrm{d}u}.
		\end{lgathered}$
		\item 证明: $(\bm{A}\bm{\cdot}\grad)\bm{r} = \bm{A}$.
	\end{enumerate}
\end{yyEx}

\begin{yyEx}
	设$\bm{k}$为常矢量, $\grad\times\bm{E} = \bm{0}$. 证明:
	\begin{enumerate}
		\item $\grad(\bm{k\cdot E}) = \bm{k\cdot}\grad\bm{E}$;
		\item $\grad(\bm{k\cdot r}) = \bm{k}$;
		\item $\begin{lgathered}
			\grad\left( \frac{\bm{k\cdot r}}{r^3} \right) = -\left[ \frac{3(\bm{k\cdot r})\bm{r}}{r^5} - \frac{\bm{k}}{r^3} \right]
		\end{lgathered}$;
		\item $\begin{lgathered}
		\grad\left( \frac{\bm{k}\times\bm{r}}{r^3} \right) =  \frac{3(\bm{k\cdot r})\bm{r}}{r^5} - \frac{\bm{k}}{r^3}
		\end{lgathered}$.
	\end{enumerate}
\end{yyEx}

\begin{yyEx}
	\begin{enumerate}
		\item 证明$\begin{lgathered}
			\bm{E}\times(\grad\times \bm{E}) = \frac{1}{2}\grad \bm{E}^2-(\bm{E}\bm{\cdot}\grad\bm{E});
		\end{lgathered}$
		\item 证明$\begin{lgathered}
			\iiint\limits_{v}\grad\times\bm{A}\mathrm{d}v = - \varoiint\limits_s\bm{A}\times\mathrm{d}\bm{s};
		\end{lgathered}$
		\item 由电磁感应定律的积分形式$\begin{lgathered}
			\oint_l\bm{E}\mathrm{d}\bm{l} = -\frac{d\iint\limits_{s}\bm{B}\bm{\cdot}\mathrm{d}\bm{s}}{\mathrm{d}l}
		\end{lgathered}$, 推出其微分形式$\begin{lgathered}
			\grad\times\bm{E} = -\frac{\partial\bm{B}}{\partial t};
		\end{lgathered}$
		
		\item 证明$\begin{lgathered}
			\oint_l(\bm{A}\times\bm{r})\bm{\cdot}\mathrm{d}\bm{l} = 2\iint\limits_{s}\bm{A}\mathrm{d}\bm{s}
		\end{lgathered}$~($\bm{A}$为常矢量).
	\end{enumerate}
\end{yyEx}

\section{习题16-20}

\begin{yyEx}
	证明:$\grad^2(uv) = u\grad^2 v+v\grad^2u+2\grad u\bm{\cdot}\grad v$.
\end{yyEx}

\begin{yyEx}
	已知$u(r,\theta,\varphi) = 2r\sin\varphi+r^2\cos\theta$, 求$\grad u,\Delta u$.
\end{yyEx}

\begin{yyEx}
	已知$\bm{A}(M) = r\cos^2\theta\bm{e}_{\rho} + r\sin\theta\bm{e}_{\theta} + z\cos\theta\bm{e}_{z}$, 求$\mathrm{div}\bm{A}$, $\mathrm{rot}\bm{A}$.
\end{yyEx}

\begin{yyEx}
	设$\bm{r} = x\bm{i}+y\bm{j}+z\bm{k}$, 在柱面坐标系和球面坐标系下, 证明$\nabla\bm{\cdot} \bm{r} = 3$.
\end{yyEx}

\begin{yyEx}
	证明下列向量场是调和场:
	\begin{enumerate}
		\item $\bm{F}(x,y,z) = (2x+y)\bm{i}+(4y+x+2z)\bm{j}+(2y-6z)\bm{k}$;
		\item $\bm{F}(x,y,z) = (1+2x-5y)\bm{i}+(4y-5x+7z)\bm{j}+(7y-6z)\bm{k}$;
		\item $\bm{F}(x,y,z) = yz\bm{i}+zx\bm{j}+xy\bm{k}$.
	\end{enumerate}
\end{yyEx}
