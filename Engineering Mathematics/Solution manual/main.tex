%!TEX program = xelatex
\documentclass[10pt,UTF8]{ctexbook}
\usepackage{ctex}
\usepackage{fontspec}
\setmainfont{CMU Serif}

%B5
%\usepackage[papersize={185mm,230mm},body={155mm,190mm},left=15mm,top=10mm,includeheadfoot, head=15pt]{geometry}


%A5
\usepackage[papersize={148mm,210mm},body={122.6mm,164.6mm},left=12.7mm,top=12.7mm,includeheadfoot, head=15pt]{geometry}



\setlength{\parindent}{2em}
\setlength{\parskip}{3pt plus1pt minus2pt}
\ctexset{%
	contentsname={目\ \ 次},
	listfigurename={插\ 图\ 目\ 录},
	listtablename={表\ 格\ 目\ 录},
	bibname={参\ 考\ 文\ 献},
	chapter={name={第,章},
		number=\chinese{chapter},
		format=\raggedright,
		nameformat={\color{blue}\huge\heiti},
		titleformat={\color{black}\huge\heiti},
		beforeskip={0pt}
	},
	section={format=\raggedright,
		nameformat={\large\bfseries},
		titleformat={\large\bfseries}
	},
	subsection={format=\raggedright,
		nameformat={\bfseries},
		titleformat={\bfseries}
	}
}
%%===== 中英文字体
%\setmainfont{Adobe Garamond Pro} %{Minion Pro} % 衬线字体 roman \rmfamily
%\setsansfont{Myriad Pro} % 无衬线字体 sans serif \sffamily
%\setmonofont{Consolas}   % 等宽字体 typewriter \ttfamily
%% 中文字体
%\setCJKmainfont[BoldFont={方正小标宋_GBK},%FZHeiGB18030,FZSongHeiGB18030}, %
%                ItalicFont={FZBeiWeiKaiShuS}]%FZKaiGB18030
% {FZShuSongGB18030}%{FZBoYaSong,NSimSun,Adobe Song Std L}
%\setCJKsansfont{FZHeiGB18030}
%\setCJKmonofont{FZKaiGB18030}
\newcommand{\song}{\songti}
\newcommand{\hei}{\heiti}
\newcommand{\kai}{\kaishu}
%\setCJKfamilyfont{BWKai}{FZBeiWeiKaiShuS}\newcommand{\BWKai}{\CJKfamily{BWKai}}
\newcommand{\BWKai}{\kai}

\newcommand*\norm[1]{\|#1\|}
\newcommand*\abs[1]{\lvert#1\rvert}
%===== 常用宏包
\usepackage{amsmath,amssymb,amsfonts,bm}
\usepackage{esint}
\usepackage{extarrows}
\usepackage{fancyhdr}  % 页眉和页脚
\usepackage{listings}  % 源代码
\usepackage{makeidx}\makeindex   % 索引
\usepackage{graphicx,xcolor}\graphicspath{{figures/}}
\usepackage{subfigure} % 子插图
%\usepackage{picins} % 图文混排, texlive 不支持
%\piccaption{This is caption\label{aaa}}
%\parpic[r]{\includegraphics[width=5cm]{aaa.jpg}}
%\usepackage{picinpar}
%\begin{figwindow}
%\usepackage{epstopdf} % 处理 eps 图片
\usepackage{tikz}
\usetikzlibrary{arrows,backgrounds,scopes,plotmarks}
\usepackage{mathtools}
\usepackage{bm}
\usepackage{tensor}

%===== 背景图片
\usepackage{wallpaper}
%\URCornerWallPaper{0.1}{ecnu03s.png} % 在右上角插入背景图片

%%===== 每章生成小目录
%\usepackage{minitoc}
%\setcounter{minitocdepth}{2} % 显示到 subsection
%%\nomtcrule  % 去掉 minitoc 的横线
%\dominitoc[n] % 使 minitoc 起作用, n 表示不显示 contents 标题

%===== 参考文献与链接
% \usepackage{chapterbib} % 分章参考文献
\usepackage[numbers,sort&compress,sectionbib]{natbib}
\setlength{\bibsep}{0.5ex}
%\usepackage[xetex,pagebackref]{hyperref}
\usepackage[xetex,pagebackref]{hyperref}
\hypersetup{CJKbookmarks=true,colorlinks=true,citecolor=blue,%
	linkcolor=blue,urlcolor=blue,bookmarksnumbered=true,%
	bookmarksopen=true,bookmarksopenlevel=1,breaklinks=true}
%\renewcommand*{\backrefalt}[4]{%
%    \ifcase #1 No citations.%
%    \or Cited on page #2.%
%    \else Cited on pages #2.%
%    %\else #1 Cited on pages #2.%
%    \fi
%    }

%===== 浮动图表的标题
\usepackage[margin=2em,labelsep=period,skip=0.5em,font=normalfont]{caption}
\DeclareCaptionFormat{mycaption}%
{{\heiti\color{blue} #1}#2{\kaishu #3}}
\captionsetup{format=mycaption,tablewithin=chapter,figurewithin=chapter}

%===== 浮动对象距离设置
\usepackage{float}
\setlength{\floatsep}{10pt plus 3pt minus 2pt}
% 出现在页面顶部或底部中的浮动对象之间的竖直距离
\setlength{\textfloatsep}{10pt plus 3pt minus 2pt}
% 出现在页面顶部或底部中的浮动对象与文本之间的竖直距离
\setlength{\intextsep}{10pt plus 3pt minus 2pt}
% 出现在页面中间的浮动对象上下文之间的竖直距离

%===== 表格
\usepackage{longtable}
\usepackage{colortbl}
\usepackage{booktabs}
\usepackage{multirow,bigdelim}  % 多行
\usepackage{fancyvrb}
\fvset{formatcom=\color{blue},frame=single,rulecolor=\color{red}}

%===== 给文字加边框和背景色
%\usepackage{framed}  % shaded 环境
\definecolor{shadecolor}{gray}{0.9}
\definecolor{notecolor}{gray}{0.95}
\usepackage[framemethod=tikz]{mdframed}  % shaded 环境
\newmdenv[linecolor=green,middlelinewidth=0.5pt,%
roundcorner=3pt,backgroundcolor=yellow!5,%
innertopmargin=0.5em,innerbottommargin=0.5em,%
innerleftmargin=3pt,innerrightmargin=3pt,%
skipbelow=0.5em,skipabove=1em,%
splittopskip=\topskip]{Block}
\newmdenv[linecolor=green,middlelinewidth=0pt,%
outerlinewidth=0.5pt,
roundcorner=3pt,backgroundcolor=shadecolor,%
innerbottommargin=3pt,innerrightmargin=5pt,%
innerleftmargin=5pt,%
skipabove=0pt]{mathbox}
\newmdenv[linecolor=blue!5!green,middlelinewidth=0.3pt,%
roundcorner=3pt,backgroundcolor=red!2!white,%
% frametitle={Hello},frametitlebackgroundcolor=green!50,%
skipabove=5pt,skipbelow=2pt,%
innerleftmargin=0pt]{notebox}
\newmdenv[linewidth=0pt,roundcorner=10pt,backgroundcolor=gray!5,%
innertopmargin=1em,innerbottommargin=1em,%
innerleftmargin=2em,innerrightmargin=2em,%
skipbelow=1em,skipabove=2em,%
splittopskip=\topskip]{Quote}
\newmdenv[linecolor=red,middlelinewidth=0.5pt,%
roundcorner=3pt,backgroundcolor=white,%
innertopmargin=0em,innerbottommargin=1em,%
innerleftmargin=1ex,innerrightmargin=1ex,%
skipbelow=0.5em,skipabove=0em,%
splittopskip=\topskip]{AlgoBox}
\newenvironment{note}%
{\begin{notebox}%
		\begin{list}{\textcolor{red}{$\dag$}}{%
				\leftmargin1.8em\labelwidth1.0em\labelsep0.5em
				\itemsep0pt\itemindent0pt\parsep0pt\topsep0pt}%
			\item}
		{\end{list}\end{notebox}}
\newenvironment{think}%
{\begin{notebox}%
		\begin{list}{\raisebox{0ex}{\textcolor{red}{$\clubsuit$}}}{% \HandPencilLeft  \WhiteKnightOnWhite
				\leftmargin1.8em\labelwidth1.0em\labelsep0.5em
				\itemsep0pt\itemindent0pt\parsep0pt\topsep0pt}%
			\item}
		{\end{list}\end{notebox}}
%\newenvironment{subtit}%
%    {\begin{Block}\centering\large\hei}
%    {\end{Block}}

%===== 自定义列表
\newcounter{muni}
\newenvironment{nlist}%
{\begin{list}{{\hfill\upshape (\arabic{muni})}}%
		{\usecounter{muni}\leftmargin30pt\labelwidth24pt\labelsep.5em%
			\itemsep2pt\parsep0pt\topsep0pt\itemindent0pt}}
	{\end{list}}
\newenvironment{blist}%
{\begin{list}{{\hfill\raisebox{1.12pt}{$\bullet$}}}{%
			\leftmargin2em\labelwidth1.5em\labelsep0.5em
			\itemsep0pt\itemindent0pt\parsep0pt\topsep0pt}}
	{\end{list}}
\newcounter{exer}\numberwithin{exer}{chapter}
\newcounter{exertmp}\setcounter{exertmp}{0}
\newenvironment{exerlist}%
{%\noindent{\color{red}\rule{\textwidth}{1pt}}\par
	\begin{list}{{\hfill\color{blue}\upshape{\textit{练习}} \arabic{chapter}.\arabic{exer}}}%
		{\usecounter{exer}\setcounter{exer}{\theexertmp}%
			\leftmargin30pt\labelwidth24pt\labelsep.5em%
			\itemsep0.5em\parsep0pt\topsep0pt\itemindent0pt}}
	{\end{list}}
%    {\setcounter{exertmp}{\value{exer}}\end{list}}

%===== list
\usepackage{enumitem}
\setlist{itemsep=0.05\baselineskip,itemindent=0em,%
	partopsep=0pt,parsep=0ex,topsep=0.1\baselineskip,%
	labelwidth=1.0em,labelsep=0.5em,leftmargin=2.5em}
\setlist[enumerate,1]{label=(\arabic*), ref=(\arabic*)}
\setlist[enumerate,2]{label=\emph{\alph*}), ref=\theenumi.\emph{\alph*}}
\setlist[enumerate,3]{label=\roman*), ref=\theenumii.\roman*}

%%===== 定理环境
\usepackage[amsmath,thref,thmmarks,hyperref]{ntheorem}
\theorempreskipamount1.2em  % spacing before the environment
\theorempostskipamount0em % spacing after the environment
%\theorempostwork{\noindent}
\theoremstyle{plain}
\theoremheaderfont{\normalfont\rmfamily\bfseries\color{blue}}
\theorembodyfont{\normalfont\rmfamily\BWKai\color{black}}
\theoremindent0em
\theoremseparator{\hspace{0.2em}}
\theoremnumbering{arabic}
%\newtheorem{theorem}{\color{blue}定理}[chapter]
%\newtheorem{definition}{\color{blue}定义}[chapter]
%\newtheorem{property}[theorem]{\color{blue}性质}
%\newtheorem{lemma}{\color{blue}引理}[chapter]
%\newtheorem{corollary}{\color{blue}推论}[chapter]
%\newtheorem{remark}{\color{blue}注记}[chapter]
\colorlet{thmcolor}{green}


%\RequirePackage[most]{tcolorbox}
\definecolor{tssteelblue}{RGB}{70,130,180}
\definecolor{tsorange}{RGB}{255,138,88}
\definecolor{tsblue}{RGB}{23,74,117}
\definecolor{tsforestgreen}{RGB}{21,122,81}
\definecolor{tsyellow}{RGB}{255,185,88}
\definecolor{tsgrey}{RGB}{200,200,200}
\newenvironment{boxdefinition}
{\begin{tcolorbox}
		[enhanced jigsaw,breakable,pad at break*=1mm,
		colback=yellow!10!white,boxrule=0pt,frame hidden,
		borderline west={1.5mm}{-2mm}{tsforestgreen}]}
{\end{tcolorbox}}

\newtheorem{envdefinition}{习题}%[chapter]

\newenvironment{yyEx}
{\begin{boxdefinition}\begin{envdefinition}}
{\end{envdefinition}\end{boxdefinition}}

\newtheorem*{yySolution}{\color{blue}解:}
\newtheorem*{yySolution2}{\color{blue}另解:}
\newtheorem*{yyProof}{\color{blue}证明:}

\newenvironment{boxtheorem}
{\begin{tcolorbox}
		[enhanced jigsaw,breakable,pad at break*=1mm,
		colback=black!5,colframe=tsorange]}
	{\end{tcolorbox}}
\newtheorem{envtheorem}{定理}
\newenvironment{theorem}
{\begin{boxtheorem}\begin{envtheorem}}
		{\end{envtheorem}\end{boxtheorem}}

\newtheorem{envdef}{定义}
\newenvironment{definition}
{\begin{boxtheorem}\begin{envdef}}
		{\end{envdef}\end{boxtheorem}}

\newtheorem{envlemma}{引理}
\newenvironment{boxlemma}
{\begin{tcolorbox}
		[enhanced jigsaw,breakable,pad at break*=1mm,
		colback=tsyellow!20,boxrule=0pt,frame hidden]}
	{\end{tcolorbox}}
\newenvironment{lemma}
{\begin{boxlemma}\begin{envlemma}}
		{\end{envlemma}\end{boxlemma}}
	

%%
%%\theoremheaderfont{\normalfont\itshape\color{blue}}
%\theorembodyfont{\normalfont\rmfamily\color{black}}
%\newtheorem{example}{\color{blue}例}[chapter]

\newenvironment{proof}[1][证明]%
{\par\vspace{-2ex}\noindent\normalfont{\hei\color{blue} #1.} \upshape}
{\mbox{}\hfill\scalebox{1.2}{\ensuremath{\Box}}\medskip}
\newcommand{\mysolve}{{\upshape\hei\color{blue} 解}}

%===== 数学公式
% \setlength{\abovedisplayskip}{4pt plus1pt minus1pt}     %公式前的距离
% \setlength{\belowdisplayskip}{4pt plus1pt minus1pt}     %公式后面的距离
% \setlength{\arraycolsep}{2pt}   %在一个array中列之间的空白长度
\numberwithin{equation}{chapter}
\allowdisplaybreaks[4]
\usepackage{array}
%\usepackage{yhmath}
%\usepackage{esint} % 积分符号
%\usepackage{bbding}
%\usepackage{skak}

\usepackage[ntheorem]{empheq} % 数学公式加框和背景色
\usepackage[many]{tcolorbox}
\tcbset{highlight math %
	style={enhanced, colframe=blue!40,colback=yellow!20,arc=4pt,boxrule=1pt}}
%% Examples: equation
%\begin{equation}
%\tcbhighmath{E = mc^2}
%\end{equation}
%
%% Example: align
%\begin{empheq}[box=\tcbhighmath]{align}
%a&=b\\
%E&=mc^2 + \int_a^a x\, dx
%\end{empheq}
\newtcbox{\inlinebox}[1][]{%
	nobeforeafter, notitle, box align=center,
	fontupper=\color{blue}\bfseries, leftright skip=0.5ex,
	left=0.5mm,right=0.5mm,top=0mm,bottom=0mm,boxrule=0.8pt,
	colframe=red!80!white,colback=yellow!20!white,#1}
\newcommand{\mybox}[1]{%
	\raisebox{0.7ex}[0pt][0pt]{\inlinebox{#1}}}

%===== 算法
\usepackage{algorithm}  %\usepackage{algorithm,algorithmic}
%\usepackage[compatible]{algpseudocode}
\usepackage{algpseudocode} % algorithmicx
% algorithmicx是algorithmic的改进版,由几个子包组成,
% 包括 algpseudocode.sty, algcompatible.sty 等
\floatname{algorithm}{\color{blue} 算法}
\algrenewcommand{\algorithmiccomment}[1]{\quad{\color{red}\%\ #1}}
%\algsetup{linenosize=\small}
\numberwithin{algorithm}{chapter}
\renewcommand{\listalgorithmname}{算\ 法\ 目\ 录}
\makeatletter
\newenvironment{breakalgo}[2]{%
	\captionsetup{margin=0pt,justification=RaggedRight,singlelinecheck=false}%
	% \def\@fs@cfont{\bfseries}%
	% \let\@fs@capt\relax%
	\par\noindent%
	\begin{AlgoBox}
		\noindent\captionof{algorithm}{#1}\label{#2}%
		\vspace{-0.7\baselineskip}%
		\noindent\rule{\linewidth}{.4pt}\vspace{-0.3\baselineskip}%
	}{\end{AlgoBox}}
\newenvironment{breakalgon}{%
	\captionsetup{margin=0pt,justification=RaggedRight,singlelinecheck=false}%
	%\def\@fs@cfont{\bfseries}%
	%\let\@fs@capt\relax%
	\par\noindent%
	\medskip%
	\rule{\linewidth}{.8pt}%
	\vspace{-0.5\baselineskip}%
}{%
	\vspace{-.75\baselineskip}%
	\rule{\linewidth}{.4pt}%
	\medskip%
}
\makeatother

%===== 源代码格式
\renewcommand{\lstlistlistingname}{源代码目录}
\renewcommand{\lstlistingname}{MATLAB 源代码}
\lstset{language=Matlab}
\lstset{escapechar=`}
\lstset{basicstyle=\ttfamily\small,showstringspaces=false,tabsize=2}
\lstset{flexiblecolumns=true}
\lstset{xleftmargin=1ex,xrightmargin=1ex}
\lstset{frame=tblr,frameround=tttt}  %单线, 圆角框
%%\lstset{frame=TBLR}  %双线方框
%\lstset{frame=shadowbox,rulesepcolor=\color{blue}}
\lstset{commentstyle=\color{red},keywordstyle=\color{blue},caption=\lstname,%
	breaklines=true,backgroundcolor=\color{lightgray!20}}
%\lstset{framexleftmargin=3em,framexrightmargin=1em,framextopmargin=0.3em,framexbottommargin=0.3em}
%\lstdefinestyle{numbers}{numbers=left,stepnumber=1,numberstyle=\small,numbersep=1em}
\lstset{numbers=left, numberstyle=\small, stepnumber=1, numbersep=1em}

%\makeatletter
%\lstnewenvironment{mcode}[1][]
%  {\lstset{language=Matlab,basicstyle=\small\ttfamily,
%    numbers=none,nolol,title=\textcolor{blue}{\textsf{MATLAB}},frameround=tttt,
%    backgroundcolor=\color{lightgray!20},rulecolor=\color{blue!5!green},%
%    xleftmargin=0.5em,xrightmargin=0.5em,#1}%
%    \csname\@lst @SetFirstNumber\endcsname}
%    {\csname\@lst @SaveFirstNumber\endcsname}
%\makeatother
%\newcommand{\emcode}{\addtocounter{lstlisting}{-1}}

%===== 页眉和页脚
\pagestyle{fancy}
\fancyhf{}  %清除以前对页眉页脚的设置
% 定义页眉与正文间双隔线
%\newcommand{\makeheadrule}{%
%    \makebox[0pt][l]{\rule[.7\baselineskip]{\headwidth}{0.5pt}}%
%    \rule[0.85\baselineskip]{\headwidth}{0.8pt}\vskip-.8\baselineskip
%    }
%\makeatletter
%\renewcommand{\headrule}{%
%    {\if@fancyplain\let\headrulewidth\plainheadrulewidth\fi
%     \makeheadrule}}
%\makeatother
% 画单隔线
%\renewcommand{\headrulewidth}{0.5pt} % 页眉下面的分隔线
%\renewcommand{\footrulewidth}{0pt}   % 页脚上面的分隔线
\renewcommand{\chaptermark}[1]{\markboth{\CTEXthechapter\ \ #1}{}} % 章标题
\fancyhead[RE]{\leftmark}
\renewcommand{\sectionmark}[1]{\markright{\thesection\ \ #1}{}} % 节标题
\fancyhead[LO]{\rightmark}
\fancyhead[RO,LE]{$\cdot$\ \thepage\ $\cdot$}

%===== 习题解答
\usepackage{answers}
\Newassociation{sol}{Solution}{ans}
\newenvironment{nproof}[1][证明]% 习题解答专用
{\par\noindent\normalfont{\hei\color{blue} #1.} \color{blue}\kaishu\upshape}
{\mbox{}\hfill\scalebox{1.2}{\ensuremath{\Box}}\medskip}

%===== 自定义命令
\renewcommand{\C}{\mathbb{C}}
\newcommand{\Cm}{\mathbb{C}^{m\times m}}
\newcommand{\Cn}{\mathbb{C}^{n\times n}}
\newcommand{\Cnm}{\mathbb{C}^{n\times m}}
\newcommand{\Cmn}{\mathbb{C}^{m\times n}}
\newcommand{\R}{\mathbb{R}}
\newcommand{\Rm}{\mathbb{R}^{m\times m}}
\newcommand{\Rn}{\mathbb{R}^{n\times n}}
\newcommand{\Rmn}{\mathbb{R}^{m\times n}}
\newcommand{\Xbb}{\mathbb{X}}
\newcommand{\Pbb}{\mathbb{P}}
\newcommand{\Zbb}{\mathbb{Z}}
\newcommand{\Vbb}{{\mathbb{V}}}
\newcommand{\Wbb}{{\mathbb{W}}}
\newcommand{\Lbb}{{\mathbb{L}}}
%
\newcommand{\A}{\mathcal{A}}
\renewcommand{\H}{\mathcal{H}}
\newcommand{\K}{\mathcal{K}}
\renewcommand{\L}{\mathcal{L}}
\renewcommand{\O}{\mathcal{O}}
%\newcommand{\DD}{\mathcal{D}}
\newcommand{\FF}{\mathcal{F}}
\newcommand{\PP}{\mathcal{P}} % 集合
\newcommand{\QQ}{\mathcal{Q}} % 集合
\renewcommand{\SS}{\mathcal{S}} % 集合, 子空间
\newcommand{\WW}{\mathcal{W}} % 集合, 子空间
\newcommand{\TT}{\mathcal{T}} % 集合
\newcommand{\ZZ}{\mathcal{Z}} % 集合
\newcommand{\ZZn}{\mathcal{Z}^{n\times n}} % Z-矩阵集合

\newcommand{\II}{\mathrm{\bf I}} % 不变算子
\newcommand{\EE}{\mathrm{\bf E}} % 位移算子
%
\renewcommand{\Re}{\mathrm{Re}}
\renewcommand{\Im}{\mathrm{Im}}
\newcommand{\ii}{\bm{\mathrm{i}\,}}
\newcommand{\Ran}{\mathrm{Ran}}
\newcommand{\Ker}{\mathrm{Ker}}
\newcommand{\ddiv}{\mathrm{div}}
\newcommand{\gap}{\mathrm{gap}}
\newcommand{\vvec}{\mathrm{vec}}
%\newcommand{\myvec}[1]{\mathrm{\textbf{#1}}}
\newcommand{\rr}{\bm{\mathrm{r}}}  % Numerical Radius
\newcommand{\Co}{\mathrm{Co}}
\newcommand{\TV}{\mathrm{TV}}
\newcommand{\Toep}{\bm{\textsf{T}}}
\newcommand{\Hankel}{\bm{\textsf{H}}}
\newcommand{\Circ}{\bm{\textsf{C}}}
\newcommand{\LDLT}{\ensuremath{\mathrm{LDL}^\T}}
\newcommand\opt{{\rm opt}}
%
\newcommand{\lam}{\lambda}
\newcommand{\Lam}{\Lambda}
\newcommand{\eps}{\varepsilon}
\newcommand{\xt}{{x_*}}
\newcommand{\yt}{{y_*}}
%\newcommand{\T}{\intercal}
%\newcommand{\T}{{\raisebox{1pt}[0pt]{\scriptsize$\intercal$}}}
\renewcommand{\d}{\mspace{4mu}\mathrm{d}}
\newcommand{\p}{\partial}
\newcommand{\grad}{\nabla}
\newcommand{\adots}{\reflectbox{$\ddots$}}
\newcommand{\ie}{\emph{i.e.}}
%
\newcommand{\beq}{\begin{equation}}
\newcommand{\eeq}{\end{equation}}
\newcommand{\bbm}{\begin{bmatrix}}
	\newcommand{\ebm}{\end{bmatrix}}
\newcommand{\ol}[1]{\overline{#1}}
\newcommand{\wt}[1]{\widetilde{#1}}
\newcommand{\Cond}{\kappa}
%
\newcommand{\dis}{\displaystyle}
\newcommand{\code}[1]{\textcolor{blue}{\ttfamily #1}}
\newcommand{\function}[1]{\textcolor{blue}{\bfseries #1}}
\newcommand{\ip}[1]{\ensuremath{( #1 )}}
\newcommand{\mycite}[1]{{\upshape\cite{#1}}}
\newcommand{\myind}[1]{{\hei\upshape\color{blue} #1 }\index{#1}}
\newcommand{\myindd}[2]{{\hei\upshape\color{blue} #1}\index{#2}}
\newcommand{\mydef}{\triangleq}
\newcommand{\mycolon}{\!:\!}
\newcommand{\mymid}{\,:\,}
\newcommand{\myss}{\scriptstyle}
\newcommand{\ssrm}[1]{{\scriptscriptstyle\mathrm{#1}}}
\newcommand{\myem}[1]{{\hei\upshape\textcolor{blue}{#1}}}
\newcommand{\myif}[2]{\textbf{if} #1 \textbf{then} #2}
\newcommand{\rA}{r_{\scriptscriptstyle\! A}}
\newcommand{\lev}{\textit{lev}}
\newcommand{\bslx}{\hfill{\upshape\color{blue}(留作练习)}}
%
\DeclareMathOperator{\sspan}{span}
\DeclareMathOperator{\ddim}{dim}
\DeclareMathOperator{\diag}{diag}
\DeclareMathOperator{\tridiag}{tridiag}
\DeclareMathOperator{\mvec}{vec}
\DeclareMathOperator{\rank}{rank}
\DeclareMathOperator{\sign}{sign}
\DeclareMathOperator{\fl}{fl}
\DeclareMathOperator{\der}{D} % 导数算子
\DeclareMathOperator*{\argmin}{argmin}
\DeclareMathOperator{\tr}{tr}
\newcommand{\MATLAB}{MATLAB}
%
\DeclareMathAlphabet{\mathsfsl}{OT1}{cmss}{m}{n}
\SetMathAlphabet{\mathsfsl}{bold}{OT1}{cmss}{bx}{n}
\newcommand{\DD}{\mathsfsl{D}}
%
\newcommand{\tbc}{\textcolor{blue}{\it To be continued ... }\bigskip}
%
\renewcommand{\baselinestretch}{1.3}



\begin{document}

\frontmatter
\begin{titlepage}\Large
\renewcommand{\thefootnote}{\fnsymbol{footnote}}
  \pdfbookmark[0]{封面}{Cover}
  \begin{center}
  \vspace*{3cm}
  {\hei\Huge 工程数学学习指南} \bigskip

  {\hei\LARGE 杨~~勇~~编著} \medskip

  \today

  \end{center}
\end{titlepage}




\clearpage{\pagestyle{empty}}
\vspace*{0.04\textheight}
\begin{center}\Large\
	\textbf{
		前~~言\bigskip
	}
\end{center}


本书是《工程数学: 复变函数、矢量分析与场论、数学物理方法》一书的课后习题题解. 郭玉翠教授的这本书继承了许多位北京邮电大学数学系教授的教学经验,但是令人遗憾的是此书的习题却一直没有一份答案可供参考. 而作者写本书的目的是在这里给大家提供帮助, 力求使这门课不那么难学.

文学家王蒙曾说过: 最高的数学和最高的诗一样, 都充满了理想, 充满了智慧, 充满了创造, 充满了章法, 充满了和谐也充满了挑战. 诗和数学又都充满了灵感, 充满了激情, 充满了人类的精神力量. 那些从诗中体验到数学的诗人是好诗人, 那些从数学中体会到诗意的人是好数学家. 最高的诗是数学, 这是文学家对数学的推崇. 亲爱的同学们, 诗言志, 希望你们能从这里读出诗的理想, 接受诗的智慧, 学会诗的创造, 掌握诗的章法, 体验诗的和谐, 勇敢地迎接未来的挑战.

基础数学分为代数、几何、分析三大类. 而工程数学则是一门应用数学, 同学们在前面的学习中惟独没有这种应用数学. 初次走进数学物理方法, 第一个感觉是难, 这个感觉恰如第一次接触“诗经”的“风、雅、颂”一样. 不要怕难, 这是诗意的挑战. 在学习这门课的过程中, 抽象概念会变得形象直观, 给我们打下深刻的烙印. 通过严格的逻辑推理和抽象思维的训练, 这些基础知识将融化在你的血液中. 你会感觉数学和诗一样, 充满了灵感, 充满了激情, 充满了人类的精神力量, 你的数学生涯也就化为诗样的人生.

郭玉翠教授这本书的习题是多了一些, 这也要请同学们根据自己的情况适当地选择, 不必题题都做; 更不要因为有几个题做不出来而失去信心. 对作者而言, 一题之解, 有时也累日始成, 可以说不时就遇到了不小的困难, 倾注了不少心力. 对读者而言, 作者切望本书是备而不用、备而少用. 有能力的同学应当尝试独立完成每一道习题. 如碰到一个题一时做不出来, 宁肯暂时搁一搁, 也不要轻易翻看本解答. 譬如登山, 经过艰苦努力上了峰顶, 自有其乐趣和成就感. 反之, 如在未尽全力之前就任人抬上去,则不惟无益, 实足以挫折信心.

尽管做了不少努力以使读者学习本课程会比较容易些, 但是, 理解这样的课程内容并获得灵活使用它解决具体问题的能力是个复杂的过程. 这门课程的难度相比数学分析会呈现出一个飞跃. 初学者不能指望这个过程会在完全不遇到困难的情况下轻松地通过. 大家会遇到许多与前不同的、大大小小的困境, 这是完全可以理解的, 在一定意义上来说也是正常的. 具有强烈求知欲的学而不厌者方能成功地完成这个困难的过程.如果读者在阅读本书时遇到了困难, 可与作者联系, 电子邮件:\url{Yang945841548@bupt.edu.cn}

龚政同学仔细审阅了本书初稿, 并对第九章的部分推导提供了很好的修改意见, 特致谢意. 作者在参考了这些意见的基础上, 对该章的初稿进行了修改, 使得该章的质量有了很大的提高. 作者还要感谢母校的同学们给予的支持和帮助.

囿于作者的水平和经验, 书中不妥以至谬误之处, 在所难免. 尚祈同学们不吝指教. 最后, 赠送一句话与大家共勉:
\begin{center}
	{\BWKai\color{blue}
		纸上得来终觉浅,绝知此事要躬行。\bigskip
	}
\end{center}
\begin{flushright}
	{\kaishu{~~杨~~勇~~~~~~~~~~~}}
	
	2019年 9 月于北京昌平
\end{flushright}



\clearpage{\pagestyle{empty}\cleardoublepage}
\pdfbookmark[1]{目次}{Contents}
\tableofcontents
%\faketableofcontents  % for minitoc
%\clearpage{\pagestyle{empty}\cleardoublepage}
%\listoffigures
%\clearpage{\pagestyle{empty}\cleardoublepage}
%\listoftables
%\clearpage{\pagestyle{empty}\cleardoublepage}
%\listofalgorithms
%\clearpage{\pagestyle{empty}\cleardoublepage}
%\lstlistoflistings %显示所有源代码目录


\mainmatter

\setcounter{chapter}{0}
\chapter{复变函数及其导数与积分}
	工程数学的第一部分是复变函数, 在这一个部分有两个地方值得注意. 
	
	(1)要强调复变量$z$和$\overline{z}$的作用, 利用它来实现实变量$x$和$y$与复变量之间对于各种关系和公式的互换. 希望同学们能掌握怎样将实分析的问题用复变量表示; 怎样将复分析中的各种漂亮定理运用到其他方面.
	
	(2)在学习中应当突出级数和积分这两种表示方法. 可以说,由Weierstrass提出的\myind{幂级数方法}和由Cauchy提出的\myind{积分表示方法}是解析函数的两个主要研究方法, 这两种方法交替地出现成为了复变函数部分的主线. 同学们应当尽早了解和熟悉它们. 大家要学会尽可能将复分析中涉及的各个定理用这两种方法或其推论给出. 例如, 开映射定理可以作为幂级数的局部性质的推论; 单值性定理则可从幂级数的收敛圆的性质导出.
	
	顾名思义,"复分析"是研究以复数为变量的函数. 在这一章中, 我们要先温习中学中所学过的复数的各种表示及代数运算, 讨论复平面的拓扑; 然后将平面$\mathbb{R}^2$上的极限理论推广到复平面, 将实函数关于变量$x$和$y$的可导性和求导关系用复变量来表示; 之后, 讨论一下扩充复平面. 前面这一小部分仅仅讨论复函数对实变量$x$和$y$的可导性与微分, 这些内容仅仅是微积分的简单推广, 利用微积分的工具即可解决. 在复变函数的理论中, 我们主要要讨论的是关于复变量$z$可导的复函数. 即极限$\begin{lgathered} \lim_{z\to z_0}\frac{f(z)-f(z_0)}{z-z_0} \end{lgathered}$存在的函数————解析函数. 引入解析函数的概念后, 要讨论一些它的基本性质, 介绍一些初等解析函数. 这一章的最后, 要定义复函数的曲线积分, 并介绍重要的Cauchy公式. Cauchy公式是1825年左右Cauchy在研究流体力学时发现的. 他将$\mathbb{C}$中区域$\Omega$上的解析函数表示为沿$\Omega$的边界$\partial\Omega$上的含参变量积分, 为解析函数的研究提供了一个非常有力的工具. 解析函数的许多性质可以由Cauchy公式得到. 这一小部分是本章最后要介绍的内容.
\section{复数域}
	在中学数学中我们已经学过复数的表示和代数运算, 本节我们将从复数的定义开始, 介绍一些复数的基本性质.
    \begin{definition}
        一复数是个有序的实数对(简称序对)$(a,b)$. "有序的"的意思是, 如果$a\neq b$,那么$(a,b)$和$(b,a)$就认为是不同的.
    \end{definition}
	设$x=(a,b),y=(c,d)$是两个复数. 当且仅当$a=c$且$b=d$时, 我们写成$x=y$.(注意, 这个定义并不多余). 我们定义复数的加法与乘法:
	\begin{align}
	    x+y&\triangleq (a+c,b+d),\\
	  xy&\triangleq (ac-bd,ad+bc).  
	\end{align}
	二次方程$x^2+1=0$在实数中无解, 在此, 我们可以为它形式地引入虚根$\mathrm{i} = (0,1)$, $\mathrm{i}$满足$\mathrm{i}^2 = -1$. 由此, 我们可以将复数表示成较惯用的记号$(a,b) = a+b\mathrm{i}$.
	
	\begin{theorem}[Cauchy准则/复数域完备性定理]
		复数列$\{z_n \}$在$\mathbb{C}$中收敛的充分必要条件是$\{z_n\}$为一个Cauchy列, 即
		\begin{equation*}
			\forall\varepsilon>0,\exists N\in\mathbb{N} (n,m>N\Rightarrow \abs{z_n-z_m}<\varepsilon).
		\end{equation*}
	\end{theorem}
	
	\begin{theorem}[Euler 公式]
		\begin{equation*}
			\mathrm{e}^{\mathrm{i}z} = \cos z+\mathrm{i}\sin z.
		\end{equation*}
	\end{theorem}

	\begin{definition}[复平面上拓扑的几个定义]
		$z_0$的$\varepsilon$-圆盘邻域为\begin{equation*}
			U(z_0,\varepsilon) = \{z\big| \abs{z-z_0}<\varepsilon \},
		\end{equation*}
		$z_0$的$\varepsilon$-空心圆盘邻域为\begin{equation*}
			U_0(z_0,\varepsilon) = U(z_0,\varepsilon)\backslash \{z_0\}
		\end{equation*}
		$S\subset \mathbb{C}$称为开集,若\begin{equation*}
			\forall z\in S,\exists\varepsilon>0 (D(z,\varepsilon)\subset S).
		\end{equation*}
		开集在$\mathbb{C}$中的余集称为闭集.
		$\mathbb{C}$中任意不空的集合$F$的直径$\mathrm{diam}F$定义为\begin{equation*}
			\mathrm{diam}F = \sup\{ \abs{z-w}\big| z,w\in F \}.
		\end{equation*}
	\end{definition}

	\begin{theorem}[闭集套定理]
		$\{F_n\}$是$\mathbb{C}$中单调非增的一列非空闭集, 并且$\mathrm{diam}F_n\to 0$, 则存在唯一的一个点$z_0\in\mathbb{C}$, 使得
		\begin{equation*}
			\{z_0\} = \bigcap_{n = 1}^{+\infty}F_n.
		\end{equation*}
	\end{theorem}

	\begin{theorem}[开覆盖定理]
		$\mathbb{C}$中任意有界闭集都是紧集, 换言之, 它的每个开覆盖都有有限子覆盖.
	\end{theorem}
	
	\begin{theorem}[Bolzano-Weierstrass]
		$\mathbb{C}$中任意有界序列必有收敛子列.
	\end{theorem}

\section{习题1-20}


\begin{yyEx}
	计算下列各题.
	\begin{enumerate}
		\item 设$z = \begin{lgathered}\frac{(1+\mathrm{i})(2-\mathrm{i})(3-\mathrm{i})}{(3+\mathrm{i})(2+\mathrm{i})}\end{lgathered}$, 求$\abs{z}$.
		\item 当$z = \begin{lgathered}\frac{1+\mathrm{i}}{1-\mathrm{i}}\end{lgathered}$时, 求$z^{100}+z^{75}+z^{50}$的值.
		\item 设$z = (2-3\mathrm{i})(-2+\mathrm{i})$, 求$\arg z$.
		\item 求复数$\begin{lgathered}\frac{(\cos 5\theta + \mathrm{i}\sin 5\theta)^2}{(\cos 3\theta - \mathrm{i}\sin 3\theta)^2}\end{lgathered}$的指数表示式.
		\item 求复数$\begin{lgathered}z = \tan\theta - \mathrm{i}~\left(\frac{\pi}{2}<\theta<\pi\right)\end{lgathered}$的三角表示式.
		\item 设$f(z) = 1-\overline{z}$, $z_1 = 2+3\mathrm{i},z_2 = 5-\mathrm{i}$,求$\overline{f(z_1-z_2)}$.
	\end{enumerate}
\end{yyEx}

\begin{yySolution}
	\begin{enumerate}
		\item \begin{equation*}
			\abs{z} = \abs{1+\mathrm{i}}\cdot\frac{\abs{2-\mathrm{i}}}{\abs{2+\mathrm{i}}}\cdot\frac{\abs{3-\mathrm{i}}}{\abs{3+\mathrm{i}}} = \abs{1+\mathrm{i}} = \sqrt{2}.
		\end{equation*}
		\item 可以计算得到$z =\begin{lgathered}\frac{1+\mathrm{i}}{1-\mathrm{i}} = \mathrm{i}\end{lgathered}$, 因此, $z^4 = 1$.
		所以,\begin{equation*}
			z^{100}+z^{75}+z^{50} = 1 +\mathrm{i}^3 +\mathrm{i}^2 = -\mathrm{i}.
		\end{equation*}
		\item $z = (2-3\mathrm{i})(-2+\mathrm{i}) = -1+8\mathrm{i}$位于复平面的第二象限.
		
		因此, 它的辐角主值是\begin{equation*}
				\arg z = \pi - \arctan 8.
		\end{equation*}
		\item \begin{equation*}
			\frac{(\cos 5\theta + \mathrm{i}\sin 5\theta)^2}{(\cos 3\theta - \mathrm{i}\sin 3\theta)^2} = \frac{(\mathrm{e}^{5\mathrm{i}})^2}{(\mathrm{e}^{-3\mathrm{i}})^2} = \mathrm{e}^{16\mathrm{i}}
		\end{equation*}
		\item \begin{equation*}
			\begin{split}
				z &= \tan\theta - \mathrm{i} = \frac{1}{\cos\theta}\left( \sin\theta - \mathrm{i}\cos\theta \right) \\
				&= \sec\theta \left[\cos(\theta-\frac{\pi}{2})+\sin(\theta-\frac{\pi}{2})   \right]
			\end{split}
		\end{equation*}
		\item 注意到$\overline{f(z)} = 1-z$, 因此,\begin{equation*}
			\overline{f(z_1-z_2)} = 1-[(2+3\mathrm{i})-(5-\mathrm{i})] = 4-4\mathrm{i}.
		\end{equation*}
	\end{enumerate}
\end{yySolution}


\begin{yyEx}
	证明下列各题.
	\begin{enumerate}
		\item 若$z$为非零复数, 证明$\abs{z^2 - \overline{z}^2} \leqslant 2z\overline{z}$.
		\item 设$z = x+\mathrm{i}y$,试证$\begin{lgathered}\frac{\abs{x}+\abs{y}}{\sqrt{2}}\leqslant \abs{z} \leqslant \abs{x}+\abs{y}\end{lgathered}$.
		\item 试证$\begin{lgathered}\frac{z_1}{z_2}\geqslant 0(z_2\neq 0)\end{lgathered}$的充要条件为$\abs{z_1+z_2} = \abs{z_1}+\abs{z_2}$.
		\item 证明使得$z^2 = \abs{z}^2$成立的$z$一定是实数.
		\item 设复数$z\neq \pm \mathrm{i}$, 试证$\begin{lgathered}\frac{z}{1+z^2}\end{lgathered}$是实数的充要条件为$\abs{z} = 1$或$\mathrm{Im}(z) = 0$.
	\end{enumerate}
\end{yyEx}

\begin{yySolution}
		\begin{enumerate}
			\item 根据三角不等式得到:
			\begin{equation*}
			\abs{z^2 - \overline{z}^2} \leqslant \abs{z^2} + \abs{\overline{z}^2} = z\overline{z} + z\overline{z} = 2z\overline{z}.
			\end{equation*}
			
			\item 根据三角不等式得到:
			\begin{equation*}
			\abs{z} = \abs{x+\mathrm{i}y} \leqslant \abs{x} + \abs{\mathrm{i}y} = \abs{x} + \abs{y}.
			\end{equation*}
			另一方面, 根据Cauchy不等式,
			\begin{equation*}
			\abs{x}+\abs{y} = 1\cdot\abs{x}+1\cdot\abs{y}\leqslant \sqrt{(1^2+1^2)(x^2+y^2)} = \sqrt{2}\abs{z}.
			\end{equation*}
			
			\item 注意到\begin{equation*}
			\begin{split}
			&\abs{z_1+z_2}^2 = \abs{z_1}^2 + 2\mathrm{Re}z_1\overline{z_2} + \abs{z_2}^2; \\
			&(\abs{z_1}+\abs{z_2})^2 = \abs{z_1}^2 + 2\abs{z_1\overline{z_2}}+\abs{z_2}^2.
			\end{split}
			\end{equation*}
			因此,$\abs{z_1+z_2} = \abs{z_1}+\abs{z_2}$当且仅当$\mathrm{Re}z_1\overline{z_2} = \abs{z_1\overline{z_2}}$, 即$\mathrm{Im}z_1\overline{z_2} = 0$.
			
			可以看到\begin{equation*}
			\abs{z_1+z_2} = \abs{z_1}+\abs{z_2} \Leftrightarrow z_1\overline{z_2} = \abs{z_1\overline{z_2}}\geqslant 0 \Leftrightarrow \frac{z_1}{z_2}\geqslant 0
			\end{equation*}
			
			\item 设$z^2 = \abs{z}^2$. 若$z = 0$, 则$z$是实数.
			
			若$z\neq 0$, 由于复数域是域, 所以它的乘法对于每个非零元都有唯一逆元. 因此,
			\begin{equation*}
			z = z^{-1}z^2 = z^{-1}\abs{z}^2 = z^{-1}z\overline{z} = \overline{z}.
			\end{equation*}
			这说明$\mathrm{Im}z = 0$,即$z$是实数.
			
			\item 引入$w = z/(1+z^2)$. 
			则\begin{equation*}
			w\text{是实数} \Leftrightarrow w = \overline{w} \Leftrightarrow \overline{z}(1+z^2) = z(1+\overline{z}^2)
			\Leftrightarrow (z-\overline{z})(1-\abs{z}^2) = 0.
			\end{equation*}
		\end{enumerate}	
\end{yySolution}


\begin{yyEx}
	设$x,y$是实数,$z_1 = x+\sqrt{11}+y\mathrm{i},~z_2 = x-\sqrt{11}+y\mathrm{i}$且有$\abs{z_1}+\abs{z_2} = 12$.问动点$(x,y)$的轨迹是什么曲线.
\end{yyEx}

\begin{yySolution}
	设$\abs{z_1} = 6+t$, 则$\abs{z_2} = 6-t$. 这两式平方后作差得到:
	\begin{equation*}
		4x\sqrt{11} = 24t.
	\end{equation*}
	代回$\abs{z_1} = 6+t$, 再平方, 得到
	\begin{equation*}
		x^2 + 2x\sqrt{11} + 11 + y^2 = (6+\frac{\sqrt{11}}{6}x)^2.
	\end{equation*}
	整理, 得\begin{equation*}
		\frac{x^2}{36}+\frac{y^2}{25} = 1.
	\end{equation*}
	这说明动点$(x,y)$的轨迹是一个焦点在$x$轴上的椭圆.
\end{yySolution}

\begin{yyEx}
	设$z$为复数, 求方程$z + \abs{\overline{z}} = 2+\mathrm{i}$的解.
\end{yyEx}

\begin{yySolution}
	根据题意, 于是有\begin{equation*}
		z = 2-\abs{z} + \mathrm{i}.
	\end{equation*}
	两端取模后平方, 得到:
	\begin{equation*}
		\abs{z}^2 = (2-\abs{z})^2 + 1.
	\end{equation*}
	解得:$\abs{z} = 5/4$.
	于是\begin{equation*}
		z = \frac{3}{4} + \mathrm{i}.
	\end{equation*}
\end{yySolution}

\begin{yyEx}
	满足不等式$\abs{\frac{z-\mathrm{i}}{z+\mathrm{i}}}\leqslant 2$的所有点$z$构成的集合是有界区域还是无界区域?
\end{yyEx}

\begin{yySolution}
	当$z\in \mathbb{R}$时, 我们有
	\begin{equation*}
		\abs{\frac{z-\mathrm{i}}{z+\mathrm{i}}} \equiv 1\leqslant 2.
	\end{equation*}
	由于$\mathbb{R}$在复平面上是无界的. 所以, 满足不等式$\abs{\frac{z-\mathrm{i}}{z+\mathrm{i}}}\leqslant 2$的所有点$z$构成的集合是一个无界区域.
\end{yySolution}

\begin{yyEx}
	若复数$z$满足$z\overline{z} + (1-2\mathrm{i})z+(1+2\mathrm{i})\overline{z} + 3 = 0$,试求$\abs{z+2}$的取值范围.
\end{yyEx}

\begin{yySolution}
	回忆圆方程的复数表示: 若$C\in\mathbb{R},B\in\mathbb{C},B\overline{B}-C>0$, 则方程
	\begin{equation*}
		z\overline{z} + B\overline{z} + \overline{B}z + C = 0
	\end{equation*}
	表示一个以$\begin{lgathered}
		-B
	\end{lgathered}$为圆心, $\begin{lgathered}
		\sqrt{B\overline{B}-C}
	\end{lgathered}$为半径的圆周.
	
	因此, 本题中的复数$z$位于一个以$-1-2\mathrm{i}$为圆心, 半径为$\sqrt{2}$的圆周上.
	
	可以看到点$z_0 = -2$到圆心的距离为$\sqrt{5}$, 因而, $\abs{z+2}$的取值范围为$[\sqrt{5}-\sqrt{2},\sqrt{5}+\sqrt{2}]$
\end{yySolution}

\begin{yyEx}
	设$a\geqslant 0$, 在复数集$\mathbb{C}$中解方程$z^2 + 2\abs{z} = a$.
\end{yyEx}

\begin{yySolution}
	
\end{yySolution}

\begin{yyEx}
	方程$\abs{z+2-3\mathrm{i}} = \sqrt{2}$代表什么曲线.
\end{yyEx}

\begin{yySolution}
	这方程在复平面上表示一个以$-2+3\mathrm{i}$为圆心, $\sqrt{2}$为半径的圆周.
\end{yySolution}

\begin{yyEx}
	不等式$\abs{z-2}+\abs{z+2}<5$所表示的区域的边界是什么曲线?
\end{yyEx}

\begin{yySolution}
	不难看出, $\abs{z-2}+\abs{z+2}<5$所表示的区域是一个开的椭圆盘. 显然, 它的边界是\begin{equation*}
		\{z\big|~\abs{z-2}+\abs{z+2} = 5 \}
	\end{equation*}
	仿照习题1.3的步骤, 可化简得到:
	\begin{equation*}
		\left\{z = x+\mathrm{i}y \bigg| \frac{x^2}{25/4}+\frac{y^2}{9/4} =1 \right\}
	\end{equation*}
\end{yySolution}
\medskip
\begin{note}
	上面这个解答中用到了一个"显然", 严格来说这是需要证明的. 证明如下:
	引入$G = \{z\big|~\abs{z-2}+\abs{z+2} < 5 \},~F = \{z\big|~\abs{z-2}+\abs{z+2} \leqslant 5 \}$.
	
	任取$z_0\in G$, 存在$\varepsilon = [5-(\abs{z_0-2}+\abs{z_0+2})]/2$, 这时,$\forall z\in U(z_0,\varepsilon)$,有
	\begin{equation*}
		\begin{split}
			\abs{z-2}+\abs{z+2} &= \abs{z-z_0+z_0-2}+\abs{z-z_0+z_0+2}\\
			&\leqslant 2\abs{z-z_0} + \abs{z_0-2}+\abs{z_0+2} \\
			&< 2\varepsilon + \abs{z_0-2}+\abs{z_0+2} = 5
		\end{split}
	\end{equation*}
	即$\forall z_0\in G,\exists \varepsilon>0, U(z_0,\varepsilon)\subset G$.这说明$G$是开集.
	
	同理, 任取$z_0\in F^c$, 则存在$\varepsilon = [(\abs{z_0-2}+\abs{z_0+2})-5]/2$, 使得$\forall z_0\in U(z_0,\varepsilon)$,有\begin{equation*}
		\begin{split}
		\abs{z-2}+\abs{z+2} &= \abs{z-z_0+z_0-2}+\abs{z-z_0+z_0+2}\\
		&\geqslant \abs{z_0-2}+\abs{z_0+2}-2\abs{z-z_0}\\
		&>\abs{z_0-2}+\abs{z_0+2}-2\varepsilon = 5.
		\end{split}
	\end{equation*}
	即$\forall z_0\in F^c,\exists \varepsilon>0, U(z_0,\varepsilon)\subset F^c$.这说明$F^c$是开集, 也就是$F$是闭集.
	
	由于$G$是开集,$F$是闭集, $G\subset F$, 所以$G^o = G$, $\overline{G} \subset F$.
	
	注意到集合$G$和$F$分别可以化简为
	\begin{equation*}
		\begin{split}
			G = \left\{z = x+\mathrm{i}y \bigg| \frac{x^2}{25/4}+\frac{y^2}{9/4} <1 \right\};\\
			F = \left\{z = x+\mathrm{i}y \bigg| \frac{x^2}{25/4}+\frac{y^2}{9/4}\leqslant 1 \right\}.
		\end{split}
	\end{equation*}
	因而$\forall z\in F$, 令$\begin{lgathered}
		z_k = (1-\frac{1}{k})z
	\end{lgathered}$, 则$G\ni z_k\to z$.
	这说明$F$中的每个点都是$G$的聚点, 也就是$F\subset \overline{G}$.
	
	因此, $F = \overline{G}$, $\partial G = \overline{G}\backslash G^o = F\backslash G$.
	
	即\begin{equation*}
	\partial G = \left\{z = x+\mathrm{i}y \bigg| \frac{x^2}{25/4}+\frac{y^2}{9/4} =1 \right\}
	\end{equation*}
\end{note}

\begin{yyEx}
	求方程$\begin{lgathered}\abs{\frac{2z-1-i}{2-(1-i)z}} = 1\end{lgathered}$所表示曲线的直角坐标方程.
\end{yyEx}

\begin{yySolution}
	回忆初等数学中的Appollonius定理:
	当$z_1,z_2\in \mathbb{C},k>0$且$k\neq 1$时, 集合\begin{equation*}
		\left\{ z\in\mathbb{C}\bigg| \abs{\frac{z-z_1}{z-z_2}}  = k \right\}
	\end{equation*}
	表示一个圆心在$(z_1-k^2z_2)/(1-k^2)$, 半径是$\abs{(z_1-z_2)k/(1-k^2)}$的圆周.
	
	因此, 该题目中的方程所表示的曲线是一个圆心在原点, 半径是$1$的圆周. 因此, 它的直角坐标方程是\begin{equation*}
		x^2+y^2 = 1.
	\end{equation*}
\end{yySolution}

\begin{yyEx}
	方程$\abs{z+1-2\mathrm{i}} = \abs{z-2+\mathrm{i}}$所表示的曲线是连接点$\underline{\quad\quad}$和$\underline{\quad\quad}$的线段的垂直平分线.
\end{yyEx}

\begin{yySolution}
	这方程表示\begin{equation*}
		d(z,-1+2\mathrm{i}) = d(z,2-\mathrm{i})
	\end{equation*}
	即这方程是表示点$-1+2\mathrm{i}$和$2-\mathrm{i}$连线的垂直平分线.
\end{yySolution}

\begin{yyEx}
	对于映射$\begin{lgathered}\omega = \frac{1}{2}\left( z+\frac{1}{z} \right)\end{lgathered}$, 求出圆周$\abs{z} = 4$的像.
\end{yyEx}

\begin{yySolution}
	这正是熟知的Жуко́вский(儒可夫斯基)函数. 因此,本题中不妨引入$z = r\mathrm{e}^{i\theta}$, 其中$r = 4$,并设$w = u + \mathrm{i}v$,
	则由\begin{equation*}
		\frac{1}{2}\left( z+\frac{1}{z} \right) = \frac{1}{2}\left( r\mathrm{e}^{i\theta} + \frac{1}{r\mathrm{e}^{i\theta}} \right) = \frac{1}{2}\left[ r(\cos\theta+\mathrm{i}\sin\theta) + \frac{1}{r}(\cos\theta-\mathrm{i}\sin\theta)  \right]
	\end{equation*}
	得到:
	\begin{equation*}
		\begin{dcases}
			&u = \frac{1}{2}\left( r+\frac{1}{r} \right)\cos\theta, \\
			&v = \frac{1}{2}\left( r-\frac{1}{r} \right)\sin\theta.
		\end{dcases}
	\end{equation*}
	这正是椭圆的参数方程表示.
	
	代入$r=4$化为椭圆的标准方程得到:
	\begin{equation*}
		\frac{u^2}{(17/8)^2}+\frac{v^2}{(15/8)^2} = 1.
	\end{equation*}
\end{yySolution}

\begin{yyEx}
	证明$\begin{lgathered}\lim\limits_{z\to z_0}\frac{\mathrm{Im}(z)-\mathrm{Im}(z_0)}{z-z_0}\end{lgathered}$不存在.
\end{yyEx}

\begin{yySolution}
	让$z-z_0$沿$y=0$趋于零, 我们有
	\begin{equation*}
		\lim\limits_{z\to z_0}\frac{\mathrm{Im}(z)-\mathrm{Im}(z_0)}{z-z_0} = \lim\limits_{x\to 0} \frac{0}{x} = 0.
	\end{equation*}
	让$z-z_0$沿$x=0$趋于零, 则
	\begin{equation*}
		\lim\limits_{z\to z_0}\frac{\mathrm{Im}(z)-\mathrm{Im}(z_0)}{z-z_0} = \lim\limits_{y\to 0} \frac{y}{\mathrm{i}y} = -\mathrm{i}.
	\end{equation*}
	二者不等, 说明此极限不存在.
\end{yySolution}

\begin{yyEx}
	求$\lim\limits_{z\to 1+\mathrm{i}}\left( 1+z^2+z^4 \right)$.
\end{yyEx}

\begin{yySolution}
    因为此函数连续, 所以我们有
    \begin{equation*}
        \lim\limits_{z\to 1+\mathrm{i}}\left( 1+z^2+z^4 \right) = 1+(1+\mathrm{i})^2+(1+\mathrm{i})^4 = -3+2\mathrm{i}.
    \end{equation*}
\end{yySolution}

\begin{yyEx}
	证明函数$f(z) = u(x,y) + iv(x,y)$在点$z_0 = x_0 + iy_0$处连续的充要条件是$u(x,y)$和$v(x,y)$在$(x_0,y_0)$处连续.
\end{yyEx}

\begin{yySolution}
	只需证明以下的引理.
	\begin{lemma}
		设$f(z)$是定义在集合$S$上的函数,$z_0$是这集合的极限点,\\ 则$\lim\limits_{z\to z_0}f(z) = A$的充分必要条件是
		\begin{equation*}
			\lim\limits_{z\to z_0}\mathrm{Re}f(z) = \mathrm{Re}A,~~\lim\limits_{z\to z_0}\mathrm{Im}f(z) = \mathrm{Im}A.
		\end{equation*}
	\end{lemma}
	\begin{proof}
		根据复数的模的定义, 不难得到以下不等式
		\begin{equation*}
			\begin{split}
				\max\{ &\abs{\mathrm{Re}f(z)-\mathrm{Re}A},\abs{\mathrm{Im}f(z)-\mathrm{Im}A} \}\leqslant \abs{f(z)-A}\\
				&\leqslant \abs{\mathrm{Re}f(z)-\mathrm{Re}A}+\abs{\mathrm{Im}f(z)-\mathrm{Im}A}
			\end{split}
		\end{equation*}
		因此, 若$\lim\limits_{z\to z_0}f(z) = A$, 则
		\begin{equation*}
			\forall\varepsilon>0,\exists\delta>0, \bigg(
			z\in U_0(z_0,\delta)\cap S\Rightarrow f(z)\in U(A,\varepsilon)
			\bigg).
		\end{equation*}
		而上面的不等式说明
		\begin{equation*}
			f(z)\in U(A,\varepsilon) \Rightarrow \mathrm{Re}f(z)\in U(\mathrm{Re}A,\varepsilon) ~\hat{~}~ \mathrm{Im}f(z)\in U(\mathrm{Im}A,\varepsilon)
		\end{equation*}
		这说明\begin{equation*}
		\lim\limits_{z\to z_0}\mathrm{Re}f(z) = \mathrm{Re}A,~~\lim\limits_{z\to z_0}\mathrm{Im}f(z) = \mathrm{Im}A.
		\end{equation*}
		反过来, 若上面两式成立, 则
		\begin{equation*}
		\forall\varepsilon>0,\exists\delta>0, \bigg(
		z\in U_0(z_0,\delta)\cap S\Rightarrow \mathrm{Re}f(z)\in U(\mathrm{Re}A,\varepsilon/2)~\hat{~}~\mathrm{Im}f(z)\in U(\mathrm{Im}A,\varepsilon/2)
		\bigg).
		\end{equation*}
		根据上面的不等式可推出
		\begin{equation*}
		\forall\varepsilon>0,\exists\delta>0, \bigg(
		z\in U_0(z_0,\delta)\cap S\Rightarrow f(z)\in U(A,\varepsilon)
		\bigg).
		\end{equation*}
	\end{proof}
\end{yySolution}

\begin{yyEx}
	设$z = x+iy$, 试讨论下列函数的连续性.
	\begin{enumerate}
		\item $
		\begin{lgathered}
		f(z) = \begin{dcases}
		\frac{2xy}{x^2+y^2}, &z\neq 0;\\
		0,&z=0.
		\end{dcases}
		\end{lgathered}
		$
		
		\item $\begin{lgathered}
		f(z) = \begin{dcases}
		\frac{x^3y}{x^2+y^2}, &z\neq 0;\\
		0,&z=0.
		\end{dcases}
		\end{lgathered}$
	\end{enumerate}
\end{yyEx}

\begin{yyEx}
	设$f(z)$是定义在集合$S$上的函数, $z_0$是$S$的极限点.
	
	若$\lim\limits_{z\to z_0}f(z) = A\neq 0$, 则存在$\delta>0$,使得当$0<\abs{z-z_0}<\delta$时有$\abs{f(z)}>\frac{1}{2}\abs{A}$.
\end{yyEx}

\begin{yyProof}
	根据$\lim\limits_{z\to z_0}f(z) = A\neq 0$, 知道
	\begin{equation*}
	\forall\varepsilon>0,\exists\delta>0, \bigg(
	z\in U_0(z_0,\delta)\cap S\Rightarrow f(z)\in U(A,\varepsilon)
	\bigg).
	\end{equation*}
		
	现在, 取定$\varepsilon = \abs{A}/2$,存在对应的$\delta>0$,使得当$0<\abs{z-z_0}<\delta$时有$f(z)\in U(A,\varepsilon)$.
	
	这时根据三角不等式, 有\begin{equation*}
		\abs{f(z)} = \abs{A+f(z)-A} \geqslant \abs{A} - \abs{f(z)-A} > \abs{A} - \varepsilon = \frac{1}{2}\abs{A}.
	\end{equation*}
	
\end{yyProof}

\begin{yyEx}
	求下列函数的导数.
	\begin{enumerate}
		\item 求函数$f(z) = z^2\mathrm{Im}(z)$在$z=0$处的导数.
		\item 设$f(z) = x^2 + iy^2$, 求$f'(1+i)$.
		\item 设$f(z) = x^3+y^3+ix^2y^2$, 求$\begin{lgathered}f'\left( -\frac{3}{2}+\frac{3}{2}i \right)\end{lgathered}$.
		\item 设$w^3-2zw+e^z = 0$, 求$\begin{lgathered}\frac{\mathrm{d}w}{\mathrm{d}z},\frac{\mathrm{d}^2w}{\mathrm{d}z^2}\end{lgathered}$
	\end{enumerate}
\end{yyEx}

\begin{yyEx}
	试证下列函数在$z$平面上解析, 并分别求出其导数.
	\begin{enumerate}
		\item $f(z) = \cos x\cosh y - \mathrm{i}\sin x\sinh y$;
		\item $f(z) = e^x(x\cos y-y\sin y) + \mathrm{i}e^x(y\cos y+ x\sin y)$.
	\end{enumerate}
\end{yyEx}

\begin{yySolution}
    容易知道这两个函数的实部和虚部都是一阶连续可微的, 因而要证明它们解析必须且只需证明它们都满足Cauchy-Riemann方程.
        \begin{enumerate}
            \item \begin{equation*}
                \begin{split}
                    \frac{\partial u}{\partial x} &= -\sin x\cosh y;~~\frac{\partial u}{\partial y} = \cos x\sinh y;\\
                    \frac{\partial v}{\partial x} &= -\cos x\sinh y;\frac{\partial v}{\partial y} = -\sin x\cosh y.
                \end{split}
                \end{equation*}
                因此, 它满足C-R条件.
                \begin{equation*}
                    f'(z) = \frac{\partial u}{\partial x}+\mathrm{i}\frac{\partial v}{\partial x} = -\sin x\cosh y -\mathrm{i}\cos x\sinh y.
                \end{equation*}
                \item 化简这个函数:
                \begin{equation*}
                    \begin{split}
                        f(z) &= \mathrm{e}^x(x+\mathrm{i}y)(\cos y+\mathrm{i}\sin y)\\
                        &=z\mathrm{e}^z.
                    \end{split}
                \end{equation*}
            注意到它独立于$\overline{z}$, 所以它满足C-R条件.
            直接求导, 得到\begin{equation*}
                f'(z) = (z+1)\mathrm{e}^z.
            \end{equation*}
        \end{enumerate}
    
\end{yySolution}

\begin{yyEx}
	若函数$f(z) = x^2+2xy-y^2 + i(y^2 + axy -x^2)$在复平面内处处解析, 那么实常数$a$的值是多少?
\end{yyEx}

\begin{yySolution}
    替换\begin{equation*}
        x = \frac{z + \overline{z}}{2},~~y = \frac{z-\overline{z}}{2\mathrm{i}}.
    \end{equation*}
    得到\begin{equation*}
        f(z) = \frac{1}{4} \left((a+2-4\mathrm{i}) z^2+(2-a) \overline{z}^2\right).
    \end{equation*}
    由于$f(z)$解析, 所以它与$\overline{z}$独立, 因此, $a = 2$.
\end{yySolution}

\section{习题21-40}

\begin{yyEx}
	如果$f'(z)$在单位圆$\abs{z}<1$内处处为零, 且$f(0) = -1$, 那么在$\abs{z}<1$内, $f(z)\equiv$?
\end{yyEx}

\begin{yyEx}
	设$f(0) = 1,~f'(0) = 1+i$, 求$\begin{lgathered}\lim\limits_{z\to 0}\frac{f(z) - 1}{z}\end{lgathered}$.
\end{yyEx}

\begin{yyEx}
	设$f(z) = u + iv$在区域$D$内是解析的, 且$u+v$是实常数, 证明$f(z)$在$D$内是常数.
\end{yyEx}

\begin{yyProof}
    由于$u+v$是实常数, 所以\begin{equation*}
        \frac{\partial u}{\partial x} + \frac{\partial v}{\partial x} = \frac{\partial u}{\partial y} + \frac{\partial v}{\partial y} = 0.
    \end{equation*}
    根据Cauchy-Riemann条件, 知道这等价于\begin{equation*}
        \begin{dcases}
            &\frac{\partial u}{\partial x} - \frac{\partial u}{\partial y} = 0;\\
            &\frac{\partial u}{\partial y} + \frac{\partial u}{\partial x} = 0.
        \end{dcases}
    \end{equation*}
    上面可看作一个关于$\begin{lgathered}\frac{\partial u}{\partial x},\frac{\partial u}{\partial y}\end{lgathered}$的齐次线性方程组, 由于系数矩阵满秩, 所以没有非零解.
    因此得到\begin{equation*}
        f'(z) = \frac{\partial u}{\partial x} - \mathrm{i}\frac{\partial u}{\partial y} = 0.
    \end{equation*}
    因此,$f(z)$是常数.
\end{yyProof}

\begin{yyEx}
	写出导函数$f'(z) = \frac{\partial u}{\partial x}+i\frac{\partial v}{\partial x}$在区域$D$内解析的充要条件.
\end{yyEx}

\begin{yySolution}
    $u,v$在区域$D$内可微并满足Cauchy-Riemann条件.
\end{yySolution}

\begin{yyEx}
	若解析函数$f(z) = u+iv$的实部$u = x^2 - y^2$, 求$f(z)$.
\end{yyEx}

\begin{yySolution}
    根据$u = x^2 - y^2$得到\begin{equation*}
        \frac{\partial u}{\partial x} = 2x,~~\frac{\partial u}{\partial y} = -2y
    \end{equation*}
    这说明\begin{equation*}
        \mathrm{d}v = \frac{\partial u}{\partial x}\mathrm{d}y - \frac{\partial u}{\partial y}\mathrm{d}x = 2x\mathrm{d}y +2y\mathrm{d}x.
    \end{equation*}
    解得\begin{equation*}
        v = 2xy + C.
    \end{equation*}
    因此, $f(z) = (x^2-y^2)+\mathrm{i}(2xy+C)$.
\end{yySolution}

\begin{yyEx}
	已知$u-v = x^2 - y^2$, 试确定解析函数$f(z) = u + iv$.
\end{yyEx}

\begin{yySolution}
    对$u-v = x^2 - y^2$两边分别对$x$和$y$求偏导得到
    \begin{equation*}
        \begin{dcases}
            &\frac{\partial u}{\partial x}-\frac{\partial v}{\partial x} = 2x, \\
            &\frac{\partial u}{\partial y}-\frac{\partial v}{\partial y} = -2y.
        \end{dcases}
    \end{equation*}
    根据Cauchy-Riemann方程, 可化为\begin{equation*}
        \begin{dcases}
            &\frac{\partial u}{\partial x}+\frac{\partial u}{\partial y} = 2x, \\
            &\frac{\partial u}{\partial y}-\frac{\partial u}{\partial x} = -2y.
        \end{dcases}
    \end{equation*}
    解得\begin{equation*}
        \frac{\partial u}{\partial x} = x+y,~~\frac{\partial u}{\partial y} = x-y.
    \end{equation*}
    这说明\begin{equation*}
        f'(z) = \frac{\partial u}{\partial x}-\mathrm{i}\frac{\partial u}{\partial y} = (x+y)-\mathrm{i}(x-y) = (1-\mathrm{i})z.
    \end{equation*}
    求积分得到\begin{equation*}
        f(z) = \frac{1-\mathrm{i}}{2}z^2 + C.
    \end{equation*}
    注意到当$z = 0$时, $u-v = 0$, 由此得到$C = 0$.
    因此, $f(z) = \frac{1-\mathrm{i}}{2}z^2$.
\end{yySolution}


\begin{yyEx}
	解方程$\sin z + \mathrm{i}\cos z = 4i$.
\end{yyEx}

\begin{yySolution}
	原方程等价于\begin{equation*}
		 4 = \cos z -\mathrm{i}\sin z = \exp(-\mathrm{i}z).
	\end{equation*}
	即\begin{equation*}
		-\mathrm{i}z = \ln 4 + 2k\pi\mathrm{i}, k\in\mathbb{Z}.
	\end{equation*}
	由此解得\begin{equation*}
		z = 2n\pi - \mathrm{i}\ln 4,n\in\mathbb{Z}.
	\end{equation*}
\end{yySolution}

\begin{yyEx}
	设$f(z) = \frac{1}{5}z^5 - (1+\mathrm{i})z$, 求$f'(z) = 0$的所有根.
\end{yyEx}

\begin{yySolution}
	根据\begin{equation*}
		f'(z) = z^4 - (1+\mathrm{i}) = 0
	\end{equation*}
	得到\begin{equation*}
		z = \sqrt[8]{2} \exp(\frac{\pi}{16}+\frac{k\pi}{2}),~k = 0,1,2,3.
	\end{equation*}
\end{yySolution}

\begin{yyEx}
	设$f(z) = u(x,y)+ iv(x,y)$为$z = x+iy$的解析函数, 若记
	\begin{equation*}
		w(z,\overline{z}) = u\left( \frac{z+\overline{z}}{2},\frac{z-\overline{z}}{2i} \right) + iv\left( \frac{z+\overline{z}}{2},\frac{z-\overline{z}}{2i} \right),
	\end{equation*}
	证明:$\frac{\partial w}{\partial \overline{z}} = 0$.
\end{yyEx}

\begin{yyProof}
	根据$z,\overline{z}$与$x,y$的关系, 我们有\begin{equation*}
		\frac{\partial}{\partial z} = \frac{1}{2}\left(\frac{\partial}{\partial x}-\mathrm{i}\frac{\partial}{\partial y}  \right),\frac{\partial}{\partial \overline{z}} = \frac{1}{2}\left(\frac{\partial}{\partial x}+\mathrm{i}\frac{\partial}{\partial y}  \right)
	\end{equation*}
	因此,\begin{equation*}
		\begin{split}
			\frac{\partial w}{\partial \overline{z}} &= \frac{1}{2}\left(\frac{\partial}{\partial x}+\mathrm{i}\frac{\partial}{\partial y}  \right)(u+\mathrm{i}v) \\
			&= \frac{1}{2}\left(\frac{\partial u}{\partial x} - \frac{\partial v}{\partial y}\right) + \frac{\mathrm{i}}{2}\left(\frac{\partial u}{\partial y} + \frac{\partial v}{\partial x}\right).
		\end{split}
	\end{equation*}
	得$\frac{\partial w}{\partial \overline{z}} = 0$等价于\begin{equation*}
		\frac{\partial u}{\partial x} = \frac{\partial v}{\partial y},~~\frac{\partial u}{\partial y} = - \frac{\partial v}{\partial x}
	\end{equation*}
	即$u, v$满足Cauchy-Riemann方程, 证毕.
\end{yyProof}

\begin{yyEx}
	设$
	\begin{lgathered}
	f(z) = \begin{dcases}
	\frac{xy^2(x+iy)}{x^2+y^4}, &z\neq 0,\\
	0,&z=0,
	\end{dcases}
	\end{lgathered}
	$试证$f(z)$在原点满足柯西-黎曼方程, 但却不可导.
\end{yyEx}

\begin{yyEx}
	计算下列积分.
	\begin{enumerate}
		\item 设$c$为沿原点$z = 0$到点$z = 1+i$的直线段, 计算$\begin{lgathered}\int_c2\overline{z}\mathrm{d}z\end{lgathered}$.
		\item 设$c$为正向圆周$\abs{z-4} = 1$,计算$\begin{lgathered}\int_c \frac{z^2-3z+2}{(z-4)^2}\mathrm{d}z\end{lgathered}$.
		\item 设$c$为从原点沿$y^2 = x$至$i+i$的弧段, 求积分$\begin{lgathered}\int_c(x+iy^2)\mathrm{d}z\end{lgathered}$.
		\item 设$c$为不经过点$1$与$-1$的正向简单闭曲线, 计算$\begin{lgathered}\oint_c\frac{z}{(z-1)(z+1)^2}\mathrm{d}z\end{lgathered}$.
		\item 设$c_1:\abs{z} = 1$为负向, $c_2:\abs{z} = 3$为正向, 计算$\begin{lgathered}\oint_{c=c_1+c_2}\frac{\sin z}{z^2}\mathrm{d}z\end{lgathered}$.
		\item 设$c$为正向圆周$\abs{z} = 2$, 计算$\begin{lgathered}\oint_c\frac{\cos z}{(1-z)^2}\mathrm{d}z\end{lgathered}$.
		\item 设$c$为正向圆周$\abs{z} = \frac{1}{2}$, 计算$\begin{lgathered}\oint_c\frac{z^3\cos \frac{1}{z-2}}{(1-z)^2}\mathrm{d}z\end{lgathered}$.
		\item 设$c$为从$0$到$1+\frac{\pi}{2}i$的直线段, 计算积分$\begin{lgathered}\int_cze^z\mathrm{d}z\end{lgathered}$.
		\item 设$c$为正向圆周$x^2+y^2-2x = 0$, 计算$\begin{lgathered}\oint_c\frac{\sin(\frac{\pi}{4}z)}{z^2-1}\mathrm{d}z\end{lgathered}$.
		\item 设$c$为正向圆周$\abs{z-i} = 1$, 计算$\begin{lgathered}\oint_c\frac{z\cos z}{z^2-1}\mathrm{d}z\end{lgathered}$.
		\item 设$c$为正向圆周$\abs{z} = 3$, 计算$\begin{lgathered}\oint_c\frac{z+\overline{z}}{\abs{z}}\mathrm{d}z\end{lgathered}$.
		\item 设$c$为负向圆周$\abs{z} = 4$, 计算$\begin{lgathered}\oint_c\frac{e^z}{(z-\pi i)^5}\mathrm{d}z\end{lgathered}$.
	\end{enumerate}
\end{yyEx}

\begin{yyEx}
	设$\begin{lgathered}f(z) = \oint_{\abs{\xi} = 4}\frac{e^\xi}{\xi - z}\mathrm{d}\xi\end{lgathered}$,其中$\abs{z}\neq 4$, 求$f'(\pi i)$.
\end{yyEx}

\begin{yyEx}
	设$f(z)$在单连通区域$B$内处处解析且不为零, $c$为$B$内任何一条简单闭曲线, 求积分$\begin{lgathered}\oint_c\frac{f''(z)+2f'(z)+f(z)}{f(z)}\mathrm{d}z\end{lgathered}$.
\end{yyEx}

\begin{yyEx}
	设$f(z)$在区域$D$内解析, $c$为$D$内任何一条正向简单闭曲线, 它的内部含于$D$. 如果$f(z)$在$c$上的值为$2$, 那么对$c$内任一点$z_0$, $f(z_0)$的值是什么?
\end{yyEx}

\begin{yyEx}
	设$f(z) = \oint_{\abs{\xi} = 2}\frac{\sin(\frac{\pi}{2}\xi)}{\xi-z}\mathrm{d}\xi$, 其中$\abs{z} \neq 2$, 求$f'(3)$.
\end{yyEx}

\begin{yyEx}
	若函数$u(x,y) = x^3+axy^2$为某一解析函数的虚部, 则常数$a$的值是多少?
\end{yyEx}

\begin{yyEx}
	计算下列积分:
	\begin{enumerate}
		\item $
		\begin{lgathered}
			\oint_{\abs{z} = R}\frac{6z}{(z^2-1)(z+2)}\mathrm{d}z,\text{其中}R>0,R\neq 1\text{且}R\neq 2;
		\end{lgathered}
		$
		\item $\begin{lgathered}
			\oint_{\abs{z} = 2}\frac{\mathrm{d}z}{z^4+2z^2+2}.
		\end{lgathered}$
	\end{enumerate}
\end{yyEx}

\begin{yyEx}
	设$f(z)$在单连通区域$B$内解析,且满足$\abs{1-f(z)}<1~(z\in B)$. 试证:
	\begin{enumerate}
		\item 在$B$内处处有$f(z)\neq 0$;
		\item 对于$B$内任意一条闭曲线$c$, 都有$\begin{lgathered}
		\oint_{c}\frac{f''(z)}{f(z)}\mathrm{d}z = 0.
		\end{lgathered}$
	\end{enumerate}
\end{yyEx}

\begin{yyEx}
	设$f(z)$在圆域$\abs{z-a}<R$内解析,若$\begin{lgathered}
	\max_{\abs{z-a} = r}\abs{f(z)} = M(r)~(0<r<R),
	\end{lgathered}$
	证明\begin{equation*}
		\abs{f^{(n)}(a)}\leqslant \frac{n!M(r)}{r^n}~~(n = 1,2,\cdots).
	\end{equation*}
\end{yyEx}

\begin{proof}
	由Cauchy公式得到:\begin{equation*}
		f^{(n)}(a) = \frac{n!}{2\pi i}\int_{\abs{z-a} = r}\frac{f(w)}{(w-a)^{n+1}}\mathrm{d}w,
	\end{equation*}
	利用绝对值不等式得
	\begin{equation*}
		\begin{split}
				\abs{f^{(n)}(a)} &\leqslant \frac{n!}{2\pi}\int_{\abs{z-a} = r}\abs{\frac{f(w)}{(w-a)^{n+1}}}\abs{\mathrm{d}w}\\
				&\leqslant \frac{n!M(r)}{2\pi r^{n+1}}\int_{\abs{z-a} = r}\mathrm{d}s = \frac{n!M(r)}{r^n}.
		\end{split}
	\end{equation*}
\end{proof}

\begin{yyEx}
	求积分$\begin{lgathered}
	\oint_{\abs{z} = 1}\frac{\mathrm{e}^z}{z}\mathrm{d}z,
	\end{lgathered}$从而证明$\begin{lgathered}
	\int_{0}^\pi  \mathrm{e}^{\cos\theta}\cos(\sin\theta)\mathrm{d}\theta = \pi.
	\end{lgathered}$
\end{yyEx}

\begin{yySolution}
	利用Cauchy公式, 得到
	\begin{equation*}
		\oint_{\abs{z} = 1}\frac{\mathrm{e}^z}{z}\mathrm{d}z = 2\pi\mathrm{i}\mathrm{e}^0 = 2\pi\mathrm{i}.
	\end{equation*}
	在此积分中, 替换$z = \exp(\mathrm{i}\theta)$, 得到
	\begin{equation*}
		2\pi\mathrm{i} = \int_{-\pi}^{\pi}\mathrm{e}^{\mathrm{e}^{\mathrm{i}\theta}}\mathrm{e}^{-\mathrm{i}\theta} \mathrm{i}\mathrm{e}^{\mathrm{i}\theta}\mathrm{d}\theta
	\end{equation*}
	利用Euler公式, 化简得到:\begin{equation*}
		2\pi = \int_{-\pi}^{\pi} \mathrm{e}^{\cos\theta}\left[ \cos(\sin\theta) + \mathrm{i}\sin(\sin\theta) \right] \mathrm{d}\theta.
	\end{equation*}
	根据三角函数的奇偶性, 得到最终的结果\begin{equation*}
		\int_{0}^\pi  \mathrm{e}^{\cos\theta}\cos(\sin\theta)\mathrm{d}\theta = \pi.
	\end{equation*}
\end{yySolution}

\section{习题41-44}

\begin{yyEx}
	设$f(z)$在复平面上处处解析且有界, 对于任意给定的两个复数$a,b$, 试求极限\\ $\begin{lgathered}
	\lim\limits_{R\to +\infty}\oint_{\abs{z} = R}\frac{f(z)}{(z-a)(z-b)}\mathrm{d}z
	\end{lgathered}$并由此推证$f(a) = f(b)$(刘维尔(Liouville)定理).
\end{yyEx}



\begin{yyEx}
	设$f(z)$在$\abs{z}<R~ (R>1)$内解析,且$f(0) = 1,f'(0) = 2$, 试计算积分\begin{equation*}
		\oint_{\abs{z} = 1}(z+1)^2\frac{f(z)}{z^2}\mathrm{d}z,
	\end{equation*}
	并由此得出$\begin{lgathered}
	\int_{0}^{2\pi} \cos^2\frac{\theta}{2}f(e^{i\theta})  \mathrm{d}\theta
	\end{lgathered}$之值.
\end{yyEx}

\begin{yyEx}
	设$f(z) = u+iv$是$z$的解析函数, 证明\begin{equation*}
		\frac{\partial^2\ln(1+\abs{f(z)}^2)}{\partial x^2} + \frac{\partial^2\ln(1+\abs{f(z)}^2)}{\partial y^2} = \frac{4\abs{f'(z)}^2}{\left(1+ \abs{f(z)}^2\right)^2}.
	\end{equation*}
\end{yyEx}

\begin{yyEx}
	若$u = u(x^2+y^2)$, 试求解析函数$f(z) = u+iv$.
\end{yyEx}

\section{补充习题}

\begin{yyEx}
    将下面的复数表示为$a+\mathrm{i}b$的形式:\begin{equation*}
        \mathrm{i}^n,~(1+\mathrm{i}\sqrt{3})^n,~(1+\mathrm{i})^n+(1-\mathrm{i})^n.
    \end{equation*}
\end{yyEx}

\begin{yyEx}
    解方程$z^5 = 1-\mathrm{i}$.
\end{yyEx}

\begin{yyEx}
    设$r>0$为实数, $z=x+\mathrm{i}y$为复数, 将复数$r^z$表示为$a+\mathrm{i}b$的形式.
\end{yyEx}

\begin{yyEx}
    证明:
    \begin{equation*}
        \abs{z_1+z_2}^2+\abs{z_1-z_2}^2 = 2(\abs{z_1}^2+\abs{z_2}^2),
    \end{equation*}
    并说明其几何意义.
\end{yyEx}

\begin{yyEx}
    设$z_1,z_2,z_3$都是单位复向量, 证明:$z_1,z_2,z_3$为一正三角形顶点的充分必要条件是$z_1+z_2+z_3 = 0$.
\end{yyEx}

\begin{yyEx}
    证明:
    \begin{enumerate}
        \item  $\abs{1-\overline{z_1}z_2}^2-\abs{z_1-z_2}^2 = (1-\abs{z_1}^2)(1-\abs{z_1}^2)$;
        \item 当$\abs{z_1},\abs{z_2}<1$时, 
        $\begin{lgathered}\left|\frac{z_1-z_2}{1-\overline{z_1}z_2}\right|<1\end{lgathered}$;
        \item 当$\abs{z_1} = 1$或$\abs{z_2} = 1$ 且$z_1\neq z_2$时, $\begin{lgathered}\left|\frac{z_1-z_2}{1-\overline{z_1}z_2}\right|=1\end{lgathered}$.
    \end{enumerate}
\end{yyEx}

\begin{yyEx}
    用复变量表示过点$(1,3),(-1,4)$的直线的方程.
\end{yyEx}

\begin{yyEx}
    \begin{enumerate}
        \item 设$A,C\in\mathbb{R},B\in\mathbb{C}$, 问方程$Az\overline{z}+B\overline{z}+\overline{B}z+C = 0$在什么条件是圆方程, 求其圆心和半径;
        \item 在上面方程中令$A\to 0$, 求半径和圆心的极限, 并说明其几何意义.
    \end{enumerate}
\end{yyEx}

\begin{yyEx}
    证明$B\overline{z}+\overline{B}z+C= 0$是点$z_1,z_2$连线的垂直平分线的充要条件是$B\overline{z_1}+\overline{B}z_2+C = 0$.
\end{yyEx}

\begin{yyEx}
    设$S\subset \mathbb{C}$为任意集合. 令$S'$为$S$的所有极限点构成的集合, $S'$称为$S$的\myind{导集}. 证明:$S'$是闭集;$\overline{S} = S\cup S'$.
\end{yyEx}

\begin{yyEx}
    设$F\subset\mathbb{C}$为紧集, 证明$F$是有界闭集.
\end{yyEx}

\begin{yyEx}
    对于任意集合$S\subset\mathbb{C}$, 证明$\mathrm{diam}S = \mathrm{diam}\overline{S}$.
\end{yyEx}

\begin{yyEx}
    设$z_0\notin \mathbb{R}$,$\begin{lgathered}\lim\limits_{n\to +\infty} z_n = z_0\end{lgathered}$, 证明: 若适当选取辐角主值, 则下两式成立:
    \begin{equation*}
        \lim\limits_{n\to+\infty}\abs{z_n} = \abs{z_0},~~\lim\limits_{n\to+\infty}\mathrm{arg}z_n = \mathrm{arg}z_0.
    \end{equation*}
\end{yyEx}

\begin{yyEx}
    设$G\subset\mathbb{C}$中为任意开集, 证明: $G$可分解为至多可列个互不相交且连通的开集的并.
\end{yyEx}

\begin{yyEx}
    设$S$是给定的集合. 集合$T\subset S$称为$S$的\myind{相对闭集}, 如果$T$在$S$中的极限都在$T$内;集合$T\subset S$称为$S$的\myind{相对开集}, 如果$S\backslash T$是$S$的相对闭集. 证明:$S$连通的充分必要条件是$S$不能分解为两个非空且不交的相对开集(闭集)的并.
\end{yyEx}

\begin{yyEx}
    设$S$是连通集合, $f(z)$是$S$上的函数. 如果$\forall z_0\in S, \exists r>0$, 使得$f(z)$在$S\cap D(z_0,r)$上为常数, 证明:$f(z)$在$S$上为常数.
\end{yyEx}

\begin{yyEx}
    设$U,V$是$\mathbb{C}$中区域, 映射$f:U\to V$称为\myind{开映射}, 如果$f$将$U$中开集映为$V$中开集; $f$称为\myind{逆紧}的, 如果对$V$中任意紧集$K\subset V$, $f^{-1}(K)$是$U$中的紧集. 证明: 如果$f$是开且逆紧的映射, 则$f(U) = V$.
\end{yyEx}

\begin{yyEx}
    求\begin{equation*}
        (1+\cos\theta+\cos 2\theta+\cdots+\cos n\theta) + \mathrm{i}  (1+\sin\theta+\sin 2\theta+\cdots+\sin n\theta).  
    \end{equation*}
\end{yyEx}

\begin{yyEx}
    设$K$是$\mathbb{C}$中的紧集, $F$为$\mathbb{C}$中的闭集. 定义:
    \begin{equation*}
        \mathrm{dist}(K,F) = \mathrm{inf}\{ \abs{z-w}:z\in K,w\in F \}.
    \end{equation*}
    证明:\begin{enumerate}
        \item 如果$K\cap F=\varnothing$, 则$\mathrm{dist}(K,F)>0$;
        \item 设$D$是开集, $S\subset D$是有界闭集, 则$\mathrm{dist}(S,\partial D)>0$.
    \end{enumerate}
\end{yyEx}

\begin{yyEx}
    设$f(x,y) = x^3+3xy+y$, 求$\begin{lgathered}
    \frac{\partial f}{\partial z},\frac{\partial f}{\partial \overline{z}}
    \end{lgathered}$
\end{yyEx}

\begin{yyEx}
    设$z_0$是集合$S$的极限点, 给出并证明$z\in S,~z\to z_0$时,$f(z)$收敛的Cauchy准则.
\end{yyEx}

\begin{yyEx}
    将映射$(x,y)\mapsto (u(x,y),v(x,y))$表示为复函数$w = u+\mathrm{i}v = f(z) = f(x+\mathrm{i}y)$. 如果$f(x+\mathrm{i}y)$连续可导, 证明: 映射的Jacobi行列式满足
    \begin{equation*}
        \left|\begin{array}{cc} 
            \frac{\partial u}{\partial x} & \frac{\partial u}{\partial y} \\
            \frac{\partial v}{\partial x} & \frac{\partial v}{\partial y}
        \end{array}\right| = \left|\begin{array}{cc} 
            \frac{\partial f}{\partial z} & \frac{\partial \overline{f}}{\partial z} \\
            \frac{\partial f}{\partial \overline{z}} & \frac{\partial \overline{f}}{\partial \overline{z}}
        \end{array}\right|.
    \end{equation*}
\end{yyEx}

\begin{yyEx}
    试用$\begin{lgathered}\frac{\partial}{\partial z},\frac{\partial}{\partial \overline{z}}\end{lgathered}$表示$\begin{lgathered}\frac{\partial^2}{\partial x^2}+\frac{\partial^2}{\partial x\partial y }\end{lgathered}$.
\end{yyEx}

\begin{yyEx}
    设$\mathbb{C}_1,\mathbb{C}_2$分别是扩充复平面$S$的两个坐标平面, $\overline{B}z+B\overline{z}+C = 0$是平面$\mathbb{C}_1$中的直线,问其在平面$\mathbb{C}_2$上是什么样的曲线.
\end{yyEx}

\begin{yyEx}
    假设条件如上题所示, 设$Az\overline{z}+\overline{B}z+B\overline{z}+C = 0$是平面$\mathbb{C}_1$中的圆, 问在平面$\mathbb{C}_2$上其是什么曲线.
\end{yyEx}

\begin{yyEx}
    设$f$是$\overline{\mathbb{C}}$上$C^{\infty}$的函数.
    对坐标变换$z=1/w$, 证明\begin{equation*}
        \mathrm{d}f = \frac{\partial f}{\partial z}\mathrm{d}z + \frac{\partial f}{\partial \overline{z}}\mathrm{d}\overline{z} = \frac{\partial f}{\partial w}\mathrm{d}w + \frac{\partial f}{\partial \overline{w}}\mathrm{d}\overline{w},
    \end{equation*}
    即: 微分$\mathrm{d}f$与坐标无关.
\end{yyEx}

\begin{yyEx}
    将$f=z\overline{z}$定义到$\overline{\mathbb{C}}$上, 问$f$在$z=\infty$处是否可导?
\end{yyEx}

\begin{yyEx}
    在$\overline{\mathbb{C}}$上定义:
    \begin{equation*}
        \mathrm{d}s^2 = \frac{4\abs{\mathrm{d}z}^2}{(1+\abs{z}^2)^2}.
    \end{equation*}
    称$\mathrm{d}s^2$为\myind{球度量}. 证明:对坐标变换$z = 1/w$, 有
    \begin{equation*}
        \frac{4\abs{\mathrm{d}z}^2}{(1+\abs{z}^2)^2} = \frac{4\abs{\mathrm{d}w}^2}{(1+\abs{w}^2)^2}.
    \end{equation*}
\end{yyEx}

\begin{yyEx}
    如果$f(z)$是$\mathbb{C}$上的解析函数, 证明$\overline{f(\overline{z})}$也在$\mathbb{C}$上解析.
\end{yyEx}

\begin{yyEx}
    如果$f(z)$和$g(z)$都是$\mathbb{C}$上的解析函数, 证明$f[g(z)]$解析.
\end{yyEx}

\begin{yyEx}
    证明:\begin{enumerate}
        \item  如果$f(z),\overline{f(z)}$都解析, 则$f(z)$为常数;
        \item 如果$f(z)$解析, 且$\abs{f(z)}$为常数, 则$f(z)$为常数.
    \end{enumerate}
\end{yyEx}

\begin{yyEx}
    设$f(z) = u+\mathrm{i}v$解析, 且$u =\sin v$, 证明$f(z)$为常数.
\end{yyEx}

\begin{yyEx}
    设$f(z) = u(x,y) + \mathrm{i}v(x,y)$解析, 且$f'(z)\neq 0$, 证明: 曲线$u(x,y) = c_1$与$v(x,y) = c_2$正交, 其中$c_1,c_2$为常数.
\end{yyEx}

\begin{yyEx}
    \begin{enumerate}
        \item  设$u(x,y) = ax^2+2bxy+cy^2$, 问$u(x,y)$在什么条件下是一解析函数的实部? 如果是, 求$v(x,y)$使$f(z) = u(x,y) +\mathrm{i}v(x,y)$解析.
        \item 问向量函数$F:(x,y)\mapsto (x^2+y^2,xy)$是不是解析映射? 如果不是, 找一个映射$G:(x,y)\mapsto (u(x,y),v(x,y))$, 使得$F+G$是解析映射.
    \end{enumerate}
\end{yyEx}

\begin{yyEx}
    设$f(z)$是$C^{\infty}$的函数, 证明: 如果$f(z)$解析, 则对于任意$k\in\mathbb{N},f^{(k)}(z)$也解析.
\end{yyEx}

\begin{yyEx}
    设$f(z)$解析并有连续导函数, 且$f'(z_0)\neq 0$, 证明: 存在$z_0$的邻域$U$和$f(z_0)$的邻域$V$, 使得$f:U\to V$是一一映射. 用$C-R$方程证明$f^{-1}:V\to U$也解析.
\end{yyEx}

\begin{yyEx}
    设$f:(x,y)\mapsto (u(x,y),v(x,y))$是区域$\Omega_1$到$\Omega_2$的$C^{\infty}$同胚. 称$f$是\myind{保面积}的, 如果对$\Omega_1$内的任意以光滑曲线为边界的有界开集$O$,$O$的面积与$f(O)$的面积都相等. 证明: 如果$f$是保面积的, 且函数$f(z) = u(x,y)+\mathrm{i}v(x,y)$解析, 则$f(z) = \mathrm{e}^{\mathrm{i}\theta}z+c$, 其中$\theta$为常数.
\end{yyEx}

\begin{yyEx}
    设$z = x+\mathrm{i}y$, 直接定义$\mathrm{e}^z = \mathrm{e}^x(\cos y+\mathrm{i}\sin y)$, 证明:
    \begin{enumerate}
        \item $\mathrm{e}^z$在$\mathbb{C}$上解析, 且$(\mathrm{e}^z)' = \mathrm{e}^z$.
        \item $\mathrm{e}^{z_1+z_2} = \mathrm{e}^{z_1}\cdot\mathrm{e}^{z_2}$.
    \end{enumerate}
\end{yyEx}

\begin{yyEx}
    定义\begin{equation*}
        \cos z = \frac{\mathrm{e}^{\mathrm{i}z}+\mathrm{e}^{-\mathrm{i}z}}{2},~~\sin z = \frac{\mathrm{e}^{\mathrm{i}z}-\mathrm{e}^{-\mathrm{i}z}}{2\mathrm{i}}.
    \end{equation*}
    证明$\sin z$和$\cos z$的和角公式.
\end{yyEx}

\begin{yyEx}
    设\begin{equation*}
        f(z) = \sum_{n = 0}^{+\infty}a_n(z-z_0)^n
    \end{equation*}是$z_0$的邻域上的幂级数, $f(z)$不为常数.
    \begin{enumerate}
        \item 证明: 存在正整数$m$, 使得在$z_0$的一个邻域上$f(z)$可表示为
        \begin{equation*}
            f(z) = f(z_0) +(z-z_0)^mg(z)
        \end{equation*}
        其中$g(z)$解析, 且处处不为零;
        \item 证明: 存在$z_0$的邻域, 使得$f(z)$在此邻域上可表示为\begin{equation*}
            f(z) = f(z_0) + [(z-z_0)h(z)]^m,
        \end{equation*}
        其中$h(z)$解析且处处不为零;
        \item 设$\Omega$为区域, $f(z)$在$\Omega$上解析, 并且$\forall z_0\in\Omega$, 存在$z_0$的邻域, 使得$f(z)$可展开为$(z-z_0)$的幂级数, 证明: 如果$f(z)$不是常数, 则$f(z)$将$\Omega$中开集映为开集.
    \end{enumerate}
\end{yyEx}

\begin{yyEx}
    设$f(z) = \sum_{n=0}^{+\infty}a_nz^n$是收敛半径为$R$的幂级数. 设\begin{equation*}
        \left\{ g_m(z) = \sum_{n=0}^{+\infty}b_{mn}z^n \right\}
    \end{equation*}
    是一列幂级数, 满足$\forall m,\abs{b_{mn}}\leqslant \abs{a_n}$, 并且对于任意$n$, $\lim\limits_{m\to+\infty}b_{mn} = b_n$存在. 证明幂级数
    $g_m(z) = \sum_{n=0}^{+\infty}b_{mn}z^n$和$g(z) = \sum_{n=0}^{+\infty}b_nz^n$的收敛半径都大于等于$R$, 且对任意$0<r<R$, $g_m(z)$在$U(0,r)$上一致收敛于$g(z)$.
\end{yyEx}

\begin{yyEx}
    设$f(z) = \sum_{n = 0}^{+\infty}n^2z^n$,$g(z) = \sum_{n=1}^{+\infty}(n^2+1)z^n$.
    \begin{enumerate}
        \item 如果$g/f = a_0+a_1z+a_2z^2+\cdots$, 求$a_0,a_1,a_2$;
        \item 如果$f[g(z)] = a_0+a_1z+a_2z^2+\cdots$,求$a_0,a_1,a_2$.
    \end{enumerate}
\end{yyEx}

\begin{yyEx}
    如果级数$\sum_{n=0}^{+\infty}\abs{a_n}$收敛, 证明$\sum_{n=0}^{+\infty}a_n$收敛, 且其和与求和顺序无关.
\end{yyEx}

\begin{yyEx}
    设级数$\sum_{n=0}^{+\infty}\abs{a_n}$和$\sum_{n=0}^{+\infty}\abs{b_n}$都收敛.
    \begin{enumerate}
        \item 证明级数\begin{equation*}
            \sum_{n = 0}^{+\infty}\left(\sum_{k = 0}^n a_{n-k}b_k \right)
        \end{equation*}收敛, 且其和等于
        \begin{equation*}
            \left(\sum_{n = 0}^{+\infty}a_n\right)\cdot \left(\sum_{n = 0}^{+\infty}b_n\right);
        \end{equation*}
        \item 利用(1)证明如果幂级数$\sum_{n=0}^{+\infty}a_n$和$\sum_{n=0}^{+\infty}b_n$的收敛半径分别为$r_1,r_2$, 则其乘积的收敛半径大于等于$\min\{r_1,r_2\}$.
    \end{enumerate}
\end{yyEx}

\begin{yyEx}
    试构造$\sqrt[n]{z}$的Riemann曲面.
\end{yyEx}

\begin{yyEx}
    设$\Omega$是单位圆盘$U(0,1)$在映射$w = \mathrm{e}^z$下的像, 证明$\Omega$的面积为
    \begin{equation*}
        m(\Omega) = \mathrm{\pi}\sum_{n = 0}^{+\infty}\frac{1}{(n+1)!n!}.
    \end{equation*}
\end{yyEx}

\begin{yyEx}
    试将$(1+\mathrm{i})^{1+\mathrm{i}}$表示为$a+\mathrm{i}b$的形式, 并求其主值.
\end{yyEx}

\begin{yyEx}
    设$f(z)$在$\mathbb{C}$上解析, 并将上半平面映到上半平面, 将实轴映为实轴, 证明在实轴上$f'(z)\geqslant 0$.
\end{yyEx}

\begin{yyEx}
    设$D$是单连通区域, $z_0\notin D$, $f_1(z),f_2(z)$是$\sqrt{z-z_0}$在$D$上的两个不同的解析分支, 证明$f_1(D)\cap f_2(D)=\varnothing$.
\end{yyEx}

\begin{yyEx}
    \begin{enumerate}
        \item 设$z_1,z_2$是单位圆中任意两个互不相等的点, 证明: 存在单位圆到自身的分式线性变换$L(z)$, 使得$L(z_1) = 0$, $L(z_2)>0$. 并问: 这样的分式线性变换是否唯一?
        \item 设$L(z)$为(1)中给定的分式线性变换, 证明:$L^{-1}(z)$将实轴变为过$z_1,z_2$且与单位圆周垂直的圆.
    \end{enumerate}
\end{yyEx}

\begin{yyEx}
    证明: 将实轴(包含$\infty$)变为实轴(包含$\infty$)的分式线性变换可表示为$\begin{lgathered}w = \frac{az+b}{cz+d}\end{lgathered}$的形式, 其中$a,b,c,d$都是实数, 且当$ad-bc>0$时, 其将上半平面变为上半平面;当$ad-bc<0$时, 其将下半平面变为下半平面.
\end{yyEx}

\begin{yyEx}
    设$\begin{lgathered}w = \frac{z-a}{1-\overline{a}z}\end{lgathered}$($\abs{a}<1$), 证明:
    \begin{equation*}
        \frac{\abs{\mathrm{d}w}}{1-\abs{w}^2} = \frac{\abs{\mathrm{d}z}}{1-\abs{z}^2}.
    \end{equation*}
\end{yyEx}

\begin{yyEx}
    在中学中所学过的解析几何中, 我们知道\myind{交比}的概念:
    \begin{equation*}
        L(z) = (z,z_2,z_3,z_4) =  \frac{z-z_3}{z-z_4}:\frac{z_2-z_3}{z_2-z_4}
    \end{equation*}
    可以看到, 它是将$z_3$变为$0$, $z_2$变为$1$, $z_4$变为$\infty$的分式线性变换. 证明: 其将由$z_2,z_3,z_4$决定的圆变为实轴. 并问在什么条件下, 其将圆内部映为上半平面, 圆外部变为下半平面.
\end{yyEx}

\begin{yyEx}
    证明$(w,w_1,w_2,w_3) = (z,z_1,z_2,z_3)$是将$z_j$变为$w_j$($j=1,2,3$)的分式线性变换.
\end{yyEx}

\begin{yyEx}
    给定四个点$z_1,z_2,z_3,Z_4$按顺序位于圆周$K$上, 证明其交比$(z_1,z_2,z_3,z_4)>0$.
\end{yyEx}

\begin{yyEx}
    设$f(z) = u(x,y)+\mathrm{i}v(x,y)$,$u(x,y)$和$v(x,y)$都在$z_0=x_0+\mathrm{i}y_0$处可微. 如果
    \begin{equation*}
        \lim_{z\to z_0}\abs{\frac{f(z)-f(z_0)}{z-z_0}}
    \end{equation*}
    存在, 证明$f(z)$或$\overline{f(z)}$在$z_0$处可导.
\end{yyEx}

\begin{yyEx}
    设$a_1,a_2,a_3,a_4$两两不等, 求
    \begin{equation*}
        \sqrt{(z-a_1)(z-a_2)(z-a_3)(z-a_4)}
    \end{equation*}
    单值解析分支存在的最大区域.
\end{yyEx}

\begin{yyEx}
    利用极坐标$z = r(\cos\theta+\mathrm{i}\sin\theta)$, 证明Cauchy-Riemann方程可表示为
    \begin{equation*}
        u_r = \frac{1}{r}v_\theta,~    v_r = -\frac{1}{r}u_\theta,
    \end{equation*}
    其中$f(z) = u(r,\theta)+\mathrm{i}v(r,\theta)$, 并在极坐标下试求$f'(z)$.
\end{yyEx}

\begin{yyEx}
    如果$w =f(z)$是区域$\Omega$上的解析函数, $\gamma:t\mapsto z(t),t\in[0,b]$是$\Omega$中一光滑曲线, $f(z)$在$\gamma$上处处不为零. 令$\Gamma$是由$t\mapsto f[z(t)] =w,t\in[0,b]$定义的曲线, 证明:
    \begin{equation*}
        \int_{\gamma}\frac{f'(z)}{f(z)}\mathrm{d}z = \int_{\Gamma}\frac{\mathrm{d}w}{w}.
    \end{equation*}
\end{yyEx}

\begin{yyEx}
    设$\gamma$是一不过原点的闭曲线, 根据
    \begin{equation*}
        \int_{\gamma}\frac{\mathrm{d}w}{w} = \int_{\gamma}\mathrm{d}\mathrm{Ln}w = \int_{\gamma}\mathrm{d}\left(
        \ln\abs{w}+\mathrm{i}\mathrm{Arg}w
        \right),
    \end{equation*}
    试证明:
    \begin{equation*}
        \int_{\gamma}\frac{\mathrm{d}w}{w} = \mathrm{i}\int_{\gamma}\mathrm{d}\mathrm{Arg}w.
    \end{equation*}
    并问, 其在什么条件下为零?
\end{yyEx}

\begin{yyEx}
    设$\gamma$是$\mathbb{C}$中一有界的光滑曲线, $\overline{\gamma} =  \gamma$, $\phi(z)$是$\gamma$上的连续函数, 利用导数定义证明\begin{equation*}
        f(z) = \frac{1}{2\mathrm{\pi}\mathrm{i}}\int_{\gamma}\frac{\phi(w)}{w-z}\mathrm{d}w
    \end{equation*}
    在$\mathbb{C}\backslash\gamma$上解析.
\end{yyEx}

\begin{yyEx}
    设$f(z)$是$z_0$邻域上的函数, 且在$z_0$点连续, 证明
    \begin{equation*}
        \lim_{\varepsilon\to 0}\frac{1}{2\mathrm{\pi}\mathrm{i}}\int_{\abs{w-z_0} = \varepsilon}\frac{f(w)}{w-z_0}\mathrm{d}w = f(z_0).
    \end{equation*}
\end{yyEx}

\begin{yyEx}
    设$f(z)$在区域$\Omega$上解析, $z_0\in\Omega$, 证明:
    \begin{enumerate}
        \item $f(z)$在$z_0$邻域上可展开为$(z-z_0)$的幂级数\begin{equation*}
            f(z) = \sum_{n=0}^{+\infty}\frac{f^{(n)}(z_0)}{n!}(z-z_0)^n;
        \end{equation*}
        \item 此幂级数的收敛半径大于等于$\mathrm{dist}(z_0,\partial\Omega)$;
        \item 如果$f(x)$为$(x_0-r,x_0+r)$上的实函数, 在$x_0$处展开的Taylor级数收敛于$f(x)$,且这一级数的收敛半径$R>r$, 证明对于任意$x'\in(x_0-r,x_0+r)$,$f(x)$在$x'$处展开的Taylor级数收敛于$f(x)$.
    \end{enumerate}
\end{yyEx}

\begin{yyEx}
    计算:\begin{enumerate}
        \item \begin{equation*}
            \int_{\abs{z} = 2}\frac{\mathrm{d}z}{z^2+1};
        \end{equation*}
        \item \begin{equation*}
            \int_{\abs{z+\mathrm{i}}=1}\frac{\mathrm{e}^z}{1+z^2}\mathrm{d}z;
        \end{equation*}
        \item \begin{equation*}
            \int_{\abs{z}=2}\frac{\abs{\mathrm{d}z}}{z-1}
        \end{equation*}
        \item \begin{equation*}
            \int_{\abs{z} = 1}\overline{z}\mathrm{d}z.
        \end{equation*}
    \end{enumerate}
\end{yyEx}

\begin{yyEx}
    计算:\begin{enumerate}
        \item \begin{equation*}
            \int_{\abs{z} = 2}\frac{\mathrm{d}z}{z^3(z+3)^2};
        \end{equation*}
        \item \begin{equation*}
            \int_{\abs{z}=R}\frac{\mathrm{d}z}{(z-a)^n(z-b)},
        \end{equation*}
        其中$a,b$不在圆周$\abs{z} = R$上.
    \end{enumerate}
\end{yyEx}

\begin{yyEx}
    设$f(z)$在$U_0(z_0,R) = \{z:0<\abs{z-z_0}<R \}$上解析, 证明存在常数$c$, 使\begin{equation*}
        f(z)-\frac{c}{z-z_0}
    \end{equation*}
    在$U_0(z_0,R)$上有原函数.
\end{yyEx}

\begin{yyEx}
    设$f(z)$是区域$\Omega$上的连续函数. 如果$\gamma$是$\Omega$中的一段圆弧, $f(z)$在$\Omega\backslash\gamma$上解析, 证明$f(z)$在$\Omega$上解析.
\end{yyEx}

\begin{yyEx}
    $\{f_n(z)\}$是区域$\Omega$上的解析函数列, 且在$\Omega$上内闭一致收敛于$f(z)$, 证明:
    \begin{enumerate}
        \item $f(z)$在$\Omega$上解析;
        \item $\{f^{(k)}(z)\}$在$\Omega$上也内闭一致收敛于$f^{(k)}(z)$;
        \item 设$\{f_n(x)\}$是$[-1,1]$上连续可导的函数列, $\{f_n(0)\}$收敛, 且$\{f^{(n)}(x)\}$在$[-1,1]$上一致收敛, 则$\{f_n(x)\}$在$[-1,1]$上一致收敛, 并且
        \begin{equation*}
            [\lim_{n\to\infty}f_n(x)]' = \lim_{n\to\infty}f_n'(x).
        \end{equation*}
        请比较(2)和(3)中证明的不同之处.
    \end{enumerate}
\end{yyEx}

\begin{yyEx}
    设$f(z)$在$\Omega$上解析, 证明: 不存在$z_0\in\Omega$, 使得对于任意$n\in\mathbb{N}$,\begin{equation*}
        \abs{f^{(n)}(z_0)}\geqslant n!n^n.
    \end{equation*}
\end{yyEx}

\begin{yyEx}
    利用平均值定理证明最大模原理.
\end{yyEx}

\begin{yyEx}\begin{enumerate}
    \item 
    设$f(z),g(z)$都在$z_0$的邻域上解析, $g(z_0)\neq 0$, 讨论$f(z),g(z)$在$z_0$展开的幂级数相除后的收敛半径;
    \item 设$f(z)$在$z_0$的邻域解析, $z = g(w)$在$w_0$的邻域上解析, $z_0 = g(w_0)$, 讨论$f(z)$在$z_0$展开的幂级数复合$g(w)$在$w_0$展开的幂级数后所得幂级数的收敛半径.
\end{enumerate}\end{yyEx}

\begin{yyEx}
    设$f(z)$在$\Omega$上解析并且有无穷多个零点, $f(z)$不恒为零, 证明可将$f(z)$的零点排成一列$\{z_n\}$, 且$\{z_n\}$在$\Omega$内无聚点.
\end{yyEx}

\begin{yyEx}
    设$f(z)$在$\mathbb{C}$上解析, 且存在$n$, 使
    \begin{equation*}
        \lim_{z\to\infty}f(z)/z^n =M,
    \end{equation*}
    证明$f(z)$为阶数小于等于$n$的多项式.
\end{yyEx}

\begin{yyEx}
    设$f(z)$在区域$\Omega$上解析且不为多项式, 证明: 存在$z_0\in\Omega$, 使得对于任意$n$, 恒有$f^{(n)}(z_0)\neq 0$.
\end{yyEx}

\begin{yyEx}
    如果$u(x,y)$是$\mathbb{R}^2$上非负的调和函数, 证明$u(x,y)$为常数.
\end{yyEx}

\begin{yyEx}
    设$f(z)$在$\overline{U(0,1)}$的邻域上解析. 令$\gamma = f(\partial U(0,1))$, 证明:$\gamma$的弧长$L\geqslant 2\mathrm{\pi}\abs{f'(0)}$.
\end{yyEx}

\begin{yyEx}
    设$f(z)$将$U(0,1)$单叶地映为$\Omega$, 证明:$\Omega$的面积$m(\Omega)\geqslant \mathrm{\pi}\abs{f'(0)}^2$.
\end{yyEx}

\begin{yyEx}
    设非常数的函数$f(z)$在$1<\abs{z}<\infty$上解析, 且\begin{equation*}
        \lim_{z\to\infty}f(z)    \xlongequal{\text{记为}}f(\infty)
    \end{equation*}存在, 证明:
    \begin{enumerate}
        \item \begin{equation*}
            f(\infty) = \frac{1}{2\mathrm{\pi}}\int_{\abs{z}=R}f(R\mathrm{e}^{\mathrm{i}\theta})\mathrm{d}\theta,R>1;
        \end{equation*}
        \item 在区域$\abs{z}>1$上最大模原理对$f(z)$成立.
    \end{enumerate}
\end{yyEx}

\begin{yyEx}
    设$f(z)$在区域$\Omega$上解析, $z_0\in\Omega$, $f(z) = \sum_{n=0}^{+\infty}a_n(z-z_0)^n$是$f(z)$在$z_0$的幂级数展开, 证明:
    \begin{enumerate}
        \item 当$r>0$充分小时,\begin{equation*}
            \frac{1}{2\mathrm{\pi}}\int_{0}^{2\mathrm{\pi}}\abs{f(z_0+r\mathrm{e}^{\mathrm{i}\theta})}\mathrm{d}\theta = \sum_{n = 0}^{+\infty}\abs{a_n}^2r^{2n}.
        \end{equation*}
        \item 利用(1)证明解析函数的最大模原理;
        \item 设$f_1(z),f_2(z),\cdots,f_n(z)$都是区域$\Omega$上的解析函数. 定义$\Omega$上解析的向量函数$\bm{F}(z) = (f_1(z),\cdots,f_n(z))$, 并定义$\bm{F}(z)$在$z$的模为
        \begin{equation*}
            \norm{\bm{F}(z)} = \sqrt{\abs{f_1(z)}^2+\cdots+\abs{f_n(z)}^2}.
        \end{equation*}
        如果$\bm{F}(z)$不为常值, 则$\norm{\bm{F}(z)}$在$\Omega$内没有最大值.
    \end{enumerate}
\end{yyEx}

\begin{yyEx}
    设$f(z)$在$U(0,1)$上解析, 证明: 存在序列$\{z_n\}\subset U(0,1)$, 使得其同时满足:
    \begin{enumerate}
        \item $\lim\limits_{n\to\infty}\abs{z_n} = 1$;
        \item $\lim\limits_{n\to\infty}f(z_n)$存在.
    \end{enumerate}
\end{yyEx}

\begin{yyEx}
    若$P(z)$为$n$次多项式, 且当$\abs{z}\leqslant 1$时, $\abs{P(z)}\leqslant M$, 证明:当$R>1, \abs{z}\leqslant R$时, $\abs{P(z)}\leqslant MR^n$. 
\end{yyEx}

\begin{yyEx}
    设$D$是以有限条逐段光滑曲线为边界的有界区域, $\{ f_n(z)\}$是$D$上的解析函数列, 满足$\forall\varepsilon>0,\exists N\in\mathbb{N}$, 使得只要$n>N,m>N$就有\begin{equation*}
        \iint_{D}\abs{f_n(z)-f_m(z)}\mathrm{d}x\mathrm{d}y<\varepsilon,
    \end{equation*}
    证明:$\{f_n(z)\}$在$D$上内闭一致收敛.
\end{yyEx}

\begin{yyEx}
    证明: 如果$f(z)$在$\mathbb{C}$上解析, 且平方可积, 则$f(z)\equiv 0$.
\end{yyEx}

\begin{yyEx}
    设$z_1\neq z_2$和$w_1\neq w_2$是上半平面任意给定的两组点, 问是否存在上半平面的解析自同胚$L$, 使得$L(z_1) = w_1,~L(z_2) = w_2$?
\end{yyEx}

\begin{yyEx}
    设$f(z)$在$U(0,1)$内解析, $\mathrm{Re}f(z)\geqslant 0,f(0) = a>0$, 证明
    \begin{equation*}
        \abs{\frac{f(z)-a}{f(z)+a}}\leqslant \abs{z},~\abs{f'(0)}\leqslant 2a.
    \end{equation*}
    若$a = 1$, 则\begin{equation*}
        \frac{1-\abs{z}}{1+\abs{z}}\leqslant \abs{f(z)} \leqslant \frac{1+\abs{z}}{1-\abs{z}}.
    \end{equation*}
\end{yyEx}

\begin{yyEx}
    如果$f(z)$是上半平面到自身的解析同胚, 证明
    \begin{equation*}
        f(z) = \frac{az+b}{cz+d},
    \end{equation*}
    其中$a,b,c,d\in\mathbb{R},ad-bc>0$.
\end{yyEx}

\begin{yyEx}
    
\end{yyEx}

\chapter{复变函数的幂级数}

\section{习题1-5}

\begin{yyEx}
	选择题
	\begin{enumerate}
		\item 设$\begin{lgathered}
			a_n = \frac{(-1)^n + n\mathrm{i}}{n+4}(n = 1,2,\cdots)
		\end{lgathered}$,则$\begin{lgathered}
			\lim\limits_{n\to+\infty}a_n(\quad\quad).
		\end{lgathered}$\\
		(A) 等于$0$~~ (B)等于$1$~~ (C)等于$\mathrm{i}$~~ (D)不存在	
		\item 下列级数中, 条件收敛的级数为$(\quad\quad)$.\\
		(A) $\begin{lgathered}
		\sum_{n=1}^{+\infty}\left( \frac{1+3\mathrm{i}}{2} \right)^n
		\end{lgathered}$~~(B)$\begin{lgathered}
		\sum_{n=1}^{+\infty} \frac{(3+4\mathrm{i})^n}{n!}
		\end{lgathered}$~~(C)$\begin{lgathered}
		\sum_{n=1}^{+\infty} \frac{\mathrm{i}^n}{n}
		\end{lgathered}$~~(D)$\begin{lgathered}
		\sum_{n=1}^{+\infty} \frac{(-1)^n+\mathrm{i}}{\sqrt{n+1}}
		\end{lgathered}$		
		\item 下列级数中, 绝对收敛的级数为$(\quad\quad)$.\\
		(A) $\begin{lgathered}
		\sum_{n=1}^{+\infty}\frac{1}{n}\left( 1+\frac{\mathrm{i}}{n} \right)
		\end{lgathered}$~~(B)$\begin{lgathered}
		\sum_{n=1}^{+\infty} \left[ \frac{(-1)^n}{n} + \frac{\mathrm{i}}{2^n} \right]
		\end{lgathered}$~~(C)$\begin{lgathered}
		\sum_{n=2}^{+\infty} \frac{\mathrm{i}^n}{\ln n}
		\end{lgathered}$~~(D)$\begin{lgathered}
		\sum_{n=1}^{+\infty} \frac{(-1)^n\mathrm{i}^n}{2^n}
		\end{lgathered}$		
		\item 若幂级数$\begin{lgathered}
			\sum_{n=0}^{+\infty}c_nz^n
		\end{lgathered}$在$z = 1+2\mathrm{i}$处收敛, 那么该级数在$z = 2$处的收敛性为$(\quad\quad)$.\\		
		(A)~绝对收敛~~(B)~条件收敛~~(C)~发散~~(D)~不能确定		
		\item 设幂级数$\begin{lgathered}
		\sum_{n=0}^{+\infty}c_nz^n,\sum_{n=0}^{+\infty}nc_nz^{n-1}
		\end{lgathered}$和$\begin{lgathered}
		\sum_{n=0}^{+\infty}\frac{c_n}{n+1}z^{n+1}
		\end{lgathered}$ 的收敛半径分别为$R_1,R_2,R_3$, 则$R_1,R_2,R_3$之间的关系是$(\quad\quad)$.\\
		(A)~$R_1<R_2<R_3$~~(B)~$R_1>R_2>R_3$~~(C)~$R_1=R_2<R_3$~~(D)~$R_1=R_2=R_3$
		\item 设$0<\abs{q}<1$, 则幂级数$\begin{lgathered}
		\sum_{n=0}^{+\infty}q^{n^2}z^n
		\end{lgathered}$ 的收敛半径$R = (\quad\quad)$\\
		(A)~$\abs{q}$~~(B)~$\frac{1}{\abs{q}}$~~(C)~$0$~~(D)~$+\infty$
	\end{enumerate}
\end{yyEx}

\begin{yyEx}
	求下列幂级数的收敛半径
	\begin{enumerate}
		\item $\begin{lgathered}
			\sum_{k=0}^{+\infty}\frac{k!}{k^k}z^k
		\end{lgathered}$;
		\item $\begin{lgathered}
		\sum_{k=1}^{+\infty}k^nz^k
		\end{lgathered}$;
		\item $\begin{lgathered}
		\sum_{n=1}^{+\infty}\frac{\sin\frac{n\pi}{2}}{n}\left( \frac{z}{2} \right)^n
		\end{lgathered}$;
		\item $\begin{lgathered}
		\sum_{n = 0}^{+\infty}(2\mathrm{i})^nz^{2n+1}
		\end{lgathered}$;
		\item $\begin{lgathered}
		\sum_{n = 1}^{+\infty}\frac{(n!)^2}{n^n}z^n
		\end{lgathered}$;
		\item $\begin{lgathered}
		\sum_{n = 0}^{+\infty}(1+\mathrm{i})^nz^{n}
		\end{lgathered}$;
		\item 设函数$\begin{lgathered}
			\frac{e^z}{\cos z}
		\end{lgathered}$的泰勒展开式为$\begin{lgathered}
			\sum_{n=0}^{+\infty}c_nz^n
		\end{lgathered}$, 求幂级数$\begin{lgathered}
			\sum_{n=0}^{+\infty}c_nz^n	
		\end{lgathered}$的收敛半径$R$;
		\item 设幂级数为$\begin{lgathered}
		\sum_{n=0}^{+\infty}c_nz^n
		\end{lgathered}$的收敛半径为$R$, 求幂级数为$\begin{lgathered}
		\sum_{n=0}^{+\infty}(2^n-1)c_nz^n
		\end{lgathered}$的收敛半径;
		\item 已知级数$\begin{lgathered}
			\sum_{k=0}^{+\infty}a_kz^k
		\end{lgathered}$和$\begin{lgathered}
		\sum_{k=0}^{+\infty}b_kz^k
		\end{lgathered}$的收敛半径分别为$R_1$和$R_2$, 试确定下列级数的收敛半径\\
			$\begin{lgathered}
			\bullet\sum_{k=0}^{+\infty}a_k^nz^k
			\end{lgathered}$~~~~
			$\begin{lgathered}
			\bullet\sum_{k=0}^{+\infty}\frac{1}{a_k}z^k
			\end{lgathered}$~~~~
			$\begin{lgathered}
			\bullet\sum_{k=0}^{+\infty}a_kb_kz^k
			\end{lgathered}$~~~~
			$\begin{lgathered}
			\bullet\sum_{k=0}^{+\infty}\frac{b_k}{a_k}z^k
			\end{lgathered}$
	\end{enumerate}
\end{yyEx}

\begin{yyEx}
	求下列函数在指定点处的泰勒展开式.
	\begin{enumerate}
		\item $\arctan z$在$z = 0$处;
		\item $f(z) = \cos^2z$在$z = 0$处;
		\item $\begin{lgathered}
			\frac{1}{z^2}
		\end{lgathered}$在$z = -1$处;
		\item $\sin z$在$z = \pi/2$处;
		\item $\begin{lgathered}
		\frac{\sin z}{1+z^2}
		\end{lgathered}$在$\abs{z}<1$内点$z_0 = 0$处;
		\item $\begin{lgathered}
			\mathrm{e}^\frac{1}{1-z}
		\end{lgathered}$在$\abs{z}<1$内点$z_0 = 0$处.
	\end{enumerate}
\end{yyEx}

\begin{yyEx}
	求级数$\begin{lgathered}
		\sum_{k=0}^{+\infty}z^k,\sum_{k=1}^{+\infty}\frac{z^k}{k},\sum_{k=1}^{+\infty}\frac{z^k}{k^2}
	\end{lgathered}$的收敛半径, 并讨论它们在收敛圆周上的收敛性.
\end{yyEx}

\begin{yyEx}
	求下列函数在$z_0$处的泰勒展开式和收敛半径.
	\begin{enumerate}
		\item $\begin{lgathered}
			\frac{z}{(z+1)(z+2)},z_0 = 2;
		\end{lgathered}$
		\item $\begin{lgathered}
			\frac{1}{4-3z},z_0 = 1+\mathrm{i}.
		\end{lgathered}$
	\end{enumerate}
\end{yyEx}

\section{习题6-10}

\begin{yyEx}
	求和函数.\begin{enumerate}
		\item 求和函数$f(z) = 1+2z+3z^2+4z^3+\cdots,~\abs{z}<1$;
		\item 求和函数$\begin{lgathered}
			f(z) = \frac{1}{1\cdot 2}+\frac{z}{2\cdot 3}+\frac{z^2}{3\cdot 4}+\frac{z^3}{4\cdot 5}+\cdots,~\abs{z}<1
		\end{lgathered}$;
		\item 求幂级数$\begin{lgathered}
			\sum_{n = 0}^{+\infty}\frac{(-1)^n}{n+1}z^{n+1}
		\end{lgathered}$在$\abs{z}<1$内的和函数.
	\end{enumerate}
\end{yyEx}

\begin{yyEx}
	若函数$\begin{lgathered}
		\frac{1}{1-z-z^2}
	\end{lgathered}$在$z = 0$处的泰勒展开式为$\begin{lgathered}
		\sum_{n = 0}^{+\infty}a_nz^n
	\end{lgathered}$, 则称$\{a_n\}$为斐波那契(Fibonacci)数列, 试确定$a_n$满足的递推关系式, 并明确给出$a_n$的表达式.
\end{yyEx}

\begin{yyEx}
	求下列级数的洛朗级数.
	\begin{enumerate}
		\item 求函数$\begin{lgathered}
			\frac{1}{z(z-\mathrm{i})}
		\end{lgathered}$在$1<\abs{z-\mathrm{i}}<+\infty$内的洛朗展开式.
		\item 求复变函数$\begin{lgathered}
			\mathrm{e}^{\frac{1}{1-z}}
		\end{lgathered}$在孤立奇点$z = 1$的去心邻域 $0<\abs{z-1}<+\infty$的洛朗展开式.
		\item 求函数$\begin{lgathered}
			\mathrm{e}^z+\mathrm{e}^{\frac{1}{z}}
		\end{lgathered}$在$0<\abs{z}<+\infty$内的洛朗展开式.
		\item 将函数$\begin{lgathered}
		\frac{\ln (2-z)}{z(z-1)}
		\end{lgathered}$在$1<\abs{z-\mathrm{i}}<+\infty$在$0<\abs{z-1}<1$内展开成洛朗级数.
		\item 把下列各函数展开成$z$的幂级数, 并指出它们的收敛半径.\\
		(a) $\begin{lgathered}
			\mathrm{e}^{z^2}\sin z^2
		\end{lgathered}$~~(b)$\begin{lgathered}
		\mathrm{e}^{\frac{z}{z-1}}
		\end{lgathered}$.
	\end{enumerate}
\end{yyEx}

\begin{yyEx}
	把下列各函数在指定的圆环域内展开成洛朗级数.
	\begin{enumerate}
		\item $\begin{lgathered}
			\frac{1}{z(1-z)^2},0<\abs{z}<1,0<\abs{z-1}<1
		\end{lgathered}$;
		\item $\begin{lgathered}
		\frac{z^2-2z+5}{(z-2)(z^2+1)},1<\abs{z}<2;
		\end{lgathered}$
		\item $\begin{lgathered}
		\sin z\cdot \sin\frac{1}{z},0<\abs{z}<+\infty;
		\end{lgathered}$
		\item $\begin{lgathered}
		\frac{1}{z^2-3z+2},1<\abs{z}<2;
		\end{lgathered}$
		\item $\begin{lgathered}
		\sin\frac{z}{z-1},\abs{z-1}>0.
		\end{lgathered}$
	\end{enumerate}
\end{yyEx}

\begin{yyEx}
	求下列洛朗级数的收敛域.
	\begin{enumerate}
		\item 设函数$\cot z$在原点的去心邻域$0<\abs{z}<R$内的洛朗展开式$\begin{lgathered}
			\sum_{n = -\infty}^{+\infty} c_nz^n
		\end{lgathered}$, 求该洛朗级数的收敛域的外半径$R$.
		\item 求级数$\begin{lgathered}
			\frac{1}{z^2} + \frac{1}{z}+1+z+z^2+\cdots
		\end{lgathered}$的收敛域.
		\item 求双边幂级数$\begin{lgathered}
			\sum_{n = 1}^{+\infty}(-1)^n\frac{1}{(z-2)^n}+\sum_{n = 1}^{+\infty}(-1)^n\left( 1-\frac{z}{2} \right)^n
		\end{lgathered}$的收敛域.
		\item 若$\begin{lgathered}
			c_n = \begin{dcases}
				3^n+(-1)^n,&n = 0,1,2,\cdots,\\
				4^n,&n = -1,-2,\cdots,
			\end{dcases}
		\end{lgathered}$求双边幂级数$\begin{lgathered}
			\sum_{n = -\infty}^{+\infty}c_nz^n
		\end{lgathered}$的收敛域.
	\end{enumerate}
\end{yyEx}

\section{习题11-15}

\begin{yyEx}
	判断幂级数$\begin{lgathered}
		\sum_{n = 1}^{+\infty}\frac{\mathrm{i}^n}{n^{\alpha}}(\alpha>0)
	\end{lgathered}$的收敛性与绝对收敛性.
\end{yyEx}

\begin{yyEx}
	设$f(z)$在圆环域$H:R_1<\abs{z-z_0}<R_2$内的洛朗展开式为$\begin{lgathered}
	\sum_{n = -\infty}^{+\infty}c_n(z-z_0)^n
	\end{lgathered}$,$c$为$H$内绕$z_0$的任一条正向简单闭曲线, 求$\begin{lgathered}
		\oint_c\frac{f(z)}{(z-z_0)^2}\mathrm{d}z = ?
	\end{lgathered}$
\end{yyEx}

\begin{yyEx}
	若函数$\begin{lgathered}
		f(z) = \frac{1}{z(z+1)(z+4)}
	\end{lgathered}$在原点处的极点的阶为$m$, 那么$m = ?$
\end{yyEx}

\begin{yyEx}
	设$f(z)$在区域$D$内解析,$z_0$为$D$内的一点,$d$为$z_0$到$D$的边界上各点的最短距离, 那么当$\abs{z-z_0}<d$时, $\begin{lgathered}
		f(z) = \sum_{n = 0}^{+\infty}c_n(z-z_0)^n
	\end{lgathered}$成立, 求$c_n$.
\end{yyEx}

\begin{yyEx}
	试证明:
	\begin{enumerate}
		\item $\begin{lgathered}
			\abs{\mathrm{e}^z-1}\leqslant \mathrm{e}^{\abs{z}}-1\leqslant \abs{z}\mathrm{e}^{\abs{z}}~(\abs{z}<+\infty);
		\end{lgathered}$
		\item $\begin{lgathered}
		(3-\mathrm{e})\abs{z}\leqslant \abs{\mathrm{e}^z-1}\leqslant (\mathrm{e}-1)\abs{z}~(\abs{z}<1).
		\end{lgathered}$
	\end{enumerate}
\end{yyEx}

\section{习题16-20}

\begin{yyEx}
	设函数$f(z)$在圆域$\abs{z}<R$内解析, $\begin{lgathered}
		S_n = \sum_{k=0}^n\frac{f^{(k)}(0)}{k!}z^k
	\end{lgathered}$,试证:
	\begin{enumerate}
		\item $\begin{lgathered}
			S_n(z) = \oint_{\abs{\xi} = r}f(\xi)\frac{\xi^{n+1}-z^{n+1}}{\xi-z}\frac{\mathrm{d}\xi}{\xi^{n+1}}~(\abs{z}<r<R);
		\end{lgathered}$
		\item $\begin{lgathered}
		f(z) - S_n(z) =\frac{z^{n+1}}{2\pi\mathrm{i}} \oint_{\abs{\xi} = r}\frac{f(\xi)}{\xi^{n+1}(\xi-z)}\mathrm{d}\xi~(\abs{z}<r<R).
		\end{lgathered}$
	\end{enumerate}
\end{yyEx}

\begin{yyEx}
	设$\begin{lgathered}
		f(z) = \sum_{n = 0}^{+\infty}a_nz^n~(\abs{z}<R_1),g(z) = \sum_{n = 0}^{+\infty}b_nz^n~(\abs{z}<R_2)
	\end{lgathered}$, 则对任意的$r(0<r<R_1)$, 在$\abs{z}<rR_2$内有
	\begin{equation*}
		\sum_{n = 0}^{+\infty}a_nb_nz^n = \frac{1}{2\pi\mathrm{i}}\oint_{\abs{\xi} = r}f(\xi)g\left( \frac{z}{\xi} \right)\frac{\mathrm{d}\xi}{\xi}.
	\end{equation*}
\end{yyEx}

\begin{yyEx}
	设在$\abs{z}<R$内解析的函数$f(z)$有泰勒展开式\begin{equation*}
		f(z) = a_0 + a_1z+a_2z^2+\cdots + a_nz^n + \cdots,
	\end{equation*}
	试证当$0\leqslant r<R$时,\begin{equation*}
		\frac{1}{2\pi}\abs{f(r\mathrm{e}^{\mathrm{i}\theta})}^2\mathrm{d}\theta = \sum_{n = 0}^{+\infty}\abs{a_n}^2r^{2n}.
	\end{equation*}
\end{yyEx}

\begin{yyEx}
	试证在$0<\abs{z}<+\infty$内下列展开式成立:
	\begin{equation*}
		\mathrm{e}^{z+\frac{1}{z}} = c_0 + \sum_{n=1}^{+\infty}c_n\left( z^n + \frac{1}{z^n} \right),
	\end{equation*}
	其中$\begin{lgathered}
		c_n = \frac{1}{\pi}\int_{0}^{\pi}\mathrm{e}^{2\cos\theta}\cos n\theta\mathrm{d}\theta~(n = 0,1,2,\cdots).
	\end{lgathered}$
\end{yyEx}

\begin{yyEx}
	试证级数$\begin{lgathered}
		\sum_{k=1}^{+\infty}\frac{\sin k\abs{z}}{k(k+1)}
	\end{lgathered}$在全复平面上一致收敛.
\end{yyEx}

\section{习题21-25}

\begin{yyEx}
	如果$C$为正向圆周$\abs{z} = 3$, 求积分$\begin{lgathered}
		\int_Cf(z)\mathrm{d}z
	\end{lgathered}$的值, 设$f(z)$为:
	\begin{enumerate}
		\item $\begin{lgathered}
			\frac{z+2}{z(z+1)};
		\end{lgathered}$
		\item $\begin{lgathered}
		\frac{z}{(z+1)(z+2)}.
		\end{lgathered}$
	\end{enumerate}
\end{yyEx}

\begin{yyEx}
	求出下列函数的奇点(包括无穷远点), 确定它们是哪一类的奇点(对于极点, 要指出它们的级).
	\begin{enumerate}
		\item $\begin{lgathered}
			\frac{z^5}{(1-z)^2};
		\end{lgathered}$
		
		\item $\begin{lgathered}
		\frac{z^2+1}{\mathrm{e}^z};
		\end{lgathered}$
		
		\item $\begin{lgathered}
		\frac{\tan(z-1)}{z-1};
		\end{lgathered}$
		
		\item $\begin{lgathered}
		\frac{z^3+5}{z^4(z+1)};
		\end{lgathered}$
		
		\item $\begin{lgathered}
		z^4\mathrm{e}^{\frac{1}{z}}
		\end{lgathered}$
		
		\item $\begin{lgathered}
		\frac{\cos z}{z^2+1}+6z;
		\end{lgathered}$
		
		\item $\begin{lgathered}
		\frac{\sin(3z)}{z^2} - \frac{3}{z}.
		\end{lgathered}$
	\end{enumerate}
\end{yyEx}

\begin{yyEx}
	试求级数$\begin{lgathered}
		\sum_{n=0}^{+\infty}z^{n^2}
	\end{lgathered}$ 及其逐项求导级数、逐项求积级数的收敛半径, 讨论它们在$z = 1$和$z = i$时级数的收敛性.
\end{yyEx}

\begin{yyEx}
	试证$\begin{lgathered}
		\cosh(z+\frac{1}{z}) = c_0 + \sum_{k=1}^{+\infty}c_k(z^k+z^{-k})
	\end{lgathered}$, \\其中$\begin{lgathered}
		c_k = \frac{1}{2\pi}\int_{0}^{2\pi}\cos k\varphi\cosh(2\cos\varphi)\mathrm{d}\varphi.
	\end{lgathered}$
\end{yyEx}

\begin{yyEx}
	设$f(z),g(z)$分别以$z = b$为$m$级及$n$级极点, 试问$z = b$为$\begin{lgathered}
		\frac{f}{g}
	\end{lgathered}$
	的怎样的点?
\end{yyEx}

\section{习题26}

\begin{yyEx}
	请指出下列级数在零点$z = 0$级.
	\begin{equation*}
		(1)~z^2(e^{z^2} - 1);~~(2)~6\sin z^3+z^3(z^6-6).
	\end{equation*}
\end{yyEx}

\chapter{留数及其应用}

\section{内容提要}
	在前面一章中, 我们主要是利用Laurent级数讨论了有孤立奇点的函数在孤立奇点的邻域上的性质. 本章中, 我们将利用积分表示来研究这些函数. 核心的问题是怎样将Cauchy定理和Cauchy公式推广到这些有孤立奇点的函数上, 并且得到一些结论. 本章中引入了留数这个复分析中的基本概念, 并且它有许多的应用. 应用留数定理可以将沿闭曲线的积分化为计算在孤立奇点处的留数, 还可以计算一些定积分和广义积分. 最后, 还可以推出辐角原理和Rouché定理.

	(1)\myind{留数定义}~~设$f(z)$在$U_0(z_0,R)$内解析, 即$z_0$是$f(z)$的一个孤立奇点, 那么$f(z)$在$z_0$处的留数定义成
	\begin{equation*}
		\mathrm{Res}(f,z_0) = \frac{1}{2\pi\mathrm{i}}\oint_{\abs{z-z_0}=\rho}f(z)\mathrm{d}z,
	\end{equation*}
	其中$0<\rho<R$.
	
	当$\infty$为$f(z)$的孤立奇点, 即存在$R>0$, 使得$f(z)$在$\mathrm{C}\backslash \overline{U(0,R)}$上解析, 则$f(z)$在$\infty$处的留数定义成\begin{equation*}
		\mathrm{Res}(f,\infty) = -\frac{1}{2\pi\mathrm{i}}\oint_{\abs{z}=\rho}f(z)\mathrm{d}z,
	\end{equation*}
	其中$R<\rho<+\infty$.

	从留数的定义知, 当有限复数$z_0$为$f(z)$的孤立奇点时, $f(z)$在$z_0$处的留数即为$f(z)$在$z_0$附近的Laurent展式中$\begin{lgathered}
		\frac{1}{z-z_0}
	\end{lgathered}$的系数$c_{-1}$. 而当$z_0 = \infty$时, 设$f(z)$在$z = \infty$处的Laurent展式为$\begin{lgathered}
		f(z) = \sum_{n = -\infty}^{\infty}c_nz^n
	\end{lgathered}$, 则\begin{equation*}
		\mathrm{Res}(f,\infty) = -c_{-1}.
	\end{equation*}
	
	(2)\myind{留数的计算}\\	
	\myind{规则I}~当$z_0\neq\infty$为$f(z)$的$m(m\geqslant 1)$阶极点时, $f(z)$在$z_0$处的留数可以由下式计算得到
	\begin{equation*}
		\mathrm{Res}(f,z_0) = \frac{1}{(m-1)!}\lim\limits_{z\to z_0}\frac{\mathrm{d}^{m-1}}{\mathrm{d}z^{m-1}}\left[ (z-z_0)^mf(z) \right].
	\end{equation*}
	\myind{规则II}~特例是$m = 1$的情况, 也就是$z = z_0$为$f(z)$的一阶极点的时候, 上式化为
	\begin{equation*}
		\mathrm{Res}(f,z_0) = \lim\limits_{z\to z_0} (z-z_0)f(z).
	\end{equation*}
	\myind{规则III}~当$z_0\neq\infty$为$f(z)$的$m$阶极点时, 在$z_0$的邻域内, 我们有\begin{equation*}
		f(z) = \dfrac{1}{(z-z_0)^m}g(z),
	\end{equation*}
	其中$g(z)$在$z_0$解析. 此时,
	\begin{equation*}
		\mathrm{Res}(f,z_0) = \frac{1}{(m-1)!}g^{(m-1)}(z_0).
	\end{equation*}
	\myind{规则IV}~若$z_0$为$\begin{lgathered}
		f(z) = \frac{P(z)}{Q(z)}
	\end{lgathered}$的一阶极点, 那么\begin{equation*}
		\mathrm{Res}\left[ \frac{P(z)}{Q(z)},z_0 \right] = \frac{P(z_0)}{Q'(z_0)}.
	\end{equation*}
	\myind{规则V}~\begin{equation*}
		\mathrm{Res}(f,\infty) = -\mathrm{Res}\left[ f\left(\frac{1}{z}\right)\cdot\frac{1}{z^2},0 \right].
	\end{equation*}
	
	(3)\myind{本章中的主要定理}
	\begin{theorem}[留数定理]
		设$\Omega$是扩充复平面$\overline{\mathbb{C}}$中以有限条逐段光滑曲线为边界的区域, $\infty\notin\partial\Omega$, $z_1,\cdots,z_n$($z_j$可以为$\infty$)位于$\Omega$的内部, $f(z)$在$\Omega$内除去$z_1,\cdots,z_n$外解析, 在$\overline{\Omega}$上除去$z_1,\cdots,z_n$外连续, 则
		\begin{equation*}
			\int_{\partial\Omega}f(z)\mathrm{d}z = 2\pi\mathrm{i}\sum_{k=1}^n\mathrm{Res}(f,z_k).
		\end{equation*}
	\end{theorem}
	
	留数定理有以下的特例: 取$\Omega = \overline{\mathbb{C}}$, 其边界是空集,此时有以下的形式
	\begin{theorem}[留数定理的特例]
		设$f(z)$在$\mathbb{C}$内除去去$z_1,\cdots,z_n$外是解析的, 则有\begin{equation*}
			\sum_{k=1}^n\mathrm{Res}(f,z_k) + \mathrm{Res}(f,\infty) = 0. 
		\end{equation*}
	\end{theorem}
	
	\begin{theorem}[辐角原理]
		设$f(z)$在区域$D$内亚纯, $\Gamma$是$D$内一条可求长的简单闭曲线, 其内部$\Omega\subset D$, 再设$f(z)$在$\Gamma$上无零点和极点, 则\begin{equation*}
			\frac{1}{2\pi\mathrm{i}}\int_{\Gamma}\frac{f'(z)}{f(z)}\mathrm{d}z = N-P,
		\end{equation*}
		其中$N$和$P$分别是$f(z)$在$\Gamma$内部的零点和极点的个数(记重数).
	\end{theorem}
	
	\begin{theorem}[Rouché定理]
		设$\Gamma$是可求长的Jordan曲线, 且其内部属于$D$, 再设$f(z)$和$g(z)$在$D$内解析, 在$\Gamma$上满足
		\begin{equation*}
			\abs{f(z)-g(z)}<\abs{f(z)},
		\end{equation*}
		则$f(z)$和$g(z)$在$\Gamma$内的零点个数(按重数计)相同.
	\end{theorem}
	
	\begin{theorem}[分歧覆盖定理]
		设$f(z)$在区域$D$内解析, $z_0\in D$. 记$w_0 = f(z_0)$. 设$z_0$是$f(z)-w_0$的$m$阶零点, 则存在$\rho>0$和$\delta>0$, 使得对于任意$w\in U(w_0,\rho)$, $f(z)-w$在圆盘$U(z_0,\delta)$内有且恰有$m$各不同的零点.
	\end{theorem}

	\begin{theorem}[辐角原理的推广]
		设$\Omega$是$\overline{C}$中以有限条逐段光滑曲线为边界的区域, $\infty\notin\partial\Omega$, $f(z)$是$\overline{\Omega}$的邻域上的亚纯函数, 且$f(z)$在$\partial\Omega$上无零点和极点. 再设$f(z)$在$\Omega$内的零点为$z_1,\cdots,z_n$, 并设$z_j$是$f(z)$的$\alpha_j$阶的零点($j = 1,\cdots,n$); $w_1,\cdots,w_k$是$f(z)$在$\Omega$内的极点, 并设$w_j$是$f(z)$的$\beta_j$阶极点($j=1,\cdots,k$). 则对任意$\overline{\Omega}$的邻域上解析的函数$g(z)$, 有
		\begin{equation*}
			\frac{1}{2\pi\mathrm{i}}\int_{\partial\Omega}g(z)\frac{f'(z)}{f(z)}\mathrm{d}z = \sum_{j=1}^{n}\alpha_jg(z_j) -\sum_{j= 1}^{k} \beta_jg(w_j).
		\end{equation*}
	\end{theorem}

练习: 用Rouché定理证明代数基本定理.

代数基本定理有许多证明, 较早给出代数基本定理证明的是d'Alembert和Gauss, 后者一生中给出了四个证明. 我们这里给出的证明十分干净利落, 当然是因为用了在19世纪由Cauchy, Riemann和Weierstrass建立起来的复分析这个十分强大的数学武器.
\section{习题1-5}

\begin{yyEx}
	\begin{enumerate}
		\item 设$\begin{lgathered}
			f(z) = \sum_{n = 0}^{+\infty}a_nz^n
		\end{lgathered}$在$\abs{z}<R$内解析, $k$为正整数, 那么$\begin{lgathered}
		\mathrm{Res}\left[ \frac{f(z)}{z^k},0 \right] = (\quad\quad)
		\end{lgathered}$
		\item 设$a$为解析函数$f(z)$的$m$级零点, 那么$\begin{lgathered}
			\mathrm{Res}\left[ \frac{f'(z)}{f(z)},a \right] = (\quad\quad)
		\end{lgathered}$
		\item 下列函数中, $\mathrm{Res}[f(z),0] = 0$的是$(\quad\quad)$.\\
			(A) $\begin{lgathered}
				f(z) = \frac{\mathrm{e}^z-1}{z^2}
			\end{lgathered}$~~~~~~~~~(B) $\begin{lgathered}
				f(z) = \frac{\sin z}{z}-\frac{1}{z}
			\end{lgathered}$\\(C) $\begin{lgathered}
				f(z) = \frac{\sin z+\cos z}{z}
			\end{lgathered}$~~(D) $\begin{lgathered}
				f(z) = \frac{1}{\mathrm{e}^z-1}-\frac{1}{z}.
			\end{lgathered}$
		\item 下列命题中, 正确的是$(\quad\quad)$.\\
			(A) 设$f(z) = (z-z_0)^{-n}\varphi(z)$, $\varphi(z)$在$z_0$点解析, $m$为自然数, 则$z_0$为$f(z)$的$m$级极点.\\
			(B) 如果无穷远点$\infty$是函数$f(z)$的可去奇点, 那么$\mathrm{Res}[f(z),\infty] = 0$.\\
			(C) 如果$z = 0$为偶函数$f(z)$的一个孤立奇点, 那么$\mathrm{Res}[f(z),0] = 0$.\\
			(D) 若$\begin{lgathered}
				\oint_Cf(z)\mathrm{d}z = 0
			\end{lgathered}$, 则$f(z)$在$C$内无奇点.
		\item $\begin{lgathered}
		\mathrm{Res}[z^3\cos\frac{2\mathrm{i}}{z},\infty] = (\quad\quad).
		\end{lgathered}$
		\item $\begin{lgathered}
			\mathrm{Res}[z^2\mathrm{e}^{\frac{1}{z-\mathrm{i}}},\mathrm{i}] = (\quad\quad).
		\end{lgathered}$
		\item 下列命题中, 不正确的是$(\quad\quad)$.\\
			(A) 若$z_0(\neq\infty)$是$f(z)$的可去奇点或解析点, 则$\mathrm{Res}[f(z),z_0] = 0$.\\
			(B) 若$P(z)$与$Q(z)$在$z_0$解析, $z_0$为$Q(z)$的一级零点, 则
				\begin{equation*}
					\mathrm{Res}\left[ \frac{P(z)}{Q(z)},z_0 \right] = \frac{P(z_0)}{Q'(z_0)}.
				\end{equation*}
			(C)若$z_0$为$f(z)$的$m$级极点, $n\geqslant m$为自然数, 则
				\begin{equation*}
					\mathrm{Res}[f(z),z_0] = \frac{1}{n!}\lim\limits_{z\to z_0}\frac{\mathrm{d}^n}{\mathrm{d}z^n}[(z-z_0)^{n+1}f(z)].
				\end{equation*}
			(D) 如果无穷远点$\infty$为$f(z)$的一级极点, 则$z = 0$为$f(1/z)$的一级极点, 并且\begin{equation*}
				\mathrm{Res}[f(z),\infty] = \lim\limits_{z\to 0}zf\left(\frac{1}{z}\right).
			\end{equation*}
		\item 设$n>1$为正整数, 则$\begin{lgathered}
			\oint_{\abs{z} = 2} \frac{1}{z^n-1}\mathrm{d}z = (\quad\quad).
		\end{lgathered}$
		\item 积分$\begin{lgathered}
		\oint_{\abs{z} = 3/2} \frac{z^9}{z^{10}-1}\mathrm{d}z = (\quad\quad).
		\end{lgathered}$
		\item 积分$\begin{lgathered}
		\oint_{\abs{z} = 1} z^2\sin\frac{1}{z}\mathrm{d}z = (\quad\quad).
		\end{lgathered}$
	\end{enumerate}
\end{yyEx}

\begin{yyEx}
	填空题
	\begin{enumerate}
		\item $f(z) = e^{1/z}$在本性奇点$z = 0$处的留数$\mathrm{Res}f(z) = \underline{\quad\quad}$. $\begin{lgathered}
			f(z) = \frac{\mathrm{e}^{\mathrm{i}z}}{1+z^2}
		\end{lgathered}$, 则$\mathrm{Res}(f,i) = \underline{\quad\quad}.$
		\item $\begin{lgathered}
			I = \oint_{\abs{z} = 2}\frac{\mathrm{e}^z}{z^2-1}\mathrm{d}z = \underline{\quad\quad}
		\end{lgathered}$, $\begin{lgathered}
		I = \int_{\abs{z} = 1}\frac{\cos z}{z^3}\mathrm{d}z = \underline{\quad\quad}
		\end{lgathered}$
		\item $\begin{lgathered}
			I = \int_{0}^{2\pi}\frac{\mathrm{d}\theta}{1+a\cos\theta}, \abs{a}<1
		\end{lgathered}$, 则$I = \underline{\quad\quad}$. $\begin{lgathered}
		I = \int_{0}^{+\infty}\frac{\cos x}{x^2+b^2}\mathrm{d}x
		\end{lgathered}$, 则$I = \underline{\quad\quad}$.
		\item 设函数$\begin{lgathered}
			f(z) = \exp\left\{ z^2 + \frac{1}{z^2} \right\}, 
		\end{lgathered}$则$\mathrm{Res}[f(z),0] = \underline{\quad\quad}.$
		\item 设$z = a$为函数$f(z)$的$m$级极点, 那么$\begin{lgathered}
			\mathrm{Res}\left[ \frac{f'(z)}{f(z)},a \right]= \underline{\quad\quad}.
		\end{lgathered}$
		\item 设$\begin{lgathered}
			f(z) = \frac{2z}{1+z^2}
		\end{lgathered}$, 则$\mathrm{Res}[f(z),\infty] = \underline{\quad\quad}.$
		\item 设$\begin{lgathered}
			f(z) = \frac{1-\cos z}{z^5}
		\end{lgathered}$, 则$\mathrm{Res}[f(z),0] = \underline{\quad\quad}.$
		\item 积分$\begin{lgathered}
			\oint_{\abs{z} = 1} z^3\mathrm{e}^{1/z}\mathrm{d}z = \underline{\quad\quad}.
		\end{lgathered}$
		\item 积分$\begin{lgathered}
		\oint_{\abs{z} = 1} \frac{1}{\sin z}\mathrm{d}z = \underline{\quad\quad}.
		\end{lgathered}$
		\item 积分$\begin{lgathered}
		\int_{-\infty}^{+\infty} \frac{x\mathrm{e}^{ix}}{1+x^2}\mathrm{d}x = \underline{\quad\quad}.
		\end{lgathered}$
	\end{enumerate}
\end{yyEx}

\begin{yyEx}
	求下列各函数$f(z)$在有限奇点处的留数:
	\begin{enumerate}
		\item $\begin{lgathered}
			\frac{z+1}{z^2-2z};
		\end{lgathered}$
		\item $\begin{lgathered}
		\frac{1-\mathrm{e}^{2z}}{z^4};
		\end{lgathered}$
		\item $\begin{lgathered}
		\frac{1+z^4}{(z^2+1)^3};
		\end{lgathered}$
		\item $\begin{lgathered}
		\frac{z}{\cos z};
		\end{lgathered}$
		\item $\begin{lgathered}
		\cos\frac{1}{1-z};
		\end{lgathered}$
		\item $\begin{lgathered}
		z^2\sin\frac{1}{z};
		\end{lgathered}$
		\item $\begin{lgathered}
		\frac{1}{z\sin z};
		\end{lgathered}$
		\item $\begin{lgathered}
		\frac{\sinh z}{\cosh z};
		\end{lgathered}$
		\item $\begin{lgathered}
		\frac{z\mathrm{e}^z}{(z-a)^3};
		\end{lgathered}$
		\item $\begin{lgathered}
		\frac{z-\sin z}{z^6}.
		\end{lgathered}$
	\end{enumerate}
\end{yyEx}

\begin{yyEx}
	计算下列各积分(利用留数; 圆周均取正向):
	\begin{enumerate}
		\item $\begin{lgathered}
			\oint_{\abs{z} = 3/2} \frac{\sin z}{z} \mathrm{d}z;
		\end{lgathered}$
		\item $\begin{lgathered}
		\oint_{\abs{z} = 2}\frac{\mathrm{e}^{2z}}{(z-1)^2}  \mathrm{d}z;
		\end{lgathered}$
		\item $\begin{lgathered}
		\oint_{\abs{z} = 2}  \frac{z}{z^4-1}\mathrm{d}z;
		\end{lgathered}$
		\item $\begin{lgathered}
		\oint_{\abs{z} = 2}  \frac{1}{(z+\mathrm{i})^{10}(z-1)(z-3)}\mathrm{d}z;
		\end{lgathered}$
		\item $\begin{lgathered}
		\oint_{\abs{z} = 1/4}  \frac{z\sin z}{(\mathrm{e}^z-1-z)^2}\mathrm{d}z;
		\end{lgathered}$
	\end{enumerate}
\end{yyEx}

\begin{yyEx}
	求下列函数在无穷远点的留数值:
	\begin{enumerate}
		\item $\begin{lgathered}
			\frac{\mathrm{e}^z}{z^2-1};
		\end{lgathered}$
		\item $\begin{lgathered}
		\frac{1}{z};
		\end{lgathered}$
		\item $\begin{lgathered}
		\frac{\cos z}{z};
		\end{lgathered}$
		\item $\begin{lgathered}
		\frac{z^{15}}{(z^2+1)^2(z^4+2)^3};
		\end{lgathered}$
		\item $\begin{lgathered}
		(z^2+1)\mathrm{e}^z;
		\end{lgathered}$
		\item $\begin{lgathered}
		\exp\left(-\frac{1}{z^2}\right).
		\end{lgathered}$
	\end{enumerate}
\end{yyEx}

\section{习题6-10}

\begin{yyEx}
	计算下列各积分, $C$为正向圆周:
	\begin{enumerate}
		\item $\begin{lgathered}
			\oint_C \frac{5z^{27}}{(z^2-1)^4(z^4+2)^5}\mathrm{d}z,C:\abs{z} = 4;
		\end{lgathered}$
		\item $\begin{lgathered}
		\oint_C \frac{z^3}{z+1}\mathrm{e}^{\frac{1}{z}} \mathrm{d}z,C:\abs{z} = 2;
		\end{lgathered}$
		\item $\begin{lgathered}
		\oint_C \frac{\mathrm{e}^z}{z^2-1}\mathrm{d}z,C:\abs{z} = 2;
		\end{lgathered}$
		\item $\begin{lgathered}
		\oint_C \frac{\mathrm{d}z}{\varepsilon z^2+2z+\varepsilon},C:\abs{z} = 1,0<\varepsilon<1.
		\end{lgathered}$
	\end{enumerate}
\end{yyEx}

\begin{yyEx}
	计算下列积分:
	\begin{enumerate}
		\item $\begin{lgathered}
		\int_{0}^{2\pi}\frac{1}{5+3\sin\theta}\mathrm{d}\theta;
		\end{lgathered}$
		\item $\begin{lgathered}
		\int_{-\infty}^{+\infty}\frac{1}{(1+x^2)^2}\mathrm{d}x;
		\end{lgathered}$
		\item $\begin{lgathered}
		\int_{-\infty}^{+\infty}\frac{x\sin x}{(1+x^2)}\mathrm{d}x;
		\end{lgathered}$
		\item $\begin{lgathered}
		\int_{0}^{\pi/2}\frac{\mathrm{d}\theta}{1+\cos^2\theta};
		\end{lgathered}$
		\item $\begin{lgathered}
		\int_{0}^{2\pi}\frac{\mathrm{d}\theta}{a+b\cos\theta}(a^2>b^2);
		\end{lgathered}$
		\item $\begin{lgathered}
		\int_{-\infty}^{+\infty}\frac{\mathrm{d}x}{x^2+2x+2};
		\end{lgathered}$
		\item $\begin{lgathered}
		\int_{0}^{+\infty}\frac{\cos x}{(x^2+4)(x^2+1)}\mathrm{d}x;
		\end{lgathered}$
		\item $\begin{lgathered}
		\int_{-\infty}^{+\infty}\frac{\cos(2x)}{x^2+1}\mathrm{d}x.
		\end{lgathered}$
	\end{enumerate}
\end{yyEx}

\begin{yyEx}
	利用留数计算下列积分:
	\begin{enumerate}
		\item $\begin{lgathered}
			\int_{0}^{\pi}\frac{\mathrm{d}\theta}{a^2+\sin^2\theta}(a>0);
		\end{lgathered}$
		\item $\begin{lgathered}
		\int_{-\infty}^{+\infty}\frac{x^2-x+2}{x^4+10x^2+9}\mathrm{d}x;
		\end{lgathered}$
		\item $\begin{lgathered}
		\int_{0}^{+\infty}\frac{x\sin x\cos 2x}{x^2+1}\mathrm{d}x;
		\end{lgathered}$
		\item $\begin{lgathered}
		\int_{-\infty}^{+\infty}\frac{\cos(x-1)}{x^2+1}\mathrm{d}x.
		\end{lgathered}$
	\end{enumerate}
\end{yyEx}

\begin{yyEx}
	求下列条件下$f(z)/g(z)$在奇点$z_0$处的留数:
	\begin{enumerate}
		\item $f(z)$在$z_0$的邻域$G_1$内解析, 且$f(z_0)\neq 0$, 而$z_0$是$g(z)$的二级零点;
		\item $z_0$是$f(z)$的一级零点,是$g(z)$的三级零点.
	\end{enumerate}
\end{yyEx}

\begin{yyEx}
	试用各种不同的方法计算$\begin{lgathered}
		\mathrm{Res}\left[ \frac{5z-2}{z(z-1)},1 \right].
	\end{lgathered}$
\end{yyEx}

\section{习题11}

\begin{yyEx}
	证明下列各题:
	\begin{enumerate}
		\item 设$a$为$f(z)$的孤立奇点, 试证: 若$f(z)$是奇函数, 则$\mathrm{Res}[f(z),a] = \mathrm{Res}[f(z), -a]$; 若$f(z)$是偶函数, 则$\mathrm{Res}[f(z),a] = -\mathrm{Res}[f(z), -a]$.
		\item 设$f(z)$以$a$为简单极点, 且在$a$处的留数为$A$, 证明
		
		$\begin{lgathered}
			\lim\limits_{z\to a}\frac{\abs{f'(z)}}{1+\abs{f(z)}^2} = \frac{1}{\abs{A}}.
		\end{lgathered}$
		\item 若函数$\varPhi(z)$在$\abs{z}\leqslant 1$上解析, 当$z$为实数时, $\varPhi(z)$取实数而且$\varPhi(0) = 0$, $f(x,y)$表示$\varPhi(x+\mathrm{i}y)$的虚部, 试证明
		\begin{equation*}
			\int_{0}^{2\pi}\frac{t\sin\theta}{1-2t\cos\theta+t^2}f(\cos\theta,\sin\theta)\mathrm{d}\theta = \pi\varPhi(t)~(-1<t<1).
		\end{equation*}
	\end{enumerate}
\end{yyEx}


\chapter{矢量分析与场论初步}

\section{习题1-5}

\begin{yyEx}
	求向量函数$\begin{lgathered}
	\bm{F}(M) = \left( \frac{xz}{\sqrt{xz+1}-1}, \mathrm{e}^{x^2z+y^2},\frac{\sin(xy)}{y}\right)
	\end{lgathered}$的极限$\begin{lgathered}
	\lim\limits_{M\to(0,1,-1)}\bm{F}(M).
	\end{lgathered}$
\end{yyEx}

\begin{yyEx}
	求向量函数$\begin{lgathered}
	\bm{F}(M) = \left( \mathrm{e}^{x^2+y^2},yz+x^2,\frac{y-1}{1+xz} \right)
	\end{lgathered}$的极限$\begin{lgathered}
		\lim\limits_{M\to(1,0,1)}\bm{F}(M).
	\end{lgathered}$
\end{yyEx}

\begin{yyEx}
	讨论下列向量函数在指定点处的连续性:
	\begin{enumerate}
		\item $\begin{lgathered}
		\bm{F}(M) = \left( \frac{x^3+y^3}{x^2+y^2},x+y+z,x^2+z^2 \right)
		\end{lgathered}$在点$M_0(0,0,2)$;
		\item $\begin{lgathered}
		\bm{F}(M) = \left( \frac{\sin(xy)}{x},x+3y,3z+xy \right)
		\end{lgathered}$在点$M_0(1,1,-1)$.
	\end{enumerate}
\end{yyEx}

\begin{yyEx}
	求矢量函数$\bm{A}(x,y,z) = x\sin(x+y)\bm{i}+x^4y^2\bm{j}+(x+\mathrm{e}^{yz})\bm{k}$的偏导数 $\begin{lgathered}
		\frac{\partial \bm{A}}{\partial x}
	\end{lgathered}$,$\begin{lgathered}
	\frac{\partial \bm{A}}{\partial y}
	\end{lgathered}$和$\begin{lgathered}
	\frac{\partial \bm{A}}{\partial z}
	\end{lgathered}$.
\end{yyEx}

\begin{yyEx}
	\begin{enumerate}
		\item 已知$\begin{lgathered}
		\bm{A}(t) = (1+3t^2)\bm{i} - 2t^3\bm{j}+\frac{t}{2}\bm{k}
		\end{lgathered}$, 求$\begin{lgathered}
			\int_{0}^{2}\bm{A}(t)\mathrm{d}t;
		\end{lgathered}$
		\item 计算$\begin{lgathered}
			\int \bm{F}(M)\mathrm{d}x
		\end{lgathered}$, 其中$\begin{lgathered}
			\bm{F}(M) = \left( 1+3x^2,x^2+\frac{\mathrm{e}^x}{x},\ln x \right)
		\end{lgathered}$
	\end{enumerate}
\end{yyEx}

\section{习题6-10}

\begin{yyEx}
	计算下列各题:
	\begin{enumerate}
		\item 设数量场$u = \ln\sqrt{x^2+y^2+z^2}$, 求$\mathrm{div}(\mathrm{grad}u)$.
		\item 设$\bm{r}=x\bm{i}+y\bm{j}+z\bm{k}$, $r = \sqrt{x^2+y^2+z^2}$是$\bm{r}$的模, $\bm{c}$是常向量, 求$\mathrm{rot}[f(r)\bm{c}]$.
		\item 求向量场$\bm{F} = xy^2\bm{i}+y\mathrm{e}^z\bm{j}+x\ln(1+z^2)\bm{k}$ 在点$P(1,1,0)$处的散度$\mathrm{div}\bm{F}$.
	\end{enumerate}
\end{yyEx}
	
\begin{yyEx}
	设数量场$\begin{lgathered}
	u = \frac{a}{r}
	\end{lgathered}$, 其中$r = \sqrt{x^2+y^2+z^2}$, $a$为常数. 求:
	\begin{enumerate}
		\item $u$在$P(x_0,y_0,z_0)$处的梯度$\begin{lgathered}\mathrm{grad}u\big\vert_{P}\end{lgathered}$;
		\item $u$在$P$处沿$x_0\bm{i}+y_0\bm{j}+z_0\bm{k}$方向的方向导数.
	\end{enumerate}
\end{yyEx}
	
\begin{yyEx}
	一质点在力场$\bm{F} = (y-z)\bm{i}+(z-x)\bm{j}+(x-y)\bm{k}$的作用下, 沿螺旋线$x = a\cos t,y = a\sin t, z = bt$ 运动, 求其从$t = 0$到$t = 2\pi$时所做的功.
\end{yyEx}

\begin{yyEx}
	设曲面$S$由平面$x = 0,y = 0,z = 0$和$x+y+z = 1$所构成, 求向量场$\bm{F} = x\bm{i}+y\bm{j}+z\bm{k}$ 从内穿出闭曲面$S$的通量$\varPhi$.
\end{yyEx}

\begin{yyEx}
	已知数量场$\begin{lgathered}
		u =\ln\frac{1}{r}
	\end{lgathered}$, 其中 
	
	$r = \sqrt{(x-a)^2 + (y-b)^2 + (z-c)^2}$. 在空间$Oxyz$的哪些点上$\abs{\mathrm{grad}u} = 1$成立.
\end{yyEx}

\section{习题11-15}

\begin{yyEx}
	\begin{enumerate}
		\item 证明:$\begin{lgathered}
			\grad\frac{1}{r} = -\frac{\bm{r}}{r^3},\grad\frac{1}{r^3} = -\frac{3\bm{r}}{r^5}
		\end{lgathered}$, 其中$r = x\bm{i}+y\bm{j}+z\bm{k}$.
		\item 若$u = u(v,w), v = v(x,y,z), w = w(x,y,z)$,
		
		 证明:$\begin{lgathered}
			\grad u = \frac{\partial u}{\partial v}\grad v+\frac{\partial u}{\partial w}\grad w.
		\end{lgathered}$
		\item 求标量场$u = x^2+2y^2+3z^2+xy+3x-2y-6z$在点$(1,-2,1)$处的梯度大小和方向.
		\item 证明$\grad u$为常矢量的充要条件是$u$为线性函数$u = ax+by+cz+d$.
	\end{enumerate}
\end{yyEx}

\begin{yyEx}
	\begin{enumerate}
		\item 证明: $\begin{lgathered}
			\grad\bm{\cdot}\frac{\bm{r}}{r} = \frac{2}{r},\grad\bm{\cdot}(r\bm{k}) = \frac{\bm{r}}{r}\bm{\cdot}\bm{k}
		\end{lgathered}$($\bm{k}$为常矢量,$r = \sqrt{x^2+y^2+z^2}$).
		\item 若$\bm{A} = \bm{A}(u),u = u(x,y,z)$, 证明: $\begin{lgathered}
		\grad\bm{\cdot A} = \frac{\mathrm{d}\bm{A}}{\mathrm{d}u}\bm{\cdot}\grad u.
		\end{lgathered}$
		\item 可压缩流体的密度为非稳定场$\rho(x,y,z,t)$, 流体的质量守恒定律为$\begin{lgathered}
			\varoiint\limits_S\rho\bm{v\cdot}\mathrm{d}\bm{s} = -\frac{\mathrm{d}}{\mathrm{d}\bm{s}}\varoiint\limits_{\Omega}\rho\mathrm{d}\Omega
		\end{lgathered}$, 试由此推出流体力学的连续性方程$\begin{lgathered}
			\frac{\partial\rho}{\partial t}+\grad\bm{\cdot}(\rho\bm{v}) = 0.
		\end{lgathered}$
	\end{enumerate}
\end{yyEx}

\begin{yyEx}
	\begin{enumerate}
		\item 证明:$\begin{lgathered}
			\grad\times\frac{\bm{r}}{r^3} = \bm{0},\grad\times\left[ F(r)\bm{r} \right] = \bm{0}.
		\end{lgathered}$
		\item 若$\bm{k}$为常矢量, 证明$\begin{lgathered}
			\grad\times\frac{\bm{k}}{r} = \bm{k}\times\frac{\bm{r}}{r^3}.
		\end{lgathered}$
		\item 若$\bm{k}$为常矢量, 证明$\begin{lgathered}
		\grad\times\left[ F(r) \bm{k} \right] = F'(r)\frac{\bm{r}}{r}\times\bm{k}.
		\end{lgathered}$
		\item 若$\bm{A} = \bm{A}(u),u = u(x,y,z)$, 证明: $\begin{lgathered}
			\grad\bm{\cdot A} = \grad u\times\frac{\mathrm{d}\bm{A}}{\mathrm{d}u}.
		\end{lgathered}$
		\item 证明: $(\bm{A}\bm{\cdot}\grad)\bm{r} = \bm{A}$.
	\end{enumerate}
\end{yyEx}

\begin{yyEx}
	设$\bm{k}$为常矢量, $\grad\times\bm{E} = \bm{0}$. 证明:
	\begin{enumerate}
		\item $\grad(\bm{k\cdot E}) = \bm{k\cdot}\grad\bm{E}$;
		\item $\grad(\bm{k\cdot r}) = \bm{k}$;
		\item $\begin{lgathered}
			\grad\left( \frac{\bm{k\cdot r}}{r^3} \right) = -\left[ \frac{3(\bm{k\cdot r})\bm{r}}{r^5} - \frac{\bm{k}}{r^3} \right]
		\end{lgathered}$;
		\item $\begin{lgathered}
		\grad\left( \frac{\bm{k}\times\bm{r}}{r^3} \right) =  \frac{3(\bm{k\cdot r})\bm{r}}{r^5} - \frac{\bm{k}}{r^3}
		\end{lgathered}$.
	\end{enumerate}
\end{yyEx}

\begin{yyEx}
	\begin{enumerate}
		\item 证明$\begin{lgathered}
			\bm{E}\times(\grad\times \bm{E}) = \frac{1}{2}\grad \bm{E}^2-(\bm{E}\bm{\cdot}\grad\bm{E});
		\end{lgathered}$
		\item 证明$\begin{lgathered}
			\iiint\limits_{v}\grad\times\bm{A}\mathrm{d}v = - \varoiint\limits_s\bm{A}\times\mathrm{d}\bm{s};
		\end{lgathered}$
		\item 由电磁感应定律的积分形式$\begin{lgathered}
			\oint_l\bm{E}\mathrm{d}\bm{l} = -\frac{d\iint\limits_{s}\bm{B}\bm{\cdot}\mathrm{d}\bm{s}}{\mathrm{d}l}
		\end{lgathered}$, 推出其微分形式$\begin{lgathered}
			\grad\times\bm{E} = -\frac{\partial\bm{B}}{\partial t};
		\end{lgathered}$
		
		\item 证明$\begin{lgathered}
			\oint_l(\bm{A}\times\bm{r})\bm{\cdot}\mathrm{d}\bm{l} = 2\iint\limits_{s}\bm{A}\mathrm{d}\bm{s}
		\end{lgathered}$~($\bm{A}$为常矢量).
	\end{enumerate}
\end{yyEx}

\section{习题16-20}

\begin{yyEx}
	证明:$\grad^2(uv) = u\grad^2 v+v\grad^2u+2\grad u\bm{\cdot}\grad v$.
\end{yyEx}

\begin{yyEx}
	已知$u(r,\theta,\varphi) = 2r\sin\varphi+r^2\cos\theta$, 求$\grad u,\Delta u$.
\end{yyEx}

\begin{yyEx}
	已知$\bm{A}(M) = r\cos^2\theta\bm{e}_{\rho} + r\sin\theta\bm{e}_{\theta} + z\cos\theta\bm{e}_{z}$, 求$\mathrm{div}\bm{A}$, $\mathrm{rot}\bm{A}$.
\end{yyEx}

\begin{yyEx}
	设$\bm{r} = x\bm{i}+y\bm{j}+z\bm{k}$, 在柱面坐标系和球面坐标系下, 证明$\nabla\bm{\cdot} \bm{r} = 3$.
\end{yyEx}

\begin{yyEx}
	证明下列向量场是调和场:
	\begin{enumerate}
		\item $\bm{F}(x,y,z) = (2x+y)\bm{i}+(4y+x+2z)\bm{j}+(2y-6z)\bm{k}$;
		\item $\bm{F}(x,y,z) = (1+2x-5y)\bm{i}+(4y-5x+7z)\bm{j}+(7y-6z)\bm{k}$;
		\item $\bm{F}(x,y,z) = yz\bm{i}+zx\bm{j}+xy\bm{k}$.
	\end{enumerate}
\end{yyEx}



\chapter{数学物理方程及其定解条件}

\section{习题1-5}

\begin{yyEx}
	一均匀杆的原长为$l$, 一端固定, 另一端沿杆的轴线方向拉长$e$而静止, 突然放手任其振动, 试建立振动方程与定解条件.
\end{yyEx}

\begin{yyEx}
	长为$l$的弦两端固定, 开始时在$x = c$受冲量$k$的作用, 试写出相应的定解问题.
\end{yyEx}

\begin{yyEx}
	长为$l$的均匀杆, 侧面绝缘. 一端温度为零, 另一端有恒热流$q$进入(即单位时间内通过单位截面流入的热量为$q$), 杆的初始温度分布为$\begin{lgathered}\frac{x(l-x)}{2}\end{lgathered}$, 试写出相应的定解问题.
\end{yyEx}

\begin{yyEx}
	半径为$R$而表面熏黑的金属长圆柱体, 受到阳光照射, 阳光方向垂直于柱轴(图5.7), 热流强度为$M$, 写出这个圆柱的热传导问题的边界条件.
\end{yyEx}

\begin{yyEx}
	若$F(z),G(z)$为两个任意二次连续可微函数, 验证\begin{equation*}
		u = F(x+at)+G(x-at)
	\end{equation*}
	满足方程$\begin{lgathered}
		\frac{\partial^2 u}{\partial t^2} = a^2\frac{\partial^2 u}{\partial x^2}.
	\end{lgathered}$
\end{yyEx}

\section{习题6-7}

\begin{yyEx}
	验证线性齐次方程的叠加原理, 即若$u_1(x,y),u_2(x,y),\cdots,u_n(x,y),\cdots$均是线性二阶齐次方程
	\begin{equation*}
		A\frac{\partial^2 u}{\partial x^2} + 2B\frac{\partial^2 u}{\partial x\partial y} + C\frac{\partial^2 u}{\partial y^2} + D\frac{\partial u}{\partial x} + E\frac{\partial u}{\partial y} + Fu = 0
	\end{equation*}
	的解, 其中$A,B,C,D,E,F$都只是$x,y$的函数, 而且级数$u = \sum\limits_{i = 1}^{+\infty}c_iu_i(x,y)$收敛, 其中$c_i(i=1,2,\cdots)$为任意常数, 并且对$x,y$可以逐次微分两次, 求证$u = \sum\limits_{i = 1}^{+\infty}c_iu_i(x,y)$仍是原方程的解.
\end{yyEx}

\begin{yyEx}
	把下列方程化为标准型
	\begin{enumerate}
		\item $u_{xx}+4u_{xy} + 5u_{yy} + u_x+2u_y = 0$;
		\item $u_{xx}+yu_{yy} = 0$;
		\item $u_{xx}+xu_{yy} = 0$;
		\item $y^2u_{xx}+x^2u_{yy} = 0$.
	\end{enumerate}
\end{yyEx}



\chapter{分离变量法}

\section{习题1-5}

\begin{yyEx}
	就下列初始条件及边界条件求解弦振动方程
	\begin{equation*}
		u(x,0) = 0,\frac{\partial u(x,0)}{\partial t} = x(l-x);~~u(0,t) = u(l,t) = 0.
	\end{equation*}
\end{yyEx}

\begin{yyEx}
	两端固定的弦长度为$l$, 用细棒敲击弦上$x = x_0$点, 即在$x = x_0$施加冲力, 设其冲量为$I$, 求解弦的振动, 即求解定解问题
	\begin{equation*}
		\begin{dcases}
			u_{tt} - a^2u_{xx} = 0, &0\leqslant x\leqslant l,t>0,\\
			u\big|_{x = 0} = u\big|_{x = l} = 0, &t>0,\\
			u\big|_{t = 0} = 0,u_t\big|_{t = 0} = \frac{I}{\rho}\delta(x - x_0), &0\leqslant x\leqslant l.
		\end{dcases}
	\end{equation*}
\end{yyEx}

\begin{yyEx}
	长为$l$的杆, 一端固定, 另一端因受力$F_0$而伸长, 求解杆在放手后的振动. 其定解问题为
	\begin{equation*}
	\begin{dcases}
	u_{tt} - a^2u_{xx} = 0, &0\leqslant x\leqslant l,t>0,\\
	u\big|_{x = 0} = 0,~~ u_x\big|_{x = 0} = 0, &t>0,\\
	u(x,0) = \int_{0}^{x}\frac{\partial u}{\partial x}\mathrm{d}x = \int_{0}^{x}\frac{F_0}{YS}\mathrm{d}x = \frac{F_0x}{YS},~~u_t(x,0) = 0, &0\leqslant x\leqslant l.
	\end{dcases}
	\end{equation*}
\end{yyEx}

\begin{yyEx}
	长为$x$的理想传输线远端开路, 先把传输线充电到电位差$v_0$, 然后把近端短路, 求线上的电压$V(x,t)$, 其定解问题为
	\begin{equation*}
	\begin{dcases}
	V_{tt} - a^2V_{xx} = 0~~(a^2 = LC,0<x<l),&~\\
	V(0,t) = 0,~~V_x(l,t) = -\left(R+L\frac{\partial}{\partial t}i_x \right)\bigg|_{x = l} = 0,&~\\
	V(x,0) = v_0,~~V_t(x,0) = \frac{-1}{C}i_x\bigg|_{t = 0} = 0.
	\end{dcases}
	\end{equation*}
	其中$i$表示电流强度, $i_x$表示电流强度对$x$的偏导数.
\end{yyEx}

\begin{yyEx}
	设弦的两端固定于$x = 0$及$x = l$, 弦的初始位移如图6.5所示, 初速度为零, 有没有外力作用, 求弦作横向振动时的位移函数$u(x,t)$.
\end{yyEx}

\section{习题6-10}

\begin{yyEx}
	试求适合于下列初始条件及边界条件的一维热传导方程的解
	\begin{equation*}
		u\big\vert_{t = 0} = x(l-x),~~u\big\vert_{x = 0} = u\big\vert_{x = l} = 0.
	\end{equation*}
\end{yyEx}

\begin{yyEx}
	求解一维热传导方程, 其初始条件及边界条件为
	\begin{equation*}
	u\big\vert_{t = 0} = x,~~u_x\big\vert_{x = 0} = 0,~~ u_x\big\vert_{x = l} = 0.
	\end{equation*}
\end{yyEx}

\begin{yyEx}
	在圆形区域内求解$\grad^2u = 0$, 使满足边界条件:
	\begin{enumerate}
		\item $u\big\vert_{\rho = a} = A\cos\varphi$;
		\item $u\big\vert_{\rho = a} = A+B\sin\varphi$.
	\end{enumerate}
\end{yyEx}

\begin{yyEx}
	就下列初始条件和边界条件求解弦振动方程
	\begin{equation*}
		\begin{split}
			&u\big\vert_{t = 0} = \begin{dcases}
			x,&0<x\leqslant \frac{1}{2};\\
			1-x,&\frac{1}{2}<x<1.
			\end{dcases} \\
			&\frac{\partial u}{\partial t}\bigg\vert_{t = 0} = x(x-1),~~u\big\vert_{x = 0} = u\big\vert_{x = l} = 0.
		\end{split}		
	\end{equation*}
\end{yyEx}

\begin{yyEx}
	求下列定解问题
	\begin{equation*}
		\begin{dcases}
			&\frac{\partial u}{\partial t} = a^2\frac{\partial ^2u}{\partial x^2} + A,\\
			&u\big\vert_{x = 0} = u\big\vert_{x = l} = 0,\\
			&u\big\vert_{t = 0} = 0.
		\end{dcases}
	\end{equation*}
\end{yyEx}

\section{习题11-15}

\begin{yyEx}
	求满足下列定解条件的一维热传导方程$u_t = a^2u_{xx}~(0<x<l,t>0)$的解
	\begin{equation*}
		u\big\vert_{x = 0} = 10,~u\big\vert_{x = l} = 5,~u\big\vert_{t = 0} = kx, k\text{为常数}.
	\end{equation*}
\end{yyEx}

\begin{yyEx}
	试确定下列定解问题
	\begin{equation*}
	\begin{dcases}
	&\frac{\partial u}{\partial t} = a^2\frac{\partial ^2u}{\partial x^2} + f(x),\\
	&u\big\vert_{x = 0} = A,~~u\big\vert_{x = l} = B,\\
	&u\big\vert_{t = 0} = g(x).
	\end{dcases}
	\end{equation*}
	解的一般形式.
\end{yyEx}

\begin{yyEx}
	在矩形区域$0\leqslant x\leqslant a, 0\leqslant y\leqslant b$ 内求拉普拉斯方程$(u_{xx}+u_{yy} = 0)$的解, 使其满足边界条件
	\begin{equation*}
		\begin{dcases}
			u\big\vert_{x = 0} = 0, &u\big\vert_{x = a} = Ay,\\
			\frac{\partial u}{\partial y}\bigg\vert_{y = 0} = 0, &\frac{\partial u}{\partial y}\bigg\vert_{y = b} = 0.
		\end{dcases}
	\end{equation*}
\end{yyEx}

\begin{yyEx}
	求解薄膜的恒定表面浓度扩散问题. 薄膜厚度为$l$, 杂质从两面进入薄膜, 由于薄膜周围气体含有充分的杂质, 薄膜表面上的杂志浓度得以保持为恒定的$N_0$, 其定解问题为
	\begin{equation*}
		\begin{dcases}
			&u_t-a^2u_{xx} = 0,\\
			&u(0,t) = u(l,t) = N_0,\\
			&u(x,0) = 0,
		\end{dcases}
	\end{equation*}
	求解$u$.
\end{yyEx}

\begin{yyEx}
	求半带型区域$(0\leqslant x\leqslant a,y\geqslant 0)$内的静电势, 已知边界$x = 0$和$y = 0$上的电势都是零, 而边界$x = a$上的电势为$u_0$(常数).
\end{yyEx}

\section{习题16}

\begin{yyEx}
	在扇形区域内求解下列定解问题
	\begin{equation*}
		\grad^2i = 0;~~u\big|_{\varphi = 0} = u\big|_{\varphi = \alpha} = 0;~~u\big|_{\rho = R} = f(\varphi).
	\end{equation*}
\end{yyEx}


\chapter{二阶常微分方程的级数解法~~本征值问题}

\section{习题1-5}

\begin{yyEx}
	在$x = 0$的邻域内, 求解方程$(1-x^2)y''+xy'-y = 0$.
\end{yyEx}

\begin{yyEx}
	在$x = 0$的邻域内求解方程$y'' + y = 0$.
\end{yyEx}

\begin{yyEx}
	在$x = 0$的邻域内求解艾里方程$y'' - xy = 0$.
\end{yyEx}

\begin{yyEx}
	求方程$x^2y''(x)-xy'(x)+y = 0$在$x = 0$的邻域内的通解.
\end{yyEx}

\begin{yyEx}
	将下列方程化为施图姆-刘维尔型方程的标准形式:
	\begin{equation*}
		(1)y''-\cot xy'+\lambda y = 0;~~(2) xy''+(1-x)y'+\lambda y = 0.
	\end{equation*}
\end{yyEx}

\section{习题6-10}

\begin{yyEx}
	求解下列本征值问题的本征值和本征函数:
	\begin{enumerate}
		\item $X''(x) + \lambda X(x) = 0, X(0) = 0 , X'(l) = 0$;\\
		\item $X''(x) + \lambda X(x) = 0, X'(0) = 0 , X(l) = 0$;\\
		\item $X''(x) + \lambda X(x) = 0, X(0)+HX'(0) = 0 , X(l) = 0$(H\text{为常数});\\
		\item $\begin{lgathered}
			\frac{\mathrm{d}}{r\mathrm{d}r}\left( r\frac{\mathrm{d}R}{\mathrm{d}r} \right)+\frac{\lambda}{r^2}R = 0,R(a) = 0,R(b) = 0,~~0<a<b.
		\end{lgathered}$
	\end{enumerate}
\end{yyEx}

\begin{yyEx}
	已知二阶线性常微分方程的两个线性无关解$y_1(x) = e^{a/x}$和$y_2(x) = e^{-a/x}$, 求其所满足的方程.
\end{yyEx}

\begin{yyEx}
	在$x = 0$的邻域内求解方程$y'' - 2xy' + (\lambda-1)y = 0$, 当$\lambda$取什么数值时可使级数退化为多项式.
\end{yyEx}

\begin{yyEx}
	求合流超几何方程$xy''(x)+(\gamma -x)y'(x) - \alpha y(x) = 0$在$x = 0$附近的通解, 其中$\alpha,\gamma$为常数, 且$\alpha>0,1-\gamma\neq 0,1,2,\cdots$.
\end{yyEx}

\begin{yyEx}
	证明在下列有界条件下的本征值问题中, 本征函数是正交的.
	\begin{equation*}
		\begin{dcases}
			&\frac{\mathrm{d}}{\mathrm{d}x}\left[ k(x)\frac{\mathrm{d}y(x)}{\mathrm{d}x} \right] + \lambda y(x) = 0,~~x\in(a,b), \\
			&\abs{y(a)}<+\infty,~~\abs{y(b)}<+\infty,
		\end{dcases}
	\end{equation*}
	其中$k(x)$为非负的连续实函数, 且$k(a) = k(b) = 0$.
\end{yyEx}

\chapter{贝塞尔函数及其应用}

\section{习题1-5}

\begin{yyEx}
	试用平面极坐标系把二维波动方程分离变量:
	\begin{equation*}
	u_{tt} - a^2(u_{xx}+u_{yy}) = 0,
	\end{equation*}
	即得到各相关单元函数所满足的常微分方程.
\end{yyEx}

\begin{yyEx}
	写出$\mathrm{J}_0(x),\mathrm{J}_1(x),\mathrm{J}_2(x)$ ($n$为正整数)级数表达式的前$5$项.
\end{yyEx}

\begin{yyEx}
	证明$\mathrm{J}_{2n-1}(0) = 0$, 其中$n = 1,2,3,\cdots.$
\end{yyEx}

\begin{yyEx}
	证明$y = \mathrm{J}_n(\alpha x)$为方程$x^2y''+xy'+(\alpha^2x^2-n^2)y = 0$的解.
\end{yyEx}

\begin{yyEx}
	试证$y = x^{1/2}\mathrm{J}_{3/2}(x)$是方程$x^2y''+(x^2-2)y = 0$的一个解.
\end{yyEx}



\section{习题6-10}

\begin{yyEx}
	试证$y = x\mathrm{J}_n(x)$是方程$x^2y''=xy'+(1+x^2-n^2)y = 0$的一个解.
\end{yyEx}

\begin{yyEx}
	利用递推公式证明:\begin{equation*}
	(1) \mathrm{J}_2(x) = \mathrm{J}_0''(x) -\frac{1}{x}\mathrm{J}_0'(x);~~(2)\mathrm{J}_3(x) +3 \mathrm{J}_0'(x) +4\mathrm{J}_0'''(x) = 0.
	\end{equation*}
\end{yyEx}

\begin{yyEx}
	求解半径为$R$、固定边界的圆形薄膜的轴对称振动问题, 设$t = 0$时在膜上$\rho\leqslant\varepsilon$处有一冲量的垂直作用.
\end{yyEx}

\begin{yyEx}
	半径为$b$的圆形薄膜, 边缘固定, 初始形状是旋转抛物面\begin{equation*}
	u\big|_{t = 0} = (1-\rho^2/b^2)H,
	\end{equation*}
	初始速度分布为零. 求解膜的振动情况.
\end{yyEx}

\begin{yyEx}
	一均匀无限长圆柱体, 体内无热源, 通过柱体表面沿法向的热量为常数$q$, 若柱体的初始温度也为常数$u_0$, 求任意时刻柱体的温度分布.
\end{yyEx}

\section{习题11-13}

\begin{yyEx}
	圆柱空腔内电磁振荡的定解问题为\begin{equation*}
		\begin{dcases}
			&\nabla^2u + \lambda u = 0,~~\sqrt{\lambda} = \frac{\omega}{c},\\
			&u\big|_{\rho = a} = 0,\\
			&\frac{\partial u}{\partial z}\bigg|_{z = 0} = \frac{\partial u}{\partial z}\bigg|_{z = l} = 0. 
		\end{dcases}
	\end{equation*}
	试证电磁振荡的固有频率为\begin{equation*}
		\omega_{mn} = c\sqrt{\lambda} = c\sqrt{\left( \frac{x_m^{(0)}}{a} \right)^2 + \left( \frac{n\pi}{l} \right)^2},~~n = 0,1,2,\cdots;m = 1,2,\cdots.
	\end{equation*}
\end{yyEx}

\begin{yyEx}
	半径为$R$、高为$H$的圆柱内无电荷, 柱体下底和柱面保持零电位, 上底电位为$f(\rho) = \rho^2$, 求柱体内各内点的电位分布. 定解问题为(取柱坐标)\begin{equation*}
		\begin{dcases}
			\nabla^2u = 0,&~\\
			u\big|_{z = 0} = 0,&u\big|_{z = H} = \rho^2,\\
			u\big|_{\rho = 0} \neq \infty,&u\big|_{\rho = R} = 0.
		\end{dcases}
	\end{equation*}
\end{yyEx}

\begin{yyEx}
	圆柱体半径为$R$、 高为$H$, 上底保持温度$u_1$, 下底保持温度$u_2$, 侧面保持温度分布为
	\begin{equation*}
		f(z) = \frac{2u_1}{H}\left( z-\frac{H}{2} \right)z+\frac{2u_2}{H}\left( H-z \right),
	\end{equation*}
	求柱内各点的稳定温度分布.
\end{yyEx}



\chapter{勒让德多项式及其应用}

\section{习题1-5}

\begin{yyEx}
	氢原子定态问题的量子力学薛定谔方程是\begin{equation*}
		-\frac{8h^2}{8\pi^2\mu}\nabla^2u-\frac{Ze^2}{r}u = Eu,
	\end{equation*}
	其中$h,\mu,Z,e,E$都是常数, 试在球坐标系下把这个方程分离变量, 即得到相应各单变量函数满足的常微分方程.
\end{yyEx}

\begin{yyEx}
	证明:(1) $\begin{lgathered}
			x^2 = \frac{2}{3}\mathrm{P}_2(x) + \frac{1}{3}\mathrm{P}_0(x);
		\end{lgathered}$~~~(2)
		$\begin{lgathered}
			x^3 = \frac{2}{5}\mathrm{P}_3(x) + \frac{3}{5}\mathrm{P}_1(x).
		\end{lgathered}$
\end{yyEx}

\begin{yyEx}
	求证$\begin{lgathered}
		\int_{-1}^{1}(1-x^2)[\mathrm{P}_n'(x)]^2\mathrm{d}x = \frac{2n(n+1)}{2n+1}.
	\end{lgathered}$
\end{yyEx}

\begin{yyProof}
	利用递推关系\begin{equation*}
		\begin{split}
			&(1-x^2)\mathrm{P}_n'(x) = n[\mathrm{P}_{n-1}(x)-x\mathrm{P}_n(x)]; \\
			&\mathrm{P}_n'(x) = x\mathrm{P}_{n-1}'(x)+n\mathrm{P}_{n-1}(x),
		\end{split}
	\end{equation*}
	可将积分化简为\begin{equation*}
		\begin{split}
		I &\triangleq \int_{-1}^{1}(1-x^2)[\mathrm{P}_n'(x)]^2\mathrm{d}x\\
		 &= \int_{-1}^{1}n[\mathrm{P}_{n-1}(x)-x\mathrm{P}_n(x)]\mathrm{P}_n'(x)\mathrm{d}x\\
		&= \int_{-1}^{1}n\mathrm{P}_{n-1}(x)[x\mathrm{P}_{n-1}'(x)+n\mathrm{P}_{n-1}(x)]-nx\mathrm{P}_n(x)\mathrm{P}_n'(x)\mathrm{d}x.
		\end{split}
	\end{equation*}
	利用分部积分, 不难计算出\begin{equation*}
		\begin{split}
			J_l&\triangleq \int_{-1}^{1}x\mathrm{P}_l(x)\mathrm{P}_l'(x)\mathrm{d}x\\
			 &= \int_{x = -1}^{1}\frac{x}{2}\mathrm{d}[\mathrm{P}_l(x)]^2 \\
			&= 0 - \frac{1}{2}\int_{-1}^1 [\mathrm{P}_l(x)]^2\mathrm{d}x = -\frac{1}{2n+1}.
		\end{split}
	\end{equation*}
	由此计算出最终结果
	\begin{equation*}
	\begin{split}
	I &= \int_{-1}^{1}n\mathrm{P}_{n-1}(x)[x\mathrm{P}_{n-1}'(x)+n\mathrm{P}_{n-1}(x)]-nx\mathrm{P}_n(x)\mathrm{P}_n'(x)\mathrm{d}x \\
	&= nJ_{n-1}-nJ_n +n^2\int_{-1}^1[\mathrm{P}_{n-1}(x)]^2\mathrm{d}x \\
	&= \frac{-n}{2n-1}+\frac{n}{2n+1}+\frac{2n^2}{2n-1} \\
	&= \frac{2n(n+1)}{2n+1}.
	\end{split}
	\end{equation*}
\end{yyProof}

\begin{yyEx}
	证明$\begin{lgathered}
		\mathrm{P}_l(-x) = (-1)^l\mathrm{P}_l(x).
	\end{lgathered}$
\end{yyEx}

\begin{yyProof}
	在以下的Legendre多项式的表达式中
	\begin{equation*}\boxed{
		\mathrm{P}_l(x) = \sum_{r = 0}^{\lfloor l/2\rfloor}(-1)^r\frac{(2l-2r)!}{2^lr!(l-r)!(l-2r)!}x^{l-2r}}
	\end{equation*}
	当$l$是奇数时, $\mathrm{P}_l(x)$由$x$的奇数次幂的线性组合构成, 它是奇函数; 当$l$是偶数时, $\mathrm{P}_l(x)$由$x$的偶数次幂的线性组合构成, 则它是偶函数.
	
	因此,不难看出\begin{equation*}
		\mathrm{P}_l(-x) = (-1)^l\mathrm{P}_l(x).
	\end{equation*}
\end{yyProof}

\begin{yySolution2}
	见习题9.7的解答过程.
\end{yySolution2}

\begin{yyEx}
	已知$\begin{lgathered}
		\mathrm{P}_0(x) = 1,\mathrm{P}_1(x) = x,\mathrm{P}_2(x) = \frac{1}{2}(3x^2-1)
	\end{lgathered}$,用递推公式求$\mathrm{P}_3(x),\mathrm{P}_4(x)$.
\end{yyEx}

\section{习题6-10}

\begin{yyEx}
	在$(-1,1)$上, 将下列函数按勒让德多项式展开为广义傅里叶级数.
	\begin{equation*}
		f(x) = \begin{dcases}
			x,&0<x<1,\\
			0,&-1<x<0.
		\end{dcases}
	\end{equation*}
\end{yyEx}

\begin{yySolution}
	设$f(x) = \sum_{l = 0}^{+\infty}c_l\mathrm{P}_l(x)$, 则
	\begin{equation*}
		c_l = \frac{2l+1}{2}\int_{0}^{1}x\mathrm{P}_l(x)\mathrm{d}x = \frac{2l+1}{2}\int_{0}^{1}\mathrm{P}_1(x)\mathrm{P}_l(x)\mathrm{d}x.
	\end{equation*}
	心中回忆书中公式(9.43),便可计算得到:
	\begin{equation*}
		c_l = \begin{dcases}
			\frac{3}{2}\int_{0}^{1}x^2\mathrm{d}x = \frac{1}{2}, &l = 1,\\
			\frac{2l+1}{2}\frac{-\mathrm{P}_l(0)}{l(l+1)-2},&l\neq 1.
		\end{dcases}
	\end{equation*}
	从书中公式(9.20)可以轻松看出\begin{equation*}
		\mathrm{P}_{2n}(0) = (-1)^n\frac{(2n)!}{2^{2n}(n!)^2},~~\mathrm{P}_{2n+1}(0) = 0, n\in\mathbb{N}.
	\end{equation*}
	由此得到最后的结果\begin{equation*}
		\begin{split}
			f(x) &= \frac{1}{2}\mathrm{P}_1(x) + \sum_{n = 0}^{+\infty}\frac{4n+1}{2}\frac{-(-1)^n\frac{(2n)!}{2^{2n}(n!)^2}}{2n(2n+1)-2}\mathrm{P}_{2n}(x) \\
			&= \frac{1}{2}\mathrm{P}_1(x)+\sum_{n = 0}^{+\infty}\frac{(-1)^{n+1}(4n+1)(2n)!}{2^{2n+2}(2n-1)(n+1)(n!)^2}\mathrm{P}_{2n}(x).
		\end{split}
	\end{equation*}
\end{yySolution}

\begin{yySolution2}
	利用\begin{equation*}
		\boxed{f(x) = \max\{x,0\} = \frac{\abs{x}+x}{2} = \frac{\abs{x}}{2} + \frac{\mathrm{P}_1(x)}{2}}
	\end{equation*}
	和书中P267例1(2)的结论
	\begin{equation*}
		\boxed{
			\abs{x} = \frac{1}{2}\mathrm{P}_0(x) + \sum_{n = 1}^{+\infty}\frac{(-1)^{n+1}(4n+1)(2n-2)!}{2^{2n}(n+1)!(n-1)!}\mathrm{P}_{2n}(x)
		}
	\end{equation*}
	直接得到结论:
	\begin{equation*}
		f(x) = \frac{1}{4}\mathrm{P}_0(x)+\frac{1}{2}\mathrm{P}_1(x) +\frac{1}{2}\sum_{n = 1}^{+\infty}\frac{(-1)^{n+1}(4n+1)(2n-2)!}{2^{2n}(n+1)!(n-1)!}\mathrm{P}_{2n}(x).
	\end{equation*}
\end{yySolution2}

\begin{yyEx}
	利用勒让德多项式的生成函数(母函数)证明:\begin{equation*}
		\mathrm{P}_n(-1) = (-1)^n,~~\mathrm{P}_{2n-1}(0) = 0,~~\mathrm{P}_{2n}(0) = \frac{(-1)^n(2n)!}{2^{2n}(n!)^2}.
	\end{equation*}
\end{yyEx}

\begin{yyProof}
	根据Legendre多项式的\myind{生成函数}:
	\begin{equation*}
		\boxed{
			\frac{1}{\sqrt{1-2xt+t^2}} = \sum_{l = 0}^{+\infty}\mathrm{P}_l(x)t^l,~~\abs{t}<\abs{x\pm \sqrt{x^2-1}}.
		}
	\end{equation*}
	令$x = -1$, 就得到
	\begin{equation*}
		\frac{1}{\sqrt{1+2t+t^2}} = \frac{1}{1+t} = \sum_{n=0}^{+\infty}(-1)^nt^n = \sum_{n = 0}^{+\infty}\mathrm{P}_n(-1)t^n
	\end{equation*}
	这说明$\mathrm{P}_n(-1) = (-1)^n$.
	
	再根据,
	\begin{equation*}
		\frac{1}{\sqrt{1-2xt+t^2}} = \frac{1}{\sqrt{1-2(-x)(-t)+(-t)^2}}
	\end{equation*}
	即\begin{equation*}
		\sum_{l = 0}^{+\infty}\mathrm{P}_l(x)t^l = \sum_{l = 0}^{+\infty}\mathrm{P}_l(-x)(-t)^l
	\end{equation*}
	这说明了Legendre多项式的奇偶性, 即$\mathrm{P}_l(-x) = (-1)^l\mathrm{P}_l(x)$.
	
	这蕴含结论$\mathrm{P}_{2n-1}(0) = 0$.
	
	再代入$x = 0$, 有\begin{equation*}
		\frac{1}{\sqrt{1+t^2}} = \sum_{l = 0}^{+\infty}\mathrm{P}_l(0)t^l,
	\end{equation*}
	两端对$t$微商, 得\begin{equation*}
		-\frac{t}{1+t^2}\frac{1}{\sqrt{1+t^2}}\sum_{l = 1}^{+\infty}l\mathrm{P}_l(0)t^{l-1}.
	\end{equation*}
	结合以上两式, 我们有
	\begin{equation*}
		-t\sum_{l = 1}^{+\infty}l\mathrm{P}_l(0)t^{l-1} = (1+t^2)\sum_{l = 0}^{+\infty}\mathrm{P}_l(0)t^l.
	\end{equation*}
	比较两端各项系数, 并稍加整理得:
	\begin{equation*}
		\mathrm{P}_{l+1}(0) = -\frac{l}{l+1}\mathrm{P}_{l-1}(0).
	\end{equation*}
	再利用条件$\mathrm{P}_0(0) = 1$(此条件可在生成多项式中令$t = 0$立得), 可计算出\begin{equation*}
		\begin{split}
			\mathrm{P}_{2n}(0) &= (-1)^n\frac{(2n-1)(2n-3)\cdots 1}{(2n)(2n-2)\cdots 2}\\
			&= (-1)^n\frac{(2n)!}{[(2n)(2n-2)\cdots 2]^2} \\
			&=\frac{(-1)^n(2n)!}{2^{2n}(n!)^2}.
		\end{split}
	\end{equation*}
\end{yyProof}

\begin{yyEx}
	在半径为$1$的球内求解拉普拉斯方程$\nabla^2u = 0$, 使$u|_{r = 1} = 3\cos 2\theta+1$.
\end{yyEx}

\begin{yyEx}
	在半径为$1$的球内求解拉普拉斯方程$\nabla^2u = 0$, 已知在球面上
	\begin{equation*}
		u\big|_{r = 1} = \begin{dcases}
			A,&0\leqslant \theta\leqslant \alpha,\\
			0,&\alpha<\theta\leqslant\pi.
		\end{dcases}
	\end{equation*}
\end{yyEx}

\begin{yyEx}
	在半径为$1$的球外求解拉普拉斯方程$\nabla^2u = 0$, 使$u|_{r = 1} = \cos\theta$.
\end{yyEx}

\section{习题11-12}

\begin{yyEx}
	在半径为$a$的球外$(r>a)$求解:
	\begin{equation*}
		\begin{dcases}
			&\nabla^2u = 0,\\
			&u\big|_{r = a} = f(\theta,\phi).
		\end{dcases}
	\end{equation*}
\end{yyEx}

\begin{yyEx}
	(辐射速度势问题)设半径为$r_0$ 的球面径向速度分布为 \begin{equation*}
		v = v_0\frac{1}{4}(3\cos 2\theta+1)\cos wt,
	\end{equation*}这个球在空气中辐射出去的声场中的速度势满足三维波动方程: $v_{tt}-a^2\nabla^2v = 0$, 其中$a^2 = \frac{p_0r}{\rho_0}$, $p_0$是初始压强, $\rho_0$是初始密度, $r$是定压比热容的比值. 设$r_0\leqslant \lambda$(声波长), 求速度势$v$, 当$r$很大时 $v|_{r\to\infty}$的渐进表达式是什么?
\end{yyEx}
\chapter{概率不等式}
概率不等式是概率论的理论研究中的重要工具. 有力的概率不等式, 常常是创建一个定理的重要依据. 对于概率的极限定理等, 几乎所有的重要结果的论证或是借助于建立一些概率不等式, 或是通过巧妙应用已有的概率不等式. 在此, 我们将概率论中常见的一些基本不等式分类罗列如下.

\section{随机事件的不等式}

\section{分布函数、特征函数的不等式}

\section{随机变量的概率不等式}

\section{矩不等式}

\section{随机变量的概率与矩之间的不等式}

\section{部分和及其极值的矩估计}

\section{指数不等式}



\chapter{格林函数法}

\section{习题1-5}

\begin{yyEx}
	求解上半平面的狄利克雷问题
	\begin{equation*}
		\begin{dcases}
			u_{xx}+u_{yy} = 0,&y>0,\\
			u|_{y=0} = f(x).&~
		\end{dcases}
	\end{equation*}
\end{yyEx}

\begin{yyEx}
	求解上半空间的狄利克雷问题
	\begin{equation*}
	\begin{dcases}
	\nabla^2u = 0,&z>0,\\
	u|_{z=0} = f(x,y).&~
	\end{dcases}
	\end{equation*}
\end{yyEx}

\begin{yyEx}
	\begin{enumerate}
		\item 用电像法求出圆域泊松方程的格林函数$G(x,y;x_0,y_0)$, $M_0(x_0,y_0)$是圆内的一点, $G$满足
		\begin{equation*}
			\nabla^2G = \delta(x-x_0)\delta(y-y_0),~~G|_{\rho = a} = 0.
		\end{equation*}
		\item 在圆形域$\rho\leqslant a$上求拉普拉斯方程第一边值问题:
		\begin{equation*}
		\begin{dcases}
			\nabla^2u = 0,&\rho\leqslant a,\\
			u|_{\rho = a} = f(\varphi).&~
		\end{dcases}
		\end{equation*}
		\item 在圆形域$\rho\leqslant a$上求解$\nabla^2u = 0$,使满足边界条件$u|_{\rho = a} = A\cos\varphi$.
	\end{enumerate}
\end{yyEx}

\begin{yyEx}
	求区域$0\leqslant x<\infty,0\leqslant y<\infty$上的格林函数,并由此求解狄利克雷问题:
	\begin{equation*}
	\begin{dcases}
	u_{xx}+u_{yy} = 0,&~\\
	u(0,y) = f(y),&0\leqslant y\leqslant\infty\\
	u(x,0) = 0,&0\leqslant y\leqslant\infty,
	\end{dcases}
	\end{equation*}
	其中$f$为已知的连续函数,且$f(0) = 0$.
\end{yyEx}

\begin{yyEx}
	求矩形区域内泊松方程的狄利克雷边值问题的格林函数
	\begin{equation*}
	\begin{dcases}
	&\nabla^2G = -\delta(x-x_0)\delta(y-y_0),~\\
	&G|_{x= 0} = 0,~~G|_{x = a} = 0,\\
	&G|_{y= 0} = 0,~~G|_{x = b} = 0.
	\end{dcases}
	\end{equation*}
\end{yyEx}



\clearpage{\pagestyle{empty}\cleardoublepage}
\backmatter
\phantomsection


\end{document}
