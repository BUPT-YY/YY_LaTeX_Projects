\documentclass[aspectratio=43]{beamer}
\usepackage[english]{babel}
\usepackage[fntef]{ctex} % invole CJKfntef
\usepackage{ctex}
\usepackage{fontspec}
\setmainfont{CMU Serif}

\usepackage{mathrsfs}
\usepackage{amssymb,amsmath, amsthm,bm}
\usepackage{mathtools}
\input{chapters/preamble}
\title{笔试参考解答} %->->->->-> Check hyperref title <-<-<-<-<-
\author[北邮数学人学术部]{北邮数学人学术部}
\institute[BUPT]{
	Beijing University of Posts and Telecommunications%
} %You can change the Institution if you are from somewhere else
\date{Oct 17th, 2019}
%\logo{\includegraphics[width= 0.2\textwidth]{images/a-logo.png}}

\begin{document}
	
	\frame{\titlepage}
	
   \section{}
\begin{frame}{\textbf{题目1}}
\begin{block}[<+->]{Proof}
		不妨设$x>y>0$, 可在原不等式两端各项同除以$y$, 并令$t = x/y$($t>1$). 则必须且只需证明:
		\begin{equation*}
			\sqrt{t}\leqslant \frac{t-1}{\ln t}\leqslant\frac{t+1}{2}.
		\end{equation*}
		这等价于\begin{equation*}
			\frac{2(t-1)}{t+1}\leqslant \ln t\leqslant \sqrt{t}-\frac{1}{\sqrt{t}}.
		\end{equation*}
		记$f(r) = \ln r-\dfrac{2(r-1)}{r+1},r\geqslant 1$, 则
		\begin{equation*}
			f'(r) = \frac{(r-1)^2}{r(r+1)^2}\geqslant 0,
		\end{equation*}
		所以$f(t)\geqslant f(1) = 0$.\\
		再记$g(r) = \ln r - \sqrt{r}+\dfrac{1}{\sqrt{r}}$,
		则\begin{equation*}
		g'(r) = -\frac{(\sqrt{r}-1)^2}{2r\sqrt{r}}\leqslant 0,
		\end{equation*}
		这说明$g(t)\leqslant g(1) = 0$.
		
		综上, 原不等式得证.
	\end{block}
\end{frame}

\begin{frame}
	
\end{frame}

\begin{frame}{\textbf{题目2}}
求极限
\begin{itemize}
	\item
	\begin{equation*}
	\lim_{n\to\infty}\frac{1!+2!+\cdots+n!}{n!}.
	\end{equation*}
	\item
	\begin{equation*}
	\lim_{x\to 0}\frac{\sin(\ln(1+x)))-\ln(1+\sin x)}{\arcsin(\mathrm{e}^x-1)-\mathrm{e}^{\arcsin{x}}+1}.
	\end{equation*}

\end{itemize}

\end{frame}

\begin{frame}{\textbf{题目3}}
	用“$\varepsilon-\delta$ language”证明:\begin{equation*}
	\lim_{x\to 1}\frac{x^2-1}{3x^2-7x+4} = -2.
	\end{equation*}
\end{frame}

\begin{frame}{\textbf{题目4}}
	设$f(x)$在$[0,1]$上二阶可导, $f(0) = f(1) = 0$, $f'(0)f'(1)>0$, 求证:
	\begin{equation*}
	\exists \xi\in(0,1), \text{使}f''(\xi) = 0.
	\end{equation*}
\end{frame}

\begin{frame}{\textbf{题目5}}
	设$S$为有上界的非空数集, 并且上确界$\sup S\notin S$. 求证: 存在严格单调递增的数列$\{x_n\}_{n=1}^{\infty}\subseteq S$, 使得$\lim\limits_{n\to\infty}x_n = \sup S$.
	
\end{frame}

\begin{frame}{\textbf{题目6}}
	证明:函数$x^5+ax^4+bx^3+c$($c\neq 0$)不可能有5个不同的实根
	
\end{frame}

\begin{frame}{\textbf{题目7}}
	设函数$f(x)$在$[0,+\infty)$上一致连续, 且对任意$x>0$有
	\begin{equation}
	\lim_{n\to\infty}f(x+n) = 0.
	\end{equation}
	证明:$\lim\limits_{x\to+\infty}f(x) = 0$.
	
\end{frame}

\begin{frame}{\textbf{题目8}}
	通过计算求出$a$的值, 使得$x-y+a = 0$是函数
	\begin{equation*}
	f(x) = \frac{x^3}{x^2+2x-1}
	\end{equation*}
	的渐近线.
\end{frame}

\begin{frame}{\textbf{题目9}}
	计算\begin{equation*}
	\begin{vmatrix}
	x_1 &  x_2 &  x_3 &  x_4\\
	x_2 & -x_1 & -x_4 &  x_3\\
	x_3 &  x_4 & -x_1 & -x_2\\
	x_4 & -x_3 &  x_2 & -x_1\\
	\end{vmatrix}\cdot\begin{vmatrix}
	y_1 &  y_2 &  y_3 &  y_4\\
	y_2 & -y_1 & -y_4 &  y_3\\
	y_3 &  y_4 & -y_1 & -y_2\\
	y_4 & -y_3 &  y_2 & -y_1\\
	\end{vmatrix}
	\end{equation*}
	并证明Euler恒等式:\begin{equation*}
	\begin{split}
	(x_1^2&+x_2^2+x_3^2+x_4^2)(y_1^2+y_2^2+y_3^2+y_4^2)\\
	&=(x_1y_1+x_2y_2+x_3y_3+x_4y_4)^2\\
	&~~+(x_1y_1-x_2y_1-x_3y_4+x_4y_3)^2\\
	&~~+(x_1y_3+x_2y_4-x_3y_1-x_4y_2)^2\\
	&~~+(x_1y_4-x_2y_4+x_3y_2-x_4y_1)^2.
	\end{split}
	\end{equation*}
\end{frame}

\begin{frame}{\textbf{题目10}}
	设$\mathbb{F}$是一个数域. 定义数域$\mathbb{F}$上的两个一元多项式:
	\begin{equation*}
	\begin{split}
	f(x) &= x_0(x-x_1)\cdots(x-x_n)= a_0x^n+a_1x^{n-1}+\cdots+a_n,\\
	g(x) &=y_0(x-y_1)\cdots(x-y_m)= b_0x^m+b_1x^{m-1}+\cdots+b_m.
	\end{split}
	\end{equation*}
	设$\bm{A}$是如下的一个$m+n$阶的矩阵:
	\begin{equation*}
	\bm{A} = \begin{bmatrix}
	a_0 & a_1 & a_2 & \cdots & a_n & ~ & ~ & ~\\
	~   & a_0 & a_1 & a_2 & \cdots & a_n & ~ & ~\\
	~   & ~   & \ddots & \ddots & \ddots & ~ & \ddots & ~\\
	~   & ~   & ~ & a_0 & a_1 & a_2 & \cdots & a_n\\
	b_0 & b_1 & b_2 & \cdots & b_m & ~ & ~ & ~\\
	~   & b_0 & b_1 & b_2 & \cdots & b_m & ~ & ~\\
	~   & ~   & \ddots & \ddots & \ddots & ~ & \ddots & ~\\
	~   & ~   & ~ & b_0 & b_1 & b_2 & \cdots & b_m\\
	\end{bmatrix}\begin{array}{l}\left.\rule{0mm}{12mm}\right\}m\text{行}\\
	\\\left.\rule{0mm}{12mm}\right\}n\text{行}
	\end{array}
	\end{equation*}
	
\end{frame}

\begin{figure}
	证明:\begin{equation*}
	\det(\bm{A}) = x_0^m\prod_{i=1}^{n}g(x_i) = (-1)^{mn}y_0^n\prod_{j=1}^{m}f(y_j).
	\end{equation*}
\end{figure}



\section{}
\begin{frame}{}
\centering
\Huge\bfseries
\textcolor{orange}{谢~~谢}
\end{frame}
\end{document}
