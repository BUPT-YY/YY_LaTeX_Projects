%%%%%%%%%%%%%%%%%%%%%%%%%%%%%%%%%%%%%%%%%
\documentclass[12pt]{article}

\usepackage[fntef]{ctex} % invole CJKfntef
\usepackage{ctex}

\usepackage{amsmath,amssymb,amsfonts,bm,amsthm}
\usepackage{mathtools}
\usepackage{physics}
\usepackage{tensor}

\setlength\parindent{0pt} % Removes all indentation from paragraphs

\usepackage[most]{tcolorbox} % Required for boxes that split across pages

\usepackage{booktabs} % Required for better horizontal rules in tables

\usepackage{listings} % Required for insertion of code

\usepackage{etoolbox} % Required for if statements

%----------------------------------------------------------------------------------------
%	MARGINS
%----------------------------------------------------------------------------------------

\usepackage{geometry} % Required for adjusting page dimensions and margins

\geometry{
	paper=a4paper, % Change to letterpaper for US letter
	top=3cm, % Top margin
	bottom=3cm, % Bottom margin
	left=2.5cm, % Left margin
	right=2.5cm, % Right margin
	headheight=14pt, % Header height
	footskip=1.4cm, % Space from the bottom margin to the baseline of the footer
	headsep=1.2cm, % Space from the top margin to the baseline of the header
	%showframe, % Uncomment to show how the type block is set on the page
}

%----------------------------------------------------------------------------------------
%	FONT
%----------------------------------------------------------------------------------------

\usepackage[utf8]{inputenc} % Required for inputting international characters
\usepackage[T1]{fontenc} % Output font encoding for international characters

\usepackage[sfdefault,light]{roboto} % Use the Roboto font

%----------------------------------------------------------------------------------------
%	HEADERS AND FOOTERS
%----------------------------------------------------------------------------------------

\usepackage{fancyhdr} % Required for customising headers and footers

\pagestyle{fancy} % Enable custom headers and footers

\lhead{\small\assignmentClass\ifdef{\assignmentClassInstructor}{\assignmentClassInstructor}{}\ \assignmentTitle} % Left header; output the instructor in brackets if one was set
\chead{} % Centre header
\rhead{\small\ifdef{\assignmentAuthorName}{\assignmentAuthorName}{\ifdef{\assignmentDueDate}{\assignmentDueDate}{}}} % Right header; output the author name if one was set, otherwise the due date if that was set

\lfoot{} % Left footer
\cfoot{\small Page\ \thepage\ } % Centre footer
\rfoot{} % Right footer

\renewcommand\headrulewidth{0.5pt} % Thickness of the header rule
\newcommand{\myind}[1]{{\upshape\color{blue} #1 }\index{#1}}
%----------------------------------------------------------------------------------------
%	MODIFY SECTION STYLES
%----------------------------------------------------------------------------------------

\usepackage{titlesec} % Required for modifying sections

%------------------------------------------------
% Section

\titleformat
{\section} % Section type being modified
[block] % Shape type, can be: hang, block, display, runin, leftmargin, rightmargin, drop, wrap, frame
{\Large\bfseries} % Format of the whole section
{\assignmentQuestionName~\thesection} % Format of the section label
{6pt} % Space between the title and label
{} % Code before the label

\titlespacing{\section}{0pt}{0.5\baselineskip}{0.5\baselineskip} % Spacing around section titles, the order is: left, before and after

%------------------------------------------------
% Subsection

\titleformat
{\subsection} % Section type being modified
[block] % Shape type, can be: hang, block, display, runin, leftmargin, rightmargin, drop, wrap, frame
{\itshape} % Format of the whole section
{(\alph{subsection})} % Format of the section label
{4pt} % Space between the title and label
{} % Code before the label

\titlespacing{\subsection}{0pt}{0.5\baselineskip}{0.5\baselineskip} % Spacing around section titles, the order is: left, before and after

\renewcommand\thesubsection{(\alph{subsection})}

%----------------------------------------------------------------------------------------
%	CUSTOM QUESTION COMMANDS/ENVIRONMENTS
%----------------------------------------------------------------------------------------

% Environment to be used for each question in the assignment
\newenvironment{question}{
	\vspace{0.5\baselineskip} % Whitespace before the question
	\section{} % Blank section title (e.g. just Question 2)
	\lfoot{\small\itshape\assignmentQuestionName~\thesection~continued on next page\ldots} % Set the left footer to state the question continues on the next page, this is reset to nothing if it doesn't (below)
}{
	\lfoot{} % Reset the left footer to nothing if the current question does not continue on the next page
}

%------------------------------------------------

% Environment for subquestions, takes 1 argument - the name of the section
\newenvironment{subquestion}[1]{
	\subsection{#1}
}{
}

%------------------------------------------------

% Command to print a question sentence
\newcommand{\questiontext}[1]{
	\textbf{#1}
	\vspace{0.5\baselineskip} % Whitespace afterwards
}

%------------------------------------------------

% Command to print a box that breaks across pages with the question answer
\newcommand{\answer}[1]{
	\begin{tcolorbox}[breakable, enhanced]
		#1
	\end{tcolorbox}
}

%------------------------------------------------

% Command to print a box that breaks across pages with the space for a student to answer
\newcommand{\answerbox}[1]{
	\begin{tcolorbox}[breakable, enhanced]
		\vphantom{L}\vspace{\numexpr #1-1\relax\baselineskip} % \vphantom{L} to provide a typesetting strut with a height for the line, \numexpr to subtract user input by 1 to make it 0-based as this command is
	\end{tcolorbox}
}

%------------------------------------------------

% Command to print an assignment section title to split an assignment into major parts
\newcommand{\assignmentSection}[1]{
	{
		\centering % Centre the section title
		\vspace{2\baselineskip} % Whitespace before the entire section title
		
		\rule{0.8\textwidth}{0.5pt} % Horizontal rule
		
		\vspace{0.75\baselineskip} % Whitespace before the section title
		{\LARGE \MakeUppercase{#1}} % Section title, forced to be uppercase
		
		\rule{0.8\textwidth}{0.5pt} % Horizontal rule
		
		\vspace{\baselineskip} % Whitespace after the entire section title
	}
}

%----------------------------------------------------------------------------------------
%	TITLE PAGE
%----------------------------------------------------------------------------------------

\author{\textbf{\assignmentAuthorName}} % Set the default title page author field
\date{} % Don't use the default title page date field

\title{
	\thispagestyle{empty} % Suppress headers and footers
	\vspace{0.2\textheight} % Whitespace before the title
	\textbf{\assignmentClass:\ \assignmentTitle}\\[-4pt]
	\ifdef{\assignmentDueDate}{{\small Due\ on\ \assignmentDueDate}\\}{} % If a due date is supplied, output it
	\ifdef{\assignmentClassInstructor}{{\large \textit{\assignmentClassInstructor}}}{} % If an instructor is supplied, output it
	\vspace{0.32\textheight} % Whitespace before the author name
}


%----------------------------------------------------------------------------------------
%	ASSIGNMENT INFORMATION
%----------------------------------------------------------------------------------------

% Required
\newcommand{\assignmentQuestionName}{Question} % The word to be used as a prefix to question numbers; example alternatives: Problem, Exercise
\newcommand{\assignmentClass}{} % Course/class
\newcommand{\assignmentTitle}{每日一题~~\today} % Assignment title or name
\newcommand{\assignmentAuthorName}{YY} % Student name

% Optional (comment lines to remove)
\newcommand{\assignmentClassInstructor}{} % Intructor name/time/description
\newcommand{\assignmentDueDate}{} % Due date

%----------------------------------------------------------------------------------------

\begin{document}

\begin{question}

\questiontext{
	在Zermelo-Fraenkel集合论体系下叙述什么是 自然数集$\mathbb{N}$、整数环$\mathbb{Z}$、有理数域$\mathbb{Q}$和实数域$\mathbb{R}$.
}

\end{question}

以下是关于ZF集合论的简介:

最常用的集合论公理系统是ZF及其保守扩张GB(这里Z,F,G,B分别是其发明人Zermelo, Fraenkel, Gödel 以及 Bernays 名字的第一个字母)
\\ \hspace*{\fill} \\%空行
\myind{ZF}是\myind{Zermelo-Fraenkel集合论}(Zermelo-Fraenkel set theory)的简称,包括九条公理,每条公理是一个逻辑命题.注意ZF系统的所有命题中提到的对象都可以是集合,包括其中提到集合的元素时,这些元素课都可以是集合.仔细观察以下不难发现,这些公理中“某某集合存在”表达的其实都是“某某方式构造是对象算是集合”的意思.
\\ \hspace*{\fill} \\%空行
\myind{外延公理}(axiom of extensionality) 如果$x\in X$当且仅当$x\in Y$,则$X=Y$,即集合由其元素完全决定
\\ \hspace*{\fill} \\%空行
\myind{无序对公理}(axiom of pairing)$\forall x,y$,存在集合$\{x,y\}$(无序对),使得$z\in \{x,y\}$当且仅当$z=x$或$z=y$.

Note:这里允许$x=y$,此时$\{x,y\}=\{x\}$.能区分$x,y$地位的有序对则定义为集合
\begin{equation}
(x,y) = \{\{x\},\{x,y\}\}.
\end{equation}
\myind{并集公理}(axiom of union)$\forall \mathcal{A}$,存在集合$\cup\mathcal{A}$,使得$x\in\cup\mathcal{A}$当且仅当存在$A\in\mathcal{A}$满足$x\in A$.

结合无序对公理和并集公理可以归纳定义无序多元组如下:
\begin{equation}
\{x_1,\cdots,x_{n+1} \} = \{x_1,\cdots,x_{n} \} \cup \{x_{n+1}\}.
\end{equation}

\myind{幂集公理}(axiom of power set)$\forall X$,存在集合$2^{X}$(幂集),使得$A\in 2^{X}$当且仅当 $A\subset X$.这里$A\subset X$是逻辑命题
\begin{equation}
\forall x\in A\Rightarrow x\in X
\end{equation}
(即$A$是$X$的子集)的简写.
\\ \hspace*{\fill} \\%空行
\myind{分离公理}(axiom schema of separation或axiom schema of specification) 任取集合$X$以及关于$x$的命题$P(x)$,存在集合$A$使得$x\in A$当且仅当$x\in X$并且$P(x)$成立.集合$A$通常记为$\{x\in X|P(x)\}$.

结合分离公理和并集公理,就可以把交集定义为
\begin{equation}
\cap \mathcal{A} = \{x\in\cup\mathcal{A}|\forall A\in\mathcal{A},x\in A \}.
\end{equation}

注意到当$x\in X,y\in Y$时,前面定义的有序对$(x,y)$其实是一个由$X\cup Y$的子集构成的集合,这样的集合是$2^{X\cup Y}$的子集,也就是$2^{(2^{X\cup Y})}$的元素,因此利用分离公理,可以把直积(direct product)定义为
\begin{equation}
X\times Y = \{ z\in 2^{(2^{X\cup Y})}| \exists x\in X,y\in Y,\text{使得}z=(x,y) \}.
\end{equation}
直积也称作笛卡儿积(Cartesian product),其中英文Cartesian其实是将Descartes(笛卡儿)的名字形容词化之后的结果.

每个从$X$射到$Y$的映射$f:X\rightarrow Y$可以用它在$X\times Y$中的函数图像来刻画,因此也不需要添加新的公理,就可以把从$X$打到$Y$的映射定义为集合
\begin{equation}
Y^X = \{f\in 2^{X\times Y} | \forall x\in X,\text{存在唯一}\ y\in Y\ \text{使得}(x,y)\in f \}
\end{equation}
的元素.并且此时如果$(x,y)\in f$,则把$y$记为$f(x)$.

\myind{空集公理}(axiom of empty set) 存在集合$\emptyset$(空集),使得$\forall x,x\notin \emptyset$.

由外延公理可以知道空集是唯一的.你也许会认为空集公理是多余的,因为有了分离公理可以随便取个集合,再取一个该集合中元素永远不满足的性质,然后就可以构造出空集来了,比如说$\emptyset=\{x\in A|x\notin A\}$.很多的书籍上也确实是这么写的,但是这实际上依赖于另外的一条其它公理中没有提到的假设:在这个世界上确实至少存在着那么一个集合$A$.缺少这个假设,形式逻辑的推理过程就没有了起点.当然,也有群体认为下面的一条公理"无穷公理"本身就包含了一定存在一个空集的意思.


\myind{无穷公理}(axiom of infinity) 存在一个集合$\omega$(含有无穷多个元素的集合),使得$\emptyset\in\omega$,并且$x\in\omega$蕴含$x\cup\{x\}\in\omega$.

在集合论中,每个自然数$n$被归纳地定义成一个恰好有$n$个元素的特殊集合:
\begin{equation}
0=\emptyset,1=\{0\},\cdots,n+1=\{0,\cdots,n\} = n\cup\{n\}.
\end{equation}	

比较一下这个定义和上述无穷公理不难看出,自然数集$\mathbb{N}$就是满足无穷公理的最小集合.

\myind{替换公理}(axiom schema of replacement) 任取集合$X$以及关于$x,y$的逻辑命题$R(x,y)$, 如果$R$满足$\forall x\in X$, 存在唯一$y$使得该命题成立,则存在集合$Y$,使得$y\in Y$当且仅当存在一个$x\in X$使得$R(x,y)$成立.
\\ \hspace*{\fill} \\%空行
如果我们把$R(x,y)$理解成一个映射$f:x\mapsto y$,那么替换公理构造的$Y$其实就是$X$的象集$f(X)$.
\\ \hspace*{\fill} \\%空行
\myind{正则公理}(axiom of regularity)任取非空集合$X$,其中元素关于$\in$关系存在一个极小元素,即存在$x\in X$,使得$\forall y\in X,y\notin x$.
\\ \hspace*{\fill} \\%空行
注意:极小元素的选取并不一定唯一,也不一定是最小元素.

ZF是集合论的公理化体系中最简单可靠的一个.当然,简单可靠的代价就是应用上的局限性.选择公理是一个饱受争议的公理,数学家们一方面对于是否应该允许“同时”进行无穷多项的构造提出了强烈的质疑,另一方面又利用它证明了很多虚无缥缈、无法构造的东西存在.有许多数学中的著名论断都需要在ZF之外再添加这条选择公理才能推推导出来.比如线性代数中任意线性空间中基的存在性,或者抽象代数中任意环的极大理想的存在性,或者实变函数(测度论)中不可测集的存在性,或者泛函分析中的Hahn-Banach定理,它们的证明过程中都直接或者间接地用到了选择公理,或者用到了在ZF中与选择公理等价的良序定理或Zorn引理.
\\ \hspace*{\fill} \\%空行
\myind{选择公理}(axiom of choice)任取一个由两两不相交的集合构成的集合族$\mathcal{A}$,存在一个集合$C$,它与$\mathcal{A}$的每个元素(元素是集合)都恰好交于一点.
\\ \hspace*{\fill} \\%空行
选择公理的一个简单推论是:任取一个集合族$\mathcal{A}$(不一定两两不交),存在一个映射$f:\mathcal{A}\mapsto\cup\mathcal{A}$,使得$\forall A\in\mathcal{A},f(A)\in A$.换言之,允许同时在集合族$\mathcal{A}$的每个元素(集合)里选择一个元素.

%--------------------------------------------
\end{document}
