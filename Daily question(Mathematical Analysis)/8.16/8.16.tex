%%%%%%%%%%%%%%%%%%%%%%%%%%%%%%%%%%%%%%%%%
% Cleese Assignment (For Students)
% LaTeX Template
% Version 2.0 (27/5/2018)
%
% This template originates from:
% http://www.LaTeXTemplates.com
%
% Author:
% Vel (vel@LaTeXTemplates.com)
%
% License:
% CC BY-NC-SA 3.0 (http://creativecommons.org/licenses/by-nc-sa/3.0/)
% 
%%%%%%%%%%%%%%%%%%%%%%%%%%%%%%%%%%%%%%%%%

%----------------------------------------------------------------------------------------
%	PACKAGES AND OTHER DOCUMENT CONFIGURATIONS
%----------------------------------------------------------------------------------------

\documentclass[12pt]{article}

\usepackage[fntef]{ctex} % invole CJKfntef
\usepackage{ctex}

\usepackage{amsmath,amssymb,amsfonts,bm,amsthm}
\usepackage{mathtools}
\usepackage{physics}
\usepackage{tensor}

\setlength\parindent{0pt} % Removes all indentation from paragraphs

\usepackage[most]{tcolorbox} % Required for boxes that split across pages

\usepackage{booktabs} % Required for better horizontal rules in tables

\usepackage{listings} % Required for insertion of code

\usepackage{etoolbox} % Required for if statements

%----------------------------------------------------------------------------------------
%	MARGINS
%----------------------------------------------------------------------------------------

\usepackage{geometry} % Required for adjusting page dimensions and margins

\geometry{
	paper=a4paper, % Change to letterpaper for US letter
	top=3cm, % Top margin
	bottom=3cm, % Bottom margin
	left=2.5cm, % Left margin
	right=2.5cm, % Right margin
	headheight=14pt, % Header height
	footskip=1.4cm, % Space from the bottom margin to the baseline of the footer
	headsep=1.2cm, % Space from the top margin to the baseline of the header
	%showframe, % Uncomment to show how the type block is set on the page
}

%----------------------------------------------------------------------------------------
%	FONT
%----------------------------------------------------------------------------------------

\usepackage[utf8]{inputenc} % Required for inputting international characters
\usepackage[T1]{fontenc} % Output font encoding for international characters

\usepackage[sfdefault,light]{roboto} % Use the Roboto font

%----------------------------------------------------------------------------------------
%	HEADERS AND FOOTERS
%----------------------------------------------------------------------------------------

\usepackage{fancyhdr} % Required for customising headers and footers

\pagestyle{fancy} % Enable custom headers and footers

\lhead{\small\assignmentClass\ifdef{\assignmentClassInstructor}{\assignmentClassInstructor}{}\ \assignmentTitle} % Left header; output the instructor in brackets if one was set
\chead{} % Centre header
\rhead{\small\ifdef{\assignmentAuthorName}{\assignmentAuthorName}{\ifdef{\assignmentDueDate}{\assignmentDueDate}{}}} % Right header; output the author name if one was set, otherwise the due date if that was set

\lfoot{} % Left footer
\cfoot{\small Page\ \thepage\ } % Centre footer
\rfoot{} % Right footer

\renewcommand\headrulewidth{0.5pt} % Thickness of the header rule
\newcommand{\myind}[1]{{\upshape\color{blue} #1 }\index{#1}}
%----------------------------------------------------------------------------------------
%	MODIFY SECTION STYLES
%----------------------------------------------------------------------------------------

\usepackage{titlesec} % Required for modifying sections

%------------------------------------------------
% Section

\titleformat
{\section} % Section type being modified
[block] % Shape type, can be: hang, block, display, runin, leftmargin, rightmargin, drop, wrap, frame
{\Large\bfseries} % Format of the whole section
{\assignmentQuestionName~\thesection} % Format of the section label
{6pt} % Space between the title and label
{} % Code before the label

\titlespacing{\section}{0pt}{0.5\baselineskip}{0.5\baselineskip} % Spacing around section titles, the order is: left, before and after

%------------------------------------------------
% Subsection

\titleformat
{\subsection} % Section type being modified
[block] % Shape type, can be: hang, block, display, runin, leftmargin, rightmargin, drop, wrap, frame
{\itshape} % Format of the whole section
{(\alph{subsection})} % Format of the section label
{4pt} % Space between the title and label
{} % Code before the label

\titlespacing{\subsection}{0pt}{0.5\baselineskip}{0.5\baselineskip} % Spacing around section titles, the order is: left, before and after

\renewcommand\thesubsection{(\alph{subsection})}

%----------------------------------------------------------------------------------------
%	CUSTOM QUESTION COMMANDS/ENVIRONMENTS
%----------------------------------------------------------------------------------------

% Environment to be used for each question in the assignment
\newenvironment{question}{
	\vspace{0.5\baselineskip} % Whitespace before the question
	\section{} % Blank section title (e.g. just Question 2)
	\lfoot{\small\itshape\assignmentQuestionName~\thesection~continued on next page\ldots} % Set the left footer to state the question continues on the next page, this is reset to nothing if it doesn't (below)
}{
	\lfoot{} % Reset the left footer to nothing if the current question does not continue on the next page
}

%------------------------------------------------

% Environment for subquestions, takes 1 argument - the name of the section
\newenvironment{subquestion}[1]{
	\subsection{#1}
}{
}

%------------------------------------------------

% Command to print a question sentence
\newcommand{\questiontext}[1]{
	\textbf{#1}
	\vspace{0.5\baselineskip} % Whitespace afterwards
}

%------------------------------------------------

% Command to print a box that breaks across pages with the question answer
\newcommand{\answer}[1]{
	\begin{tcolorbox}[breakable, enhanced]
		#1
	\end{tcolorbox}
}

%------------------------------------------------

% Command to print a box that breaks across pages with the space for a student to answer
\newcommand{\answerbox}[1]{
	\begin{tcolorbox}[breakable, enhanced]
		\vphantom{L}\vspace{\numexpr #1-1\relax\baselineskip} % \vphantom{L} to provide a typesetting strut with a height for the line, \numexpr to subtract user input by 1 to make it 0-based as this command is
	\end{tcolorbox}
}

%------------------------------------------------

% Command to print an assignment section title to split an assignment into major parts
\newcommand{\assignmentSection}[1]{
	{
		\centering % Centre the section title
		\vspace{2\baselineskip} % Whitespace before the entire section title
		
		\rule{0.8\textwidth}{0.5pt} % Horizontal rule
		
		\vspace{0.75\baselineskip} % Whitespace before the section title
		{\LARGE \MakeUppercase{#1}} % Section title, forced to be uppercase
		
		\rule{0.8\textwidth}{0.5pt} % Horizontal rule
		
		\vspace{\baselineskip} % Whitespace after the entire section title
	}
}

%----------------------------------------------------------------------------------------
%	TITLE PAGE
%----------------------------------------------------------------------------------------

\author{\textbf{\assignmentAuthorName}} % Set the default title page author field
\date{} % Don't use the default title page date field

\title{
	\thispagestyle{empty} % Suppress headers and footers
	\vspace{0.2\textheight} % Whitespace before the title
	\textbf{\assignmentClass:\ \assignmentTitle}\\[-4pt]
	\ifdef{\assignmentDueDate}{{\small Due\ on\ \assignmentDueDate}\\}{} % If a due date is supplied, output it
	\ifdef{\assignmentClassInstructor}{{\large \textit{\assignmentClassInstructor}}}{} % If an instructor is supplied, output it
	\vspace{0.32\textheight} % Whitespace before the author name
}


%----------------------------------------------------------------------------------------
%	ASSIGNMENT INFORMATION
%----------------------------------------------------------------------------------------

% Required
\newcommand{\assignmentQuestionName}{Question} % The word to be used as a prefix to question numbers; example alternatives: Problem, Exercise
\newcommand{\assignmentClass}{} % Course/class
\newcommand{\assignmentTitle}{每日一题~~\today} % Assignment title or name
\newcommand{\assignmentAuthorName}{YY} % Student name

% Optional (comment lines to remove)
\newcommand{\assignmentClassInstructor}{} % Intructor name/time/description
\newcommand{\assignmentDueDate}{} % Due date

%----------------------------------------------------------------------------------------

\begin{document}

%----------------------------------------------------------------------------------------
%	TITLE PAGE
%----------------------------------------------------------------------------------------

%\maketitle % Print the title page
\newpage

%----------------------------------------------------------------------------------------
%	QUESTION 1
%----------------------------------------------------------------------------------------

\begin{question}

\questiontext{
	设$D = \{ (x,y)|x^2+y^2\leqslant 1 \}$是单位闭圆盘.
	\begin{equation*}
		F: D\to D
	\end{equation*}
	是一个连续映射,证明: 一定存在点$p\in D$, 使得$F(p) = p$.
}

\answer{
	\begin{proof}
		我们将证明分别以下几步.
		
		首先假设$F$是二阶连续可微的映射. 如果$F$没有不动点, 则对于任意的$p\in D, F(p)\neq p$.这时, 由$F(p)$到$p$的连接射线必与$D$的边界$\partial D$交于一点$q:=G(p)$. 利用这一点我们得到一个映射\begin{equation}
			G:D\to \partial D: p\mapsto G(p).
		\end{equation}
		由定义,将$G$限制在边界$\partial D$后, $G:\partial D\to \partial D$是恒同映射. 我们希望证明这样的映射是不存在的.
		
		首先, 连接$F(p)$与$p$的射线可以表示为:\begin{equation}
			q = F(p) + t(p-F(p)), t\geqslant 0.
		\end{equation}
		这时, 对应于$G(p)$的$t$满足$t\geqslant 1$. 因而由$(G(p),G(p)) = 1$解得:
		\begin{equation}
			t = \frac{-(F(p),p-F(p))+\sqrt{(p,p-F(p))^2 - \norm{p-F(p)}^2(\norm{F(p)}^2-1)}}{\norm{p-F(p)}^2},
		\end{equation}
		因此,\begin{equation}
			\begin{split}
				G(p) &= p + \frac{-(F(p),p-F(p))+\sqrt{(p,p-F(p))^2 - \norm{p-F(p)}^2(\norm{F(p)}^2-1)}}{\norm{p-F(p)}^2} \\
				&\times (F(p)-p).
			\end{split}
			\end{equation}
		由于$F(p)-p$在$D$上处处不为$0$, 所以$p\mapsto G(p)$也是$D$上二次连续可微的映射. 我们将其表示为:
		\begin{equation}
			G:(x,y)\mapsto (g_1(x,y),g_2(x,y)).
		\end{equation}
		利用将$G$限制在边界$\partial D$后, $G:\partial D\to \partial D$是恒同映射, 我们得到:
		\begin{equation}
			\oint_{\partial D}g_1(x,y)\mathrm{d}g_2(x,y) = \oint_{\partial D}x\mathrm{d}y = \int_{0}^{2\pi}\cos^2(\theta)\mathrm{d}\theta = \pi.
		\end{equation}
		而另一方面,利用Green公式,我们得到:
		\begin{equation}
			\begin{split}
				&\oint_{\partial D}g_1(x,y)\mathrm{d}g_2(x,y) \\
				&= \oint_{\partial D}\left[ g_1(x,y)\frac{\partial g_2(x,y)}{\partial x}\mathrm{d}x + \frac{\partial g_2(x,y)}{\partial y}\mathrm{d}y  \right] \\
				&=\iint_{D} \left[ \frac{\partial g_1(x,y)}{\partial x} \frac{\partial g_2(x,y)}{\partial y} - \frac{\partial g_1(x,y)}{\partial y}\frac{\partial g_2(x,y)}{\partial x} \right]\mathrm{d}x\mathrm{d}y.
			\end{split}
		\end{equation}
		但由于$g_1^2(x,y)+g_2^2(x,y)  \equiv 1$,微分得到:
		\begin{equation}
			g_1(x,y)\frac{\partial g_1(x,y)}{\partial x} + g_2(x,y)\frac{\partial g_1(x,y)}{\partial x} = 0;
		\end{equation}
		\begin{equation}
		g_1(x,y)\frac{\partial g_1(x,y)}{\partial y} + g_2(x,y)\frac{\partial g_1(x,y)}{\partial y} = 0.
		\end{equation}
		将其看作线性方程组,则$(g_1(x,y),g_2(x,y))$是其一个非零解, 所以必须方程组的系数行列式为$0$, 即
		\begin{equation}
			\frac{\partial g_1(x,y)}{\partial x} \frac{\partial g_2(x,y)}{\partial y} - \frac{\partial g_1(x,y)}{\partial y}\frac{\partial g_2(x,y)}{\partial x} \equiv 0.
		\end{equation}
		因而, \begin{equation}
		\begin{split}
		&\oint_{\partial D}g_1(x,y)\mathrm{d}g_2(x,y) \\
		&=\iint_{D} \left[ \frac{\partial g_1(x,y)}{\partial x} \frac{\partial g_2(x,y)}{\partial y} - \frac{\partial g_1(x,y)}{\partial y}\frac{\partial g_2(x,y)}{\partial x} \right]\mathrm{d}x\mathrm{d}y = 0.
		\end{split}
		\end{equation}
		这与\begin{equation}
		\oint_{\partial D}g_1(x,y)\mathrm{d}g_2(x,y) = \pi.
		\end{equation}
		矛盾. 所以没有不动点的二次连续可微的映射$F:D\to D$是不存在的.
		
		下面, 我们希望利用Weierstrass逼近定理进一步证明没有不动点的连续映射$F:D\to D$也是不存在的.
		
		假设这样的$F$是存在的,我们将$F$表示为\begin{equation}
			F:(x,y)\mapsto (f_1(x,y),f_2(x,y)),
		\end{equation}
		其中$f_1(x,y),f_2(x,y)$都是$D$上的连续函数.利用映射\begin{equation}
			(x,y)\mapsto \left\{
			\begin{aligned}
			(x,y) &,~\text{如果} x^2+y^2\leqslant 1;\\
			\frac{(x,y)}{\sqrt{x^2+y^2}}  &,~\text{如果} x^2+y^2>1.
			\end{aligned}
			\right.
		\end{equation}
		我们就得到一个$\mathbb{R}^2\to D$的一个连续映射,其限制在$D$上后是恒同映射. 将$F$复合这一个映射, 则可将$F$看作$\mathbb{R}^2\to D$的连续映射.特别地,我们可将$f_1(x,y),f_2(x,y)$分别看作$A = [-1,1]\times [-1,1]$上的连续函数. 因而可以用多项式在$A = [-1,1]\times [-1,1]$上一致地逼近$f_1(x,y),f_2(x,y)$. 这样的逼近在$D$上也是一致的.
		
		由假设$F$没有不动点, 而$D$是一个紧集. 因而存在$\varepsilon>0$, 使得对于任意的$p\in D$,恒有$\norm{F(p)-p}>\varepsilon$. 由于我们可以用多项式在$D$上一致逼近$f_1(x,y),f_2(x,y)$, 因而可用以多项式为分量的映射$H(p)$逼近映射$F$,使得在$D$上,\begin{equation}
			\norm{H(p)-F(p)}\leqslant \frac{\varepsilon}{4}.
		\end{equation}
		这时, $\norm{H(p)}<1+\frac{\varepsilon}{4}$. 如果令\begin{equation}
			T(p) = \frac{4}{4+\varepsilon}H(p),
		\end{equation}
		则$T(p)$是从$D$射到自身的一个二次连续可微的映射,对于任意的$p\in D$,由于
		\begin{equation}
			\begin{split}
				&\norm{T(p)-F(p)}\leqslant \norm{T(p)-H(p)} +\norm{H(p)-F(p)} \\
				&\leqslant \norm{H(p)}\left| 1-\frac{4}{4+\varepsilon} \right| + \frac{\varepsilon}{2} \leqslant \frac{4+\varepsilon}{4}\left| 1-\frac{4}{4+\varepsilon} \right| + \frac{\varepsilon}{2}\leqslant \frac{3\varepsilon}{4},
			\end{split}
		\end{equation}
		我们得到\begin{equation}
			\norm{T(p)-p}\geqslant \norm{F(p)-p} - \norm{T(p)-F(p)}>\varepsilon -\frac{3\varepsilon}{4}>0.
		\end{equation}
		因此$T(p)$在$D$中没有不动点. 这与上面我们得到的$D$到自身的二次连续可微的映射一定有不动点矛盾.
	\end{proof}
}

\end{question}
%--------------------------------------------
\end{document}
